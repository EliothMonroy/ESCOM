\section{Desarrollo}
	A continuación, se presentan algunos procesos naturales donde se hace uso de la memoria:
\subsection{Aprendizaje humano}
	El aprendizaje humano en la antigüedad, se daba de forma empírica, es decir, las personas aprendían mediante la técnica del ensayo y error, si querían obtener u hacer algo, intentaban diversos medios hasta que conseguían su objetivo. En situaciones posteriores donde tenían la misma problemática, las personas usaban ese conocimiento previamente adquirido.\\ Por lo tanto, podemos observar que es una habilidad innata al ser humano el aprendizaje, para el cual, es necesario que exista la memoria, en este caso que el cerebro humano sea capaz de recordar datos los cuales sean de utilidad para un futuro. 

\subsection{Colonia de hormigas}
	Las colonias de hormigas, se caracterizan por vivir en un hormiguero donde se encuentra una gran cantidad de hormigas, las cuales trabajan para el beneficio de toda la colonia teniendo en primer lugar a la reina y sus crias. Sin embargo, existen grupos de hormigas que se encuentran designados para salir de este hormiguero en busca de comida. Estos grupos comúnmente los podemos observar en el día a día, como las hileras de hormigas, donde una gran cantidad de hormigas se unen para salir del hormiguero y forman una linea por donde se van moviendo.\\
	En caso de que una de las hormigas del frente, encuentre un objeto u algún peligro que las otras hormigas deben rodear, esa hormiga deja ferómonas en esa zona, las cuales usarán las otras hormigas para advertir que hay algo ahí y por ende, que la ruta debe ser modificada.\\ Este ejemplo de las feromonas que deja una hormiga, podemos verlo como si esto fuera un proceso natural que ocupa memoria, debido a que las ferómonas actúan como información que se encuentra disponible durante una cantidad de tiempo determinado. Cuando una hormiga llega a esta zona, puede consultar está ``información'' para así decidir si sigue la misma ruta o crea una nueva.

\subsection{Enjambre de abejas}
	Al igual que en el proceso anterior, las abejas viven en colmenas, donde se encuentran divididas según su función, existe un grupo de abejas las cuales son las encargadas de recolectar y esparcir el polen de las plantas.\\ Estás abejas recolectoras, emprenden viajes largos en los cuales visitan todo tipo de posibilidades, desde campos hasta zonas urbanas, todo con tal de sobrevivir, sin embargo, cuando las abejas detectan que el ir hacia una zona no trae buenos resultados, estás dejan de dirigirse hacia esa zona, por lo cual, podemos observar otro ejemplo de memoria, en este caso podríamos mencionarlo que es un ejemplo de memoria colectiva.

\subsection{Genética}
	Cada vez que nace una nueva generación de seres vivos, esta contiene información de las generaciones pasadas, esto aplica para todos los animales y seres humanos, en el caso de estos últimos, se puede observar que algunas características genéticas son heredadas de padres a hijos. Tales como, el color de piel, enfermedades crónicas, color de ojos y tamaño. Para que estas combinaciones sean posibles y un hijo tenga información genética por parte de sus padres, es necesario que el ADN del hijo, tenga está información contenida en sus cromosomas, de tal forma que el ADN funciona como una gran memoria que almacena información sobre las generaciones pasadas y la presente, de un ser vivo.

\subsection{Evolución}
	Conjuntamente con el proceso de la genética mencionado anteriormente, la evolución consiste en las modificaciones que sufre una especie a lo largo de un periodo muy grande de tiempo, como lo podrían ser miles de años. Así como en la genética, en la evolución podemos observar cambios muy notorios que han sufrido diferentes especies a lo largo de los años, inclusive los seres humanos no estamos exentos de esto, por mencionar un ejemplo, la perdida de los colmillos, se menciona que anteriormente lo seres humanos teníamos dientes más filosos para poder comer carne de mejor manera, situación que hoy en día a cambiado, e inclusive si ha llegado a teorizar que el ser humano en un futuro próximo, podría perder sus dientes que fungen como colmillos. \\Estás modificaciones que sufren las especies, tienen una justificación del por qué suceden, principalmente es, porque las especies necesitan adaptarse a nuevas condiciones climáticas, a un nuevo entorno o inclusive a coexistir con otras especies, lo que requiere que los seres vivos se vayan adaptando. Podemos observar que las modificaciones que sufren las especies de cierta forma, es como si la naturaleza revisará en memorias pasadas que se usaba que ahora ya no es necesario mantener, y también la naturaleza se encuentra alerta, de que experiencias han atravesado los elementos de la especie. y como podrían mejorar.