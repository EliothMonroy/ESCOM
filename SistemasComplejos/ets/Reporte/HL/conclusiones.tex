\subsection{Conclusiones}

	La hormiga de Langton es otro de ejemplo de un autómata celular bastante interesante, los comportamientos que se generan en este son bastante peculiares y complejos ya que estos tratan de describir toda una colonia. Respecto a la implementación no hubo muchos problemas, sin embargo, el rendimiento del programa no es tan bueno como se esperaba debido a que entre mayor sea el espacio a trabajar más lenta se vuelve la simulación.

	El hecho de que la probabilidad de que la hormiga que se genere sea una reina es muy baja es otro problema ya que al no nacer tantas reinas la colonia no se puede mantener.

	También cabe señalar que las hormigas tienden a realizar un patrón que se repita indeterminadamente, y mientras más hormigas estén cerca, más probable que se llegue a este patrón antes.