\subsection{Desarrollo}
	
	El desarrollo de estos programas, se llevo acabo usando Javascript, debido a que el entorno web y del navegador permiten que se puedan desarrollar interfaces para usuario de una forma bastante eficiente. Además, los programas que requieren de usar animaciones o que hacen un uso extensivo de dibujar en un canvas (como es el caso), se ven beneficiados del entorno web que permite todo se ejecute más rápido.

	\subsubsection{Hormiga de Langton Original}
		La parte de presentación para el usuario, se encuentra en el siguiente archivo html.
		
		\paragraph{Archivo: }  hormiga-original.html
		\lstinputlisting[language=HTML]{../../HormigaLangton/hormiga-original.html}

		La lógica con la que funciona el programa, se encuentra en el siguiente archivo de Javascript.
		
		\paragraph{Archivo: }  hormiga-original.js
		\lstinputlisting[language=Java]{../../HormigaLangton/hormiga-original.js}

	\subsubsection{Hormiga de Langton Modificada}
		La parte de presentación para el usuario, se encuentra en el siguiente archivo html.

		\paragraph{Archivo: }  hormiga.html
		\lstinputlisting[language=HTML]{../../HormigaLangton/hormiga.html}

		La lógica con la que funciona el programa, se encuentra en el siguiente archivo de Javascript.
		
		\paragraph{Archivo: }  hormiga.js
		\lstinputlisting[language=Java]{../../HormigaLangton/hormiga.js}