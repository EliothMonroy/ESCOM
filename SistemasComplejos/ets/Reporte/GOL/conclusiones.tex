\subsection{Conclusiones}
	Al hacer y probar este programa se pudieron solucionar dos cuestiones, la primera es si existe una configuración el la cual el numero de células no se termine, y la respuesta a esto es que exista un oscilador el cual cambia su posición a lo largo del tiempo, otra posible opción es que haya figuras como un bloque formado por 4 cuadros en el cual no hay cambios.

	La siguiente cuestión es si existe una configuración en la cual la población crezca indefinidamente. Para lograr esto es indispensable tener un espacio que sea infinito en el cual se puedan propagar las células a través del tiempo, ya que si no se cuenta con esto en algún punto se tendrán tantos elementos vivos que empezaran a morir por sobre población.