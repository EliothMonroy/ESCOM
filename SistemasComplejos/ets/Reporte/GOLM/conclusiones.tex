\subsection{Conclusiones}

	El uso de una función y una matriz auxiliar para el calculo de las iteraciones en un autómata celular cambian el comportamiento de la función original, es decir, si se realiza la prueba y se compara la gráfica de unos de la regla del juego de la vida con y sin función auxiliar se puede apreciar que la que tiene la función auxiliar oscila con más frecuencia a diferencia de la que no utiliza una función extra, en esta la gráfica se estabiliza de una forma más rápida.

	El uso de esta técnica al trabajar con autómatas celulares es bastante útil ya que permite el observar nuevos comportamientos con base a reglas ya conocidas, lo cual puede agilizar el estudio de estos modelos. Además de que nos acerca a modelar de forma un poco más realista los procesos del universo.