\subsection{Introducción}
	Los autómatas celulares(AC) surgen en la década de 1940 con John Von Neumann, que intentaba modelar una máquina que fuera capaz de auto-replicarse, llegando así a un modelo matemático de dicha maquina con reglas complicadas sobre una red rectangular. Inicialmente fueron interpretados como conjunto de células que crecían, se reproducían y morían a medida que pasaba el tiempo. A esta similitud con el crecimiento de las células se le debe su nombre.\cite{PAGINA}

	Un autómata celular se caracteriza por contar con los siguientes elementos:
	\begin{itemize}
		\item Arreglo regular. Ya sea un plano de dos dimensiones o un espacio n-dimensional, este es el espacio de evoluciones, y cada división homogénea del arreglo es llamada célula.
		\item Conjunto de estados. Es finito y cada elemento o célula del arreglo toma un valor de este conjunto de estados. También se denomina alfabeto. Puede ser expresado en valores o colores.
		\item configuración inicial. Consiste en asignar un estado a cada una de las células del espacio de evolución inicial del sistema.
		\item Vecindades. Define el conjunto contiguo de células y posición relativa respecto a cada una de ellas. A cada vecindad diferente corresponde un elemento del conjunto de estados.
		\item Función local. Es la regla de evolución que determina el comportamiento del A. C. Se conforma de una célula central y sus vecindades. Define como debe cambiar de estado cada célula dependiendo de los estados anteriores de sus vecindades. Puede ser una expresión algebraica o un grupo de ecuaciones.
	\end{itemize}

\subsection{Planteamiento de la práctica}
	Este programa implementa la simulación de un autómata celular. Como fue expuesto en la sección 2 de este mismo documento, con la diferencia que este autómata celular implementa una nueva caracteristica, la de tener ``memoria''.

	Se cuenta con la opción para utilizar una matriz auxiliar junto con una función $\Phi$ para realizar el calculo de las matrices a través de las generaciones futuras. Esta opción consiste en lo siguiente.

	\begin{enumerate}
		\item Se elige una función $\Phi$ la cual puede ser de paridad, máximo o mínimo. También, se asigna un valor de $\tau$ el cual debe de ir de 3 a 8.
		\item Después de $\tau$ iteraciones se utiliza la función $\Phi$ para obtener una matriz auxiliar, la función $\Phi$ utiliza como parámetros los valores de las $\tau$ matrices anteriores.
		\item A la matriz auxiliar se le debe de aplicar la regla del autómata celular que se ha estado utilizado para calcular las iteraciones por lo que ahora tendremos una nueva matriz.
		\item Esto se repite las veces que se desee.
	\end{enumerate}