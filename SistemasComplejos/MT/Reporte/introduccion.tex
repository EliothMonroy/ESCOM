\section{Introducción}
	Las máquinas de Turing son un dispositivo que permite idealmente resolver cualquier problema, debido a que tiene una memoria infinita, sin embargo está implementación no es posible de realizar y deben ser acotadas a una memoria o cinta finita, de tal forma que se dice que si una máquina de Turing es capaz de resolver un problema, este problema es computable (quiere decir que una computadora lo puede resolver). En caso contrario, se dice que el problema no es computable.\cite{LIBRO}

	El ejercicio a realizar en está actividad, es diseñar e implementar una máquina de Turing que duplique la cantidad de unos de una cadena ingresada, y que además, se muestre el procedimiento en una animación y se escriba en un archivo el historial de los pasos realizados por la máquina de Turing.

