\newpage
\chapter{Pruebas de Desempeño}
\section{Aspectos Generales}
\noindent
Para sistemas de tiempo real es inaceptable que el software se apegue a la funcionalidad requerida pero sin tomar en cuenta la parte del rendimiento del mismo. Por este motivo existen las pruebas de rendimiento que se diseñan para poner a prueba dicha característica dentro del marco de un sistema integrado. Según Pressman \cite{press} \textit{"Las pruebas de rendimiento requieren de instrumentación de hardware y software"}, podemos interpretar esto como que es necesario medir la utilización de los recursos (memoria, ciclos del procesador, etc) de forma meticulosa.
Las pruebas de rendimiento se usan para descubrir problemas de rendimiento que pueden ser resultado de: falta de recursos, red con ancho de banda o conexión inadecuada, capacidades de base de datos inadecuadas, capacidades de sistema operativo deficientes o débiles, funcionalidad del sistema diseñado y otros conflictos de hardware o software que pueden conducir a un rendimiento degradado. La intención es doble:
\begin{enumerate}
    \item Comprender cómo responde el sistema conforme aumenta la carga(es decir, número de usuarios, número de transacciones o volumen de datos global)
    \item Recopilar mediciones que conducirán a 
modificaciones de diseño para mejorar el rendimiento
\end{enumerate}
\section{Objetivos}
\noindent
Las pruebas de rendimiento se diseñan para simular situaciones de carga del mundo real. Conforme aumenta el número de usuarios simultáneos del sistema el número de transacciones en línea o la cantidad de datos (descargados o subidos), las pruebas de rendimiento ayudarán a responder las siguientes preguntas:
\begin{itemize}
    \item ¿En qué punto (en términos de usuarios, transacciones o carga de datos) el rendimiento 
se vuelve inaceptable?
    \item ¿Qué componentes del sistema son responsables de la degradación del rendimiento?
    \item ¿Cuál es el tiempo de respuesta promedio para los usuarios bajo diversas condiciones de 
carga?
    \item ¿La degradación del rendimiento tiene impacto sobre la seguridad del sistema?
\end{itemize}
Para desarrollar respuestas a estas preguntas, se realizan dos tipos diferentes de pruebas de rendimiento, la prueba de carga examina la carga del mundo real en varios niveles de carga y en varias combinaciones, por otro lado, la prueba de esfuerzo fuerza a aumentar la carga hasta el punto de rompimiento para determinar cuánta capacidad puede manejar el entorno del sistema.
\section{Estrategias de Prueba}
\subsection{Prueba de carga}
\noindent
La intención de la prueba de carga es determinar cómo responderá el sistema y su entorno a varias condiciones de carga. Conforme avanzan las pruebas, las permutas de las siguientes variables definen un conjunto de condiciones de prueba:
\begin{itemize}
    \item \textbf{N} Número de usuarios concurrentes
    \item \textbf{T} Número de transacciones en línea por unidad de tiempo
    \item \textbf{D} Carga de datos procesados por el servidor en cada transacción
\end{itemize}
En todo caso, dichas variables se definen dentro de fronteras operativas normales del sistema. Conforme se aplica cada condición de prueba, se recopila una o más de las siguientes medidas: respuesta de usuario promedio, tiempo promedio para descargar una unidad estandarizada de datos y tiempo promedio para procesar una transacción.\\
La prueba de carga también puede usarse para valorar las velocidades de conexión recomendadas para los usuarios del sistema. El rendimiento global, P, se calcula de la forma siguiente:
\begin{equation}
    P= N * T * D
\end{equation}
\subsection{Prueba de esfuerzo}
\noindent
La prueba de esfuerzo es una continuación de la prueba de carga, pero en esta instancia las variables N, T y D se fuerzan a satisfacerse y luego se superan los límites operativos. La intención de estas pruebas es responder a cada una de las siguientes preguntas:
\begin{itemize}
    \item ¿Las transacciones se pierden conforme la capacidad se excede?
    \item ¿La integridad de los datos resulta afectada conforme la capacidad se excede?
    \item Si el sistema falla, ¿cuánto tiempo tardará en regresar en línea?
    \item ¿Ciertas funciones del sistema quedan descontinuadas conforme la capacidad alcanza el nivel de 80 o 90 por ciento?
\end{itemize}
A una variación de las pruebas de esfuerzo en ocasiones se le conoce como prueba pico/rebote, en este régimen de pruebas, la carga alcanza un pico de capacidad, 
luego se baja rápidamente a condiciones operativas normales y después alcanza de nuevo un pico. Al rebotar la carga del sistema, es posible determinar cuán bien el servidor puede ordenar los recursos para satisfacer una demanda muy alta y entonces liberarlos cuando reaparecen condiciones normales.