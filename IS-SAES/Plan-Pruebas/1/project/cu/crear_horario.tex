%Elioth Monroy El Chido
\section{Prueba P21: Caso de Uso Crear Horario}
\begin{longtable}{ | p{9cm} | p{.5cm} | p{.5cm} | p{5cm} | }
\hline
\textbf{Pregunta} & \textbf{Si} & \textbf{No} & \textbf{Observaciones}\\
\hline
\multicolumn{4}{| p{15cm}| }{1. Inicie sesión en el sistema con los siguientes datos: \begin{itemize}
		\item Usuario: \textbf{2015631008}
		\item Contraseña: \textbf{Juanito1}
	\end{itemize}
	 2. En el menú izquierdo seleccione la opción <<Reinscripción>> y posteriormente de click en <<Crear horario>>.}\\
	 \hline
	 2.1. ¿El sistema mostró la pantalla 3.49 Crear Horario 1? & & &\\
	\hline
	\multicolumn{4}{| p{15cm}| }{
	2.2. Evalúe la pantalla tomando como referencia 3.49 Crear Horario 1:
    \begin{checklist}
        \item Estilos CSS.
        \item Ortografía.
        \item Iconografía.
        \item Alineación.
    \end{checklist}}\\
	\hline
	 \multicolumn{4}{| p{15cm}| }{
	 3. En el campo <<Buscar materias>> ingrese el siguiente valor: \textbf{1CM10}, presionar el botón <<Agregar>> para la materia de <<Cálculo>>.}\\
\hline
3.1. ¿El sistema mostró correctamente el mensaje de error ME.15? & & &\\
\hline
3.2. ¿El sistema evitó que la materia fuera agregada al horario? & & &\\
\hline
\multicolumn{4}{| p{15cm}| }{4. Busque el grupo <<1CM1>> y agregue la materia de <<Física>>.\newline Posteriormente, busque el grupo <<1CM2>> y agregue la materia de <<Cálculo>>}\\
\hline
4.1. ¿El sistema mostró correctamente el mensaje de error ME.16? & & &\\
\hline
4.2. ¿El sistema evitó que la materia fuera agregada al horario? & & &\\
\hline
\multicolumn{4}{| p{15cm}| }{5. Busque el grupo <<1CM2>> y agregue la materia de <<Física>>}\\
\hline
5.1 ¿El sistema mostró correctamente el mensaje de error ME.17? & & &\\
\hline
5.2 ¿El sistema evitó que la materia fuera agregada al horario? & & &\\
\hline
\multicolumn{4}{| p{15cm}| }{6. Busque el grupo <<1CM1>> y agregue las materias de <<Comunicación Oral y Escrita>> y <<Análisis Vectorial>>.\newline Posteriormente presione el botón <<Guardar horario>>}\\
\hline
	 6.1. ¿El sistema mostró la pantalla 3.50 Crear Horario 2? & & &\\
	\hline
	\multicolumn{4}{| p{15cm}| }{
	6.2. Evalúe la pantalla tomando como referencia 3.50 Crear Horario 2:
    \begin{checklist}
        \item Estilos CSS.
        \item Ortografía.
        \item Iconografía.
        \item Alineación.
    \end{checklist}}\\
	\hline
\multicolumn{4}{| p{15cm}| }{7. Posteriormente, como nombre del horario escriba \textbf{Horario malo}.\newline Presione <<Registrar>>.}\\
\hline
7.1. ¿El sistema mostró correctamente el mensaje de error ME.25? & & &\\
\hline
7.2. ¿El sistema evitó que el horario fuera guardado? & & &\\
\hline
\multicolumn{4}{| p{15cm}| }{8. Presione el botón de cerrar de la ventana que quedo abierta en el paso anterior. \newline Busque el grupo <<1CM1>> y agregue todas la materias restantes al horario. \newline Posteriormente, busque el grupo <<2CV1>> e igualmente agregue todas las materias de este grupo al horario. \newline Posteriormente, presione <<Guardar horario>> y como nombre del horario escriba \textbf{Horario malo}. \newline Presione <<Registrar>>.}\\
\hline
8.1 ¿El sistema mostró correctamente el mensaje de error ME.25? & & &\\
\hline
8.2 ¿El sistema evitó que el horario fuera guardado? & & &\\
\hline
\multicolumn{4}{| p{15cm}| }{9. Presione el botón de cerrar de la ventana que quedo abierta en el paso anterior.\newline Elimine todas las materias pertenecientes al grupo <<2CV1>> (Verifique que en el horario queden solamente las seis materias del grupo <<1CM1>>).\newline Posteriormente, presione <<Guardar horario>> y como nombre del horario escriba \textbf{Horario bueno}.\newline Presione <<Registrar>>.}\\
\hline
9.1 ¿El sistema mostró correctamente el mensaje de confirmación MC.7? & & &\\
\hline
9.2 ¿El sistema guardó el horario creado? & & &\\
\hline
\multicolumn{4}{| p{15cm}| }{\textbf{Fin de la Prueba}} \\
\hline
\end{longtable}