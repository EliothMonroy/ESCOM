\section{Prueba P6: Caso de Uso Registrar materia}
\begin{longtable}{ | p{9cm} | p{.5cm} | p{.5cm} | p{5cm} | }
\hline
\textbf{Pregunta} & \textbf{Si} & \textbf{No} & \textbf{Observaciones}\\
\hline
\multicolumn{4}{| p{15cm}| }{1. Acceda a la opción <<Registro>> del menú izquierdo y de clic en la opción <<Registrar Grupo>>}\\
\hline
\multicolumn{4}{| p{15cm}| }{
		1.1. Evalúe la pantalla tomando como referencia 3.27 Registrar Materia.
    \begin{checklist}
        \item Estilos CSS.
        \item Ortografía.
        \item Iconografía.
        \item Alineación.
    \end{checklist}}\\
\hline
\multicolumn{4}{| p{15cm}| }{2. Dé clic en el botón enviar sin ingresar datos.}\\
\hline
2.1 ¿El sistema evitó el registro de la materia? & & &\\
\hline
2.2 ¿El sistema mostró correctamente el error ME.37? & & &\\
\hline
\multicolumn{4}{| p{15cm}| }{3. Inicie el proceso de registrar materia ingresando los siguientes datos (\textbf{X}):\begin{itemize}
    \item Nombre: Inteligencia Artificial
    \item Créditos: 4.39
    \item Nivel: 3
    \item Área: Formación Profesional
    \item Cupo: 25
\end{itemize}} \\
\hline
3.1 ¿El sistema mostró el mensaje de confirmación MC.1? & & &\\
\hline
3.2 ¿El sistema redirigió la página a esta misma pantalla? & & &\\
\hline
\multicolumn{4}{| p{15cm}| }{\textbf{Fin de la Prueba}} \\
\hline
\end{longtable}