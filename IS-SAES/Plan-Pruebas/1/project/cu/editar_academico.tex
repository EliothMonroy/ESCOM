%Elioth Monroy El Chido
\section{Prueba P13: Caso de Uso Editar Académico}
\begin{longtable}{ | p{9cm} | p{.5cm} | p{.5cm} | p{5cm} | }
\hline
\textbf{Pregunta} & \textbf{Si} & \textbf{No} & \textbf{Observaciones}\\
\hline
\multicolumn{4}{| p{15cm}| }{1. Inicie sesión en el sistema con los siguientes datos: \begin{itemize}
		\item Usuario: \textbf{CAGM700215GS7}
		\item Contraseña: \textbf{Escom1234}
	\end{itemize}
	2. En el menú izquierdo seleccione la opción <<Administrar>> y posteriormente de click en <<Académico>>.}\\
	 \hline
    \hline
    2.1 ¿El sistema mostró la pantalla 3.13 Administrar Académico? & & &\\
    \hline
    \multicolumn{4}{| p{15cm}| }{
	2.2. Evalúe la pantalla tomando como referencia 3.13 Administrar Académico:
    \begin{checklist}
        \item Estilos CSS.
        \item Ortografía.
        \item Iconografía.
        \item Alineación.
    \end{checklist}}\\
	\hline
	 \multicolumn{4}{| p{15cm}| }{
	 3. En el campo <<Buscar>> ingrese el siguiente RFC: \textbf{ABAJ850330LK9} y presione <<Buscar>>.\newline
	 4. Posteriormente, al visualizar el resultado de la búsqueda, presione el botón <<Editar>>.}\\
	 \hline
		4.1 ¿El sistema mostró la pantalla 3.14 Editar Académico? & & &\\
	\hline
	\multicolumn{4}{| p{15cm}| }{
		4.2. Evalúe la pantalla tomando como referencia 3.14 Editar Académico:
    \begin{checklist}
        \item Estilos CSS.
        \item Ortografía.
        \item Iconografía.
        \item Alineación.
    \end{checklist}}\\
	\hline
	\multicolumn{4}{| p{15cm}| }{
	 5. Llene el formulario con datos los siguientes datos y presione el botón <<Guardar>>: \begin{itemize}
    \item Nombre: Juan Manuel 10
		\item Apellido paterno: Abalos
		\item Apellido materno: Abad
		\item Número de empleado: 1346985249
		\item Correo electrónico: abaj5@ipn.mx
		\item Contraseña: AbadJM1
	\end{itemize}
	 }\\
\hline
5.1 ¿El sistema evitó que se guardaran los datos? & & &\\
\hline
5.2 ¿El sistema mostró correctamente el mensaje de error ME.38? & & &\\
\hline
	\multicolumn{4}{| p{15cm}| }{6. Ingrese ahora los siguientes datos y presione <<Guardar>>:\begin{itemize}
			\item Nombre: Juan Manuel 
			\item Apellido paterno: Abalos
			\item Apellido materno: Abad
			\item Número de empleado: 1346985249
			\item Correo electrónico: estebanp@ipn.mx
			\item Contraseña: AbadJM1
	\end{itemize}} \\
\hline
6.1 ¿El sistema evitó que se guardaran los datos? & & &\\
\hline
6.2 ¿El sistema mostró correctamente el mensaje de error ME.2? & & &\\
\hline
\multicolumn{4}{| p{15cm}| }{7. Ingrese ahora los siguientes datos y presione <<Guardar>>:\begin{itemize}
		\item Nombre: Juan Manuel 
		\item Apellido paterno: Abalos
		\item Apellido materno: Abad
		\item Número de empleado: 1215451212
		\item Correo electrónico: abaj5@ipn.mx
		\item Contraseña: AbadJM1
\end{itemize}} \\
\hline
7.1 ¿El sistema evitó que se guardaran los datos? & & &\\
\hline
7.2 ¿El sistema mostró correctamente el mensaje de error ME.4? & & &\\
\hline
\multicolumn{4}{| p{15cm}| }{8. Ingrese ahora los siguientes datos y presione <<Guardar>>:\begin{itemize}
		\item Nombre: Juan
		\item Apellido paterno: Abalos
		\item Apellido materno: Abad
		\item Número de empleado: 1346985249
		\item Correo electrónico: abaj5@ipn.mx
		\item Contraseña: AbadJM1
\end{itemize}} \\
\hline
8.1 ¿El sistema permitió que se guardaran los datos? & & &\\
\hline
8.2 ¿El sistema mostró correctamente el mensaje de confirmación MC.3? & & &\\
\hline
\multicolumn{4}{| p{15cm}| }{\textbf{Fin de la Prueba}} \\
\hline
\end{longtable}