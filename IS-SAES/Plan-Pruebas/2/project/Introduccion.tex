\newpage
\chapter{Descripción de la Prueba}
\section{Propósito de la prueba}
\noindent
El presente documento tiene como objetivo la ejecución de las pruebas sobre el Sistema de Administración Escolar (en adelante SAES) con base en la especificación técnica.
\section{Alcance}
\noindent
Las pruebas que se presentan a continuación son pruebas unitarias de cada uno de los casos de uso del SAES.
\begin{itemize}
    \item Inicios de sesión en los diferentes perfiles.
    \item Registro de usuarios.
    \item Registro de materias.
    \item Registro de profesores.
    \item Registro de grupos.
    \item Generación de citas de inscripción.
    \item Reinscripción del alumno.
    \item Creación y almacenamiento de horarios.
    \item Edición de registros.
    \item Etcétera
\end{itemize}
\section{Elementos involucrados}
\noindent
Como se mencionó anteriormente se trabajará con todos los Casos de Uso que conforman el sistema, por lo tanto se verificarán todas las pantallas, así como los mensajes de alerta, confirmación y error que muestra el sistema. Los mensajes así como las pantallas del sistema se pueden encontrar en la sección de Anexos.
\section{Requerimientos}
\noindent
Las pruebas se ejecutarán en el laboratorio de la Escuela Superior de Cómputo haciendo uso de los equipos que se encuentran ahí. Es necesario tener las computadoras conectadas a una red local y que cada una de ellas tenga conexión con un equipo que será designado como el servidor. El navegador a ocupar será \textit{Google Chrome}.
\begin{longtable}{ | p{8cm} | p{7cm} | }
\hline
\multicolumn{2}{|p{15cm}|}{\textbf{Datos de Prueba}}\\
\hline
\textbf{Tipo de Prueba}: & Completa\\
\hline
\textbf{Fecha de Aplicación}: & \\
\hline
\textbf{Hora Inicio}: & \\
\hline
\textbf{Hora Final}: & \\
\hline
\textbf{Nombre del líder de pruebas}: & \\
\hline
\textbf{Nombre del tester}: & \\
\hline
\end{longtable}
\section{Checklist}
\noindent
Para poder dar inicio con las pruebas compruebe que cuenta con los siguientes puntos: \begin{itemize}
    \item Lápiz.
    \item Guión de Pruebas.
    \item El equipo cuenta con conexión al servidor.
    \item El equipo cuenta con el navegador Google Chrome.
\end{itemize}
\section{Instrucciones}
\begin{itemize}
    \item Verifique que cuenta con conexión al servidor de pruebas.
    \item Abra el navegador Google Chrome.
    \item Ingrese la siguiente dirección: 
\end{itemize}