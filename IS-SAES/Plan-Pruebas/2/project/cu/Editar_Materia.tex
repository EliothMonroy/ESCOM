\section{Prueba P15: Caso de Uso Editar Materia}
\begin{longtable}{ | p{9cm} | p{.5cm} | p{.5cm} | p{5cm} | }
	\hline
	\textbf{Pregunta} & \textbf{Si} & \textbf{No} & \textbf{Observaciones}\\
	\hline
	\multicolumn{4}{| p{15cm}| }{1. Inicie sesión en el sistema con los siguientes datos: \begin{itemize}
			\item Usuario: \textbf{GUMO7702172S8}
			\item Contraseña: \textbf{Oscar1}
		\end{itemize}
		2. En el menú izquierdo seleccione la opción <<Administrar>> y posteriormente de click en <<Materia>>.\newline
		3. En el campo <<Buscar>> ingrese el siguiente nombre: \textbf{Redes de computadoras} y presione <<Buscar>>.\newline
		4. Posteriormente, al visualizar el resultado de la búsqueda, presione el botón <<Editar>>.\newline
		5. En la pantalla 3.18 Administrar Materia, llene el formulario con datos inválidos y presione el botón <<Guardar>>}\\
	\hline
	5.1 ¿El sistema evitó que se guardaran los datos? & & &\\
	\hline
	5.2 ¿El sistema mostró correctamente el mensaje de error ME.2? & & &\\
	\hline
	\multicolumn{4}{| p{15cm}| }{6. Ingrese ahora los siguientes datos y presione <<Guardar>>:\begin{itemize}
			\item Nombre: Redes de las computadoras 
			\item Nivel: 2
			\item Créditos: 14
			\item Clasificación: Formación profesional
	\end{itemize}} \\
	\hline
	6.1 ¿El sistema evitó que se guardaran los datos? & & &\\
	\hline
	6.2 ¿El sistema mostró correctamente el mensaje de error ME.3? & & &\\
	\hline
	\multicolumn{4}{| p{15cm}| }{7. Ingrese ahora los siguientes datos y presione <<Guardar>>:\begin{itemize}
			\item Nombre: Redes de las computadoras 
			\item Nivel: 2
			\item Créditos: 5
			\item Clasificación: Formación profesional
	\end{itemize}} \\
	\hline
	7.1 ¿El sistema permitió que se guardaran los datos? & & &\\
	\hline
	7.2 ¿El sistema mostró correctamente el mensaje de confirmación MC.1? & & &\\
	\hline
	\multicolumn{4}{| p{15cm}| }{\textbf{Fin de la Prueba}} \\
	\hline
\end{longtable}