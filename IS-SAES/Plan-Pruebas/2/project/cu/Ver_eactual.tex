\newpage
\section{Prueba P25: Caso de Uso Ver estado actual}
\begin{longtable}{ | p{9cm} | p{.5cm} | p{.5cm} | p{5cm} | }
\hline
\textbf{Pregunta} & \textbf{Si} & \textbf{No} & \textbf{Observaciones}\\
\hline
\multicolumn{4}{| p{15cm}| }{1. Inicie sesión con los siguientes datos \begin{itemize}
    \item \textbf{No. Boleta:} 2016630194
    \item \textbf{Contraseña:} Gabriel12
\end{itemize}}\\
\hline
\multicolumn{4}{| p{15cm}| }{
		1.1. Evalúe la pantalla tomando como referencia 3.46 Datos Generales: Alumno.
    \begin{checklist}
        \item Estilos CSS.
        \item Ortografía.
        \item Iconografía.
        \item Alineación.
    \end{checklist}}\\
\hline
1.2 ¿El sistema muestra los siguientes datos correctamente? \begin{itemize}
    \item \textbf{No. Boleta:} 2016630194
    \item \textbf{Nombre:} Gabriel Alejandro
    \item \textbf{Créditos obtenidos:} Hutrón
    \item \textbf{Porcentaje de créditos:} Rizo
    \item \textbf{Créditos por obtener:} gabihr@hotmail.com
    \item \textbf{Materias reprobadas:} Regular
    \item \textbf{Promedio:} Regular
\end{itemize} & & &\\
\hline
1.3 En la tabla que se muestra debajo, la lista de claves de materias ¿va de C101 a C112, de C213 a 221 y de C223 a C226? & &  &\\
\hline
\multicolumn{4}{| p{15cm}| }{\textbf{Fin de la Prueba}} \\
\hline
\end{longtable}