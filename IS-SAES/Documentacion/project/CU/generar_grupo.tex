\newpage
\section{Caso de Uso Generar Grupo}
\begin{longtable}{ | p{6cm} | p{10cm} |}
\hline
\textbf{NOMBRE} & Generar Grupo\\
\hline
\textbf{TIPO} & Primario\\
\hline
\textbf{DESCRIPCIÓN} & El analista registra los datos de un grupo nuevo en un formulario, asentándolo en el sistema.\\
\hline
\textbf{ENTRADAS} & Nivel: Selección de una lista numérica (entre 1 y 5)\\ & Turno: Selección de una lista (Matutino/Vespertino)\\ & Identificador: Entero Positivo\\
\hline
\textbf{SALIDAS} & Si el grupo fue dado de alta de forma exitosa se muestra un mensaje de confirmación.\\ & Si ocurrió algún error durante su registro, se muestra un mensaje con el error correspondiente.\\
\hline
\textbf{PRECONDICIONES} & El analista de gestión debe haber iniciado sesión.\\ & Es periodo de registro de grupos.\\
\hline
\textbf{POSTCONDICIONES} & El sistema se actualiza con los datos ingresados. Se muestra un mensaje de confirmación para avisar que el registro del grupo fue correcto.\\
\hline
\textbf{SITUACIONES DE ERROR} & \begin{itemize}
    \item El grupo ya ha sido registrado.
    \item Se dejaron campos obligatorios en blanco.
    \item El formato de los datos es inválido
\end{itemize}\\
\hline
\textbf{ESTADO DEL SISTEMA EN CASO DE ERROR} & El sistema no se actualiza.\\
\hline
\textbf{ACTORES} & Analista, Jefe de Gestión Escolar\\
\hline
\textbf{MENSAJES} & \textbf{Mensajes de Confirmación}: Se presentan en los siguientes casos: \begin{itemize}
    \item En caso de que se haya registrado exitosamente el nuevo grupo: MC.1.
\end{itemize}\\ & \textbf{Mensajes de Error}: Se presentan en los siguientes casos: \begin{itemize}
    \item El grupo ya ha sido registrado: ME.33
    \item Se dejaron campos obligatorios en blanco: ME.1
\end{itemize}\\
\hline
\textbf{AUTOR} & Medina Luqueño Ana Ximena\\
\hline
\end{longtable}
\vspace*{1cm}
\noindent
\Large{PROCEDIMIENTO ESTÁNDAR}
\large{}
\begin{enumerate}
    \item El usuario ingresa al sistema como analista de gestión escolar.
    \item El usuario selecciona la opción “Registrar grupo” en el menú de opciones.
    \item *El sistema carga el formulario para registrar grupos.
    \item El usuario selecciona el nivel y turno del grupo.
    \item El usuario ingresa el identificador del grupo.
    \item *El sistema valida que ninguno de los campos este vacío. \textbf{[Trayectoria 1]}
    \item *El sistema valida que los datos en el campo de identificador sean válidos. \textbf{[Trayectoria 2]}
    \item *El sistema valida que el grupo no haya sido registrado antes. \textbf{[Trayectoria 3]}
    \item *El sistema muestra el mensaje de confirmación correspondiente.
    \item *El sistema redirige a la pantalla de Registrar grupos.
    \item Fin del caso de uso.
\end{enumerate}
\vspace*{1cm}
\Large{PROCEDIMIENTOS ALTERNATIVOS}\\
\large{Trayectoria 1}\\
\textbf{Condición}: El usuario envió el formulario con algún campo vacío.
\begin{enumerate}
    \item *El sistema muestra el mensaje de error ME.1.
    \item Regresa al paso 4 de la trayectoria principal.
    \item Fin de la trayectoria.
\end{enumerate}
\large{Trayectoria 2}\\
\textbf{Condición}: Los datos en el campo de identificador son inválidos.
\begin{enumerate}
    \item *El sistema muestra el mensaje de error ME.2.
    \item Regresa al paso 5 de la trayectoria principal.
    \item Fin de la trayectoria.
\end{enumerate}
\large{Trayectoria 3}\\
\textbf{Condición}: El grupo ya ha sido registrado.
\begin{enumerate}
    \item *El sistema muestra el mensaje de error ME.33.
    \item Regresa al paso 4 de la trayectoria principal.
    \item Fin de la trayectoria.
\end{enumerate}