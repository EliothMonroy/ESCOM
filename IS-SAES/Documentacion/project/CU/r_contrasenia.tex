\newpage
\section{Caso de Uso Reestablecer contraseña}
\begin{longtable}{ | p{6cm} | p{10cm} |}
\hline
\textbf{NOMBRE} & Reestablecer contraseña\\
\hline
\textbf{TIPO} & Primario\\
\hline
\textbf{DESCRIPCIÓN} & El usuario solicita al sistema restaurar su contraseña en caso de haberla olvidado.\\
\hline
\textbf{ENTRADAS} & Usuario: \begin{itemize}
    \item En caso de ser un alumno, su número de boleta: Entero positivo de 10 dígitos con el formato de número de boleta.
    \item En caso de ser un académico o analista de gestión escolar, su RFC: Cadena de 13 caracteres con el formato de RFC.
\end{itemize}\\
\hline
\textbf{SALIDAS} & El sistema enviará un correo con un hipervínculo de restauración de contraseña a la dirección de correo al que está vinculada la cuenta del usuario.\\ & El sistema mostrará el mensaje de alerta, indicando que el correo ha sido enviado a la dirección de correo vinculada.\\
\hline
\textbf{PRECONDICIONES} & El usuario debe estar registrado en el sistema.\\
\hline
\textbf{POSTCONDICIONES} & Se inicia el caso de uso “Cambiar contraseña”.\\
\hline
\textbf{SITUACIONES DE ERROR} & Se deja el campo de usuario vacío.\\ & Se quiere reestablecer la contraseña de un usuario que no se encuentra en el sistema.\\
\hline
\textbf{ESTADO DEL SISTEMA EN CASO DE ERROR} & No se envía ningún correo con hipervínculo de restauración de contraseña y se muestra un mensaje de error. \\
\hline
\textbf{ACTORES} & Alumno, Académica, Analista\\
\hline
\textbf{MENSAJES} & \textbf{Mensajes de Confirmación}: Se presentan en los siguientes casos: \begin{itemize}
    \item Cuando el usuario ingresó un número de boleta válido y registrado en el sistema: MC.2
    \item Cuando el usuario ingresó un RFC válido y registrado en el sistema: MC.2
\end{itemize}\\
\hline
\end{longtable}
\newpage
\begin{longtable}{ | p{6cm} | p{10cm} |}
\hline
\textbf{MENSAJES} & \textbf{Mensajes de Error}: Se presentan en los siguientes casos: \begin{itemize}
    \item Cuando el usuario deja el campo de identificador (número de boleta o RFC) vacío: ME.1
    \item Cuando el usuario ingresa un identificador (número de boleta o RFC) no válido o no registrado en el sistema: ME.6
\end{itemize}\\
\hline
\textbf{AUTOR} & Medina Luqueño Ana Ximena\\
\hline
\end{longtable}
\vspace*{1cm}
\noindent
\Large{PROCEDIMIENTO ESTÁNDAR}
\large{}
\begin{enumerate}
    \item El usuario ingresa al sistema.
    \item Cargada la pantalla de inicio, selecciona la opción de iniciar sesión.
    \item Se carga el formulario de login y el usuario selecciona la opción <<Reestablecer contraseña>>.
    \item Se carga el formulario de restablecimiento de contraseña y el usuario ingresa su identificador (número de boleta para alumno y RFC para académico o analista de gestión escolar.
    \item El usuario selecciona el botón de Reestablecer Contraseña.
    \item *El sistema valida que el campo de usuario no esté vacío. \textbf{[Trayectoria 1]}
    \item *El sistema valido el dato ingresado en el campo de usuario. \textbf{[Trayectoria 2]}
    \item *El sistema envía un correo con un hipervínculo de restauración de contraseña a la dirección de correo al que está vinculada la cuenta del usuario.
    \item *El sistema muestra el mensaje de confirmación correspondiente.
    \item *El sistema redirige a la pantalla de inicio.
    \item Fin del caso de uso.
\end{enumerate}
\vspace*{1cm}
\Large{PROCEDIMIENTOS ALTERNATIVOS}\\
\large{Trayectoria 1}\\
\textbf{Condición}: El usuario envió el formulario con el campo de identificador vacío.
\begin{enumerate}
    \item *El sistema muestra el mensaje de error ME.1.
    \item Regresa al paso 4 de la trayectoria principal.
    \item Fin de la trayectoria.

\end{enumerate}
\large{Trayectoria 2}\\
\textbf{Condición}: El usuario envió el formulario con un usuario no válido.
\begin{enumerate}
    \item *El sistema muestra el mensaje de error ME.6.
    \item *El sistema limpia el campo de usuario.
    \item Regresa al paso 4 de la trayectoria principal.
    \item Fin de la trayectoria.
\end{enumerate}