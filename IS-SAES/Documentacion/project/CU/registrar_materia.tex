\newpage
\section{Caso de Uso Registrar Materia}
\begin{longtable}{ | p{6cm} | p{10cm} |}
\hline
\textbf{NOMBRE} & Registrar Materia\\
\hline
\textbf{TIPO} & Primario\\
\hline
\textbf{DESCRIPCIÓN} & El analista registra los datos de una nueva unidad de aprendizaje\\
\hline
\textbf{ENTRADAS} & Nombre de la Materia: Cadena de caracteres\\ & Créditos: Valor numérico con dos decimales\\ & Nivel: Selección de una lista numérica (entre 1 y 4)\\ & Área: Selección de área de una lista.\\ & Cupo: Selección de una lista numérica (entre 25 y 35)\\
\hline
\textbf{SALIDAS} & Se mostrará un mensaje de confirmación si los datos se registran de manera exitosa.\\ & Se mostrará un mensaje de error si se dejaron campos en blanco, o el formato de los datos no es correcto.\\
\hline
\textbf{PRECONDICIONES} & No hay\\
\hline
\textbf{POSTCONDICIONES} & Se registra una nueva materia dentro del mapa curricular.\\
\hline
\textbf{SITUACIONES DE ERROR} & \begin{itemize}
    \item Se dejaron campos obligatorios en blanco.
    \item El formato de los datos es inválido.
\end{itemize}\\
\hline
\textbf{ESTADO DEL SISTEMA EN CASO DE ERROR} & No se registra la materia.\\
\hline
\textbf{ACTORES} & Analista\\
\hline
\textbf{MENSAJES} & \textbf{Mensaje de Confirmación}: Se muestra cuando los datos han sido almacenados correctamente: MC.1\\ & \textbf{Mensajes de Error}: Se muestran en los siguientes casos: \begin{itemize}
    \item Se dejaron campos obligatorios en blanco: ME.1
    \item El formato de los datos es inválido: ME.2
\end{itemize}\\
\hline
\textbf{AUTOR} & Hernández Pineda Miguel Angel\\
\hline
\end{longtable}
\newpage
\noindent
\Large{PROCEDIMIENTO ESTÁNDAR}
\large{}
\begin{enumerate}
    \item El usuario ingresa al sistema como Analista.
    \item El usuario da click en la opción <<Registrar Materia>> de la Sección Registrar (Figura 3.15)
    \item El usuario ingresa el nombre de la materia.
    \item El usuario ingresa el número de créditos de la materia.
    \item El usuario selecciona un nivel.
    \item El usuario selecciona un área de conocimiento.
    \item El usuario da click en el botón enviar.
    \item *El sistema valida que no haya campos obligatorios en blanco. \textbf{[Trayectoria 1]}
    \item *El sistema valida los datos en el campo de nombre de la materia. \textbf{[Trayectoria 2]}
    \item *El sistema valida los datos en el campo de créditos según la regla de negocio RN.15. \textbf{[Trayectoria 3]}
    \item Se registra la materia de forma correcta en el mapa curricular.
    \item Se muestra el mensaje de confirmación MC.1.
    \item Fin del caso de uso.
\end{enumerate}
\vspace*{1cm}
\Large{PROCEDIMIENTOS ALTERNATIVOS}\\
\large{Trayectoria 1}\\
\textbf{Condición}: Se dejaron campos obligatorios en blanco.
\begin{enumerate}
    \item *El sistema muestra el mensaje de error ME.1.
    \item *El sistema marca de color rojo los campos vacíos.
    \item Regresa al punto 3 de la trayectoria principal.
    \item Fin de la trayectoria.
\end{enumerate}
\large{Trayectoria 2}\\
\textbf{Condición}: El formato de texto para el campo de nombre de la materia es inválido.
\begin{enumerate}
    \item *El sistema muestra el mensaje de error ME.2.
    \item *El sistema marca de color rojo el campo de nombre.
    \item Regresa al punto 3 de la trayectoria principal.
    \item Fin de la trayectoria.
\end{enumerate}
\large{Trayectoria 3}\\
\textbf{Condición}: El formato para el campo de créditos es inválido según la regla de negocio RN.15.
\begin{enumerate}
    \item *El sistema muestra el mensaje de error ME.2.
    \item *El sistema marca de color rojo el campo de créditos.
    \item Regresa al punto 4 de la trayectoria principal.
    \item Fin de la trayectoria.
\end{enumerate}