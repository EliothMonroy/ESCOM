\newpage
\section{Caso de Uso Cambiar Contraseña}
\begin{longtable}{ | p{6cm} | p{10cm} |}
\hline
\textbf{NOMBRE} & Cambiar Contraeña\\
\hline
\textbf{TIPO} & Primario\\
\hline
\textbf{DESCRIPCIÓN} & El usuario cambia su contraseña actual por una nueva.\\
\hline
\textbf{ENTRADAS} & Identificador del usuario: \begin{itemize}
    \item En caso de ser un alumno, su número de boleta: Entero positivo de 10 dígitos con el formato de número de boleta.
    \item En caso de ser un académico o analista de gestión escolar, su RFC: Cadena de 13 caracteres con el formato de RFC.
    \item Contraseña actual: cadena con una longitud de 6 o más caracteres, incluyendo una letra mayúscula, una letra minúscula, un número y un carácter especial.
    \item Nueva contraseña: cadena con una longitud de 6 o más caracteres, incluyendo una letra mayúscula, una letra minúscula, un número y un carácter especial.
    \item Confirmación de nueva contraseña: cadena con una longitud de 6 o más caracteres, incluyendo una letra mayúscula, una letra minúscula, un número y un carácter especial.
    \end{itemize}\\
\hline
\textbf{SALIDAS} & Un mensaje de confirmación cuando la nueva contraseña haya sido actualizada.\\ & Un mensaje de error si la nueva contraseña y su confirmación no coinciden, si la contraseña actual no coincide con la registrada en el sistema o si las contraseñas ingresadas no cumplen con la regla de negocio RN.2.\\
\hline
\end{longtable}
\newpage
\begin{longtable}{ | p{6cm} | p{10cm} |}
\hline
\textbf{PRECONDICIONES} & Estar registrado en el sistema con un número de boleta o RFC válido.\\ & Si se trata de un simple cambio de contraseña, haber iniciado sesión.\\ & Si se trata de un restablecimiento de contraseña, haber dado clic en el enlace de restablecimiento enviado a la dirección de correo asociada al usuario.
\\
\hline
\textbf{POSTCONDICIONES} & La contraseña se actualiza en el sistema para el usuario en cuestión.\\ & Si se trata de un simple cambio de contraseña, el sistema redirigirá a la página principal del sistema.\\ & Si se trata de un restablecimiento de contraseña, el sistema redirigirá al formulario de inicio de sesión.\\
\hline
\textbf{SITUACIONES DE ERROR} & \begin{itemize}
    \item La nueva contraseña y su confirmación no coinciden.
    \item Las contraseñas ingresadas no cumplen con el formato indicado en la regla de negocio RN.2.
    \item Se dejan campos en blanco al enviar el formulario.
    \item Si se trata de un simple cambio de contraseña, la contraseña actual ingresada no coincide con la registrada en el sistema.
\end{itemize}\\
\hline
\textbf{ESTADO DEL SISTEMA EN CASO DE ERROR} & No se actualizan los datos y se envía el mensaje de error correspondiente.\\
\hline
\textbf{ACTORES} & Alumno, Académica, Analista\\
\hline
\end{longtable}
\newpage
\begin{longtable}{ | p{6cm} | p{10cm} |}
\hline
\textbf{MENSAJES} & \textbf{Mensaje de Confirmación}: Se presenta cuando los campos del formulario de cambio de contraseña hayan sido llenados correctamente: MC.3 \\ & \textbf{Mensajes de Error}: Se presentan en los siguientes casos: \begin{itemize}
    \item Cuando se dejan campos en blanco en el formulario: ME.1
    \item Cuando la nueva contraseña y su confirmación no coinciden: ME.8
    \item Cuando la contraseña actual ingresada no coincide con la registrada en el sistema: ME.9
    \item Cuando las contraseñas no cumplen con el formato establecido en las reglas del negocio: ME.2
\end{itemize}\\
\hline
\textbf{AUTOR} & Medina Luqueño Ana Ximena\\
\hline
\end{longtable}
\vspace*{1cm}
\noindent
\Large{PROCEDIMIENTO ESTÁNDAR}
\large{}
\begin{enumerate}
    \item El usuario selecciona la opción Cambiar Contraseña desde el menú de usuario. 
    \item Se carga el formulario de cambio de contraseña.
    \item El usuario ingresa su contraseña actual, nueva contraseña y confirmación de esta.
    \item El usuario selecciona el botón de Reestablecer Contraseña.
    \item *El sistema valida que ninguno de los campos esté vacío. \textbf{[Trayectoria 1]}
    \item *El sistema valida que la contraseña actual coincida con la registrada en la base de datos del sistema. \textbf{[Trayectoria 2]}
    \item *El sistema valida que el dato en el campo nueva contraseña sea válido. \textbf{[Trayectoria 3]}
    \item *El sistema valida que el dato en el campo nueva contraseña y el dato en el campo confirmación de nueva contraseña sean iguales. \textbf{[Trayectoria 4]} 
    \item *El sistema muestra el mensaje de confirmación correspondiente.
    \item *El sistema redirige a la pantalla de inicio.
    \item Fin del caso de uso. 
\end{enumerate}
\vspace*{1cm}
\Large{PROCEDIMIENTOS ALTERNATIVOS}\\
\large{Trayectoria 1}\\
\textbf{Condición}: El usuario envió el formulario con algún campo vacío. (Cambio de contraseña)
\begin{enumerate}
    \item *El sistema muestra el mensaje de error ME.1.
    \item Regresa al paso 3 de la trayectoria principal.
    \item Fin de la trayectoria.
\end{enumerate}
\large{Trayectoria 2}\\
\textbf{Condición}: El usuario envió el formulario con una contraseña actual incorrecta. (Cambio de contraseña)
\begin{enumerate}
    \item *El sistema muestra el mensaje de error ME.8
    \item *El sistema limpia el campo de contraseña actual.
    \item Regresa al paso 3 de la trayectoria principal.
    \item Fin de la trayectoria.
\end{enumerate}
\large{Trayectoria 3}\\
\textbf{Condición}: El usuario envió el formulario con una nueva contraseña con un formato no válido. (Cambio de contraseña)
\begin{enumerate}
    \item *El sistema muestra el mensaje de error ME.2.
    \item *El sistema marca de color rojo y limpia los campos de nueva contraseña y confirmación de nueva contraseña.
    \item Regresa al paso 3 de la trayectoria principal.
    \item Fin de la trayectoria.
\end{enumerate}
\large{Trayectoria 4}\\
\textbf{Condición}: El usuario envió el formulario con una nueva contraseña y confirmación de nueva contraseña que no coinciden. (Cambio de contraseña)
\begin{enumerate}
    \item *El sistema muestra el mensaje de error ME.8.
    \item *El sistema marca de color rojo y limpia el campo de confirmación de nueva contraseña.
    \item Regresa al paso 3 de la trayectoria principal.
    \item Fin de la trayectoria.
\end{enumerate}