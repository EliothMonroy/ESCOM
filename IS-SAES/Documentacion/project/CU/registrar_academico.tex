\newpage
\section{Caso de Uso Registrar Académico}
\begin{longtable}{ | p{6cm} | p{10cm} |}
\hline
\textbf{NOMBRE} & Registrar Académico\\
\hline
\textbf{TIPO} & Primario\\
\hline
\textbf{DESCRIPCIÓN} & El usuario registra los datos del Académico en un formulario otorgandole un perfil en el sistema.\\
\hline
\textbf{ENTRADAS} & Nombre del Académico: Caracteres (solo letras)\\& Apellido Paterno: Caracteres (solo letras)\\& Apellido Materno: Caracteres (solo letras)\\&Número de empleado: Números con longitud de 10 dígitos\\&RFC: Números y letras con el formato de RFC con longitud de 10 dígitos\\& Correo Electrónico: Caracteres con el formato de correo electrónico\\& Contraseña: longitud de 6 a 20 caracteres.\\
\hline
\textbf{SALIDAS} & Se mostrará un mensaje de confirmación si el registro fue exitoso.\\& Se mostrará un mensaje de error cuando falten campos por llenar o el formato no sea correcto.\\
\hline
\textbf{PRECONDICIONES} & No hay\\
\hline
\textbf{POSTCONDICIONES} & Se registran los datos del Académico en la base de datos y se le otorga una cuenta\\
\hline
\textbf{SITUACIONES DE ERROR} & \begin{itemize}
    \item Se dejaron campos obligatorios en blanco.
    \item El correo ya ha sido registrado en otra cuenta.
    \item El formato de los datos es inválido.
    \item El número de empleado ya fue asignado a otro Académico.
\end{itemize}\\
\hline
\textbf{ESTADO DEL SISTEMA EN CASO DE ERROR} & No se registran los datos del Académico\\
\hline
\textbf{ACTORES} & Analista, Jefe de Gestión Escolar\\
\hline
\end{longtable}
\newpage
\begin{longtable}{ | p{6cm} | p{10cm} |}
\hline
\textbf{MENSAJES} & \textbf{Mensaje de Confirmación}: Se muestra cuando los datos han sido almacenados correctamente: MC.1\\&\textbf{Mensajes de Error}: Se muestran en los siguientes casos:\begin{itemize}
    \item Se dejaron campos obligatorios en blanco: ME.1
    \item El formato de los datos es inválido: ME.2
    \item El correo está registrado en otra cuenta: ME.3
    \item El número de empleado ya fue asignado a otro Académico: ME.5
\end{itemize}\\
\hline
\textbf{AUTOR} & Matus López Carlos Eduardo\\
\hline
\end{longtable}
\vspace*{1cm}
\noindent
\Large{PROCEDIMIENTO ESTÁNDAR}
\large{}
\begin{enumerate}
    \item El usuario ingresa al sistema como Analista o Jefe de Gestión Escolar.
    \item El usuario da click en la opción de <<Registrar Académico>> de la sección Registrar (Figura 3.14)
    \item El usuario ingresa el nombre del profesor.
    \item El usuario ingresa el apellido paterno.
    \item El usuario ingresa el apellido materno.
    \item El usuario ingresa el número de empleado.
    \item El usuario ingresa el correo electrónico.
    \item El usuario ingresa el RFC del profesor.
    \item El usuario ingresa la contraseña del profesor.
    \item El usuario da click en el botón enviar.
    \item El sistema valida que no haya campos obligatorios en blanco. \textbf{[Trayectoria 1]}
    \item *El sistema valida los datos en el campo de nombre. \textbf{[Trayectoria 2]}
    \item *El sistema valida los datos en el campo de apellido paterno. \textbf{[Trayectoria 3]}
    \item *El sistema valida los datos en el campo de apellido materno. \textbf{[Trayectoria 4]}
    \item *El sistema valida los datos en el campo de número de empleado. \textbf{[Trayectoria 5]} \textbf{[Trayectoria 9]}
    \item *El sistema valida los datos en el campo de correo electrónico. \textbf{[Trayectoria 6]}
    \item *El sistema valida los datos en el campo de RFC. \textbf{[Trayectoria 7]}
    \item *El sistema valida los datos en el campo de contraseña según la regla de negocio RN.2. \textbf{[Trayectoria 8]}
    \item Se registran los datos del profesor y se crea un perfil en el sistema.
    \item Se muestra un mensaje de confirmación MC.1.
    \item Fin del caso de uso
\end{enumerate}
\vspace*{1cm}
\Large{PROCEDIMIENTOS ALTERNATIVOS}\\
\large{Trayectoria 1}\\
\textbf{Condición}: Quedaron campos obligatorios en blanco.
\begin{enumerate}
    \item *El sistema muestra el mensaje de error ME.1.
    \item *El sistema marca de color rojo los campos vacios.
    \item Regresa al punto 3 de la trayectoria principal.
    \item Fin de la trayectoria.
\end{enumerate}
\large{Trayectoria 2}\\
\textbf{Condición}: El formato de texto para el campo nombre del alumno es inválido.
\begin{enumerate}
    \item *El sistema muestra el mensaje de error ME.2.
    \item *El sistema marca de color rojo el campo de nombre.
    \item Regresa al punto 3 de la trayectoria principal.
    \item Fin de la trayectoria.
\end{enumerate}
\large{Trayectoria 3}\\
\textbf{Condición}: El formato de texto para el campo apellido paterno es inválido.
\begin{enumerate}
    \item *El sistema muestra el mensaje de error ME.2.
    \item *El sistema marca de color rojo el campo de apellido paterno.
    \item Regresa al punto 4 de la trayectoria principal.
    \item Fin de la trayectoria.
\end{enumerate}
\large{Trayectoria 4}\\
\textbf{Condición}: El formato de texto para el campo apellido materno es inválido.
\begin{enumerate}
    \item *El sistema muestra el mensaje de error ME.2.
    \item *El sistema marca de color rojo el campo de apellido materno.
    \item Regresa al punto 5 de la trayectoria principal.
    \item Fin de la trayectoria.
\end{enumerate}
\large{Trayectoria 5}\\
\textbf{Condición}: El formato de los datos en el campo número de empleado es inválido.
\begin{enumerate}
    \item *El sistema muestra el mensaje de error ME.2.
    \item *El sistema marca de color rojo el campo número de empleado.
    \item Regresa al punto 6 de la trayectoria principal.
    \item Fin de la trayectoria.
\end{enumerate}
\large{Trayectoria 6}\\
\textbf{Condición}: El formato del correo electrónico es inválido.
\begin{enumerate}
    \item *El sistema muestra el mensaje de error ME.2.
    \item *El sistema marca de color rojo el campo de correo electrónico.
    \item Regresa al punto 7 de la trayectoria principal.
    \item Fin de la trayectoria.
\end{enumerate}
\large{Trayectoria 7}\\
\textbf{Condición}: El formato del RFC es inválido.
\begin{enumerate}
    \item *El sistema muestra el mensaje de error ME.2.
    \item *El sistema marca de color rojo el campo de RFC.
    \item Regresa al punto 8 de la trayectoria principal.
    \item Fin de la trayectoria.
\end{enumerate}
\large{Trayectoria 8}\\
\textbf{Condición}: El formato de la contraseña es inválido según la regla de negocio RN.2.
\begin{enumerate}
    \item *El sistema muestra el mensaje de error ME.2.
    \item *El sistema marca de color rojo el campo de contraseña.
    \item Regresa al punto 9 de la trayectoria principal.
    \item Fin de la trayectoria.
\end{enumerate}
\large{Trayectoria 9}\\
\textbf{Condición}: El número de empleado asignado ya ha sido registrado.
\begin{enumerate}
    \item *El sistema muestra el mensaje de error ME.5.
    \item *El sistema marca de color rojo el campo número de empleado.
    \item Regresa al punto 6 de la trayectoria principal.
    \item Fin de la trayectoria.
\end{enumerate}