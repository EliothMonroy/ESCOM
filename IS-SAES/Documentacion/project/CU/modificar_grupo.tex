\newpage
\section{Caso de Uso Modificar Grupo}
\begin{longtable}{ | p{6cm} | p{10cm} |}
\hline
\textbf{NOMBRE} & Modificar Grupo\\
\hline
\textbf{TIPO} & Secundario\\
\hline
\textbf{DESCRIPCIÓN} & El usuario modifica un horario para una materia determinada para el siguiente periodo de reinscripciones\\
\hline
\textbf{ENTRADAS} & Materia: Selección de una materia de una lista.\\ & Profesor: Selección del profesor de una lista.\\ & Grupo: Selección de un grupo de una lista.\\ & Horario: Selección de un horario de una lista.\\
\hline
\textbf{SALIDAS} & Se mostrará un mensaje de confirmación si las asignaciones fueron exitosas.\\ & Se mostrará un mensaje de error si existe la misma materia 2 o mas veces en el mismo grupo, los horarios asignados ya están siendo ocupados por otra materia en el mismo grupo.\\
\hline
\textbf{PRECONDICIONES} & Registro de profesores, grupos, horarios y materias con anterioridad.\\
\hline
\textbf{POSTCONDICIONES} & Se modificara un o varios aspectos para el periodo siguiente al grupo determinado\\
\hline
\textbf{SITUACIONES DE ERROR} & \begin{itemize}
    \item No hay materias registradas.
    \item No hay profesores del área registrados.
    \item No hay grupos registrados.
    \item No hay horarios registrados.
    \item Ya está la materia registrada en el grupo asignado.
    \item El horario asignado ya esta siendo ocupado por otra materia en el mismo grupo.
\end{itemize}\\
\hline
\textbf{ESTADO DEL SISTEMA EN CASO DE ERROR} & No se modifica el grupo seleccionado.\\
\hline
\textbf{ACTORES} & Analista, Jefe de Gestión Escolar\\
\hline
\end{longtable}
\newpage
\begin{longtable}{ | p{6cm} | p{10cm} |}
\hline
\textbf{MENSAJES} & \textbf{Mensaje de Confirmación}: Se muestra cuando los datos han sido modificados correctamente: MC.2\\ & \textbf{Mensajes de Error}: Se muestran en los siguientes casos: \begin{itemize}
    \item Se dejaron campos en blanco: ME.1
    \item No hay materias registradas: ME.10
    \item No hay profesores del área registrados: ME.11
    \item No hay grupos registrados: ME.12
    \item No hay horarios registrados: ME.13
    \item Ya está la materia registrada en el grupo asignado: ME.14
    \item El profesor ya tiene una materia asignada en el mismo horario: ME.15
    \item El horario asignado ya esta siendo ocupado por otra materia en el mismo grupo: ME.16
\end{itemize}\\
\hline
\textbf{AUTOR} & Matus López Carlos Eduardo\\
\hline
\end{longtable}
\vspace*{1cm}
\noindent
\Large{PROCEDIMIENTO ESTÁNDAR}
\large{}
\begin{enumerate}
    \item El usuario ingresa al sistema como Analista o Jefe de Gestión Escolar.
    \item El usuario da click en la opción <<Modificar Horario>> de la sección Modificar (Figura 3.) . \textbf{[Trayectoria 1]} . \textbf{[Trayectoria 2]} . \textbf{[Trayectoria 3]} . \textbf{[Trayectoria 4]}
    \item El usuario selecciona un grupo.  
  \item El usuario selecciona una materia.
    \item El usuario selecciona un profesor.
    \item El usuario selecciona un horario.
    \item El usuario da click en el botón enviar.
    \item *El sistema valida que no se dejen campos obligatorios en blanco. \textbf{[Trayectoria 5]}
    \item *El sistema valida los datos de profesor y horario según la regla de negocio RN.11. \textbf{[Trayectoria 6]}
    \item *El sistema valida los datos de materia y grupo según la regla de negocio RN.13. \textbf{[Trayectoria 7]}
    \item *El sistema valida los datos de horario y grupo según la regla de negocio RN.14. \textbf{[Trayectoria 8]}
    \item Se modifica el grupo de manera correcta para el siguiente periodo.
    \item Se muestra el mensaje de confirmación MC.1.
    \item Fin del caso de uso.
\end{enumerate}
\vspace*{1cm}
\Large{PROCEDIMIENTOS ALTERNATIVOS}\\
\large{Trayectoria 1}\\
\textbf{Condición}: No hay materias registradas.
\begin{enumerate}
\item *El sistema muestra el mensaje de error ME.10.
\item Regresa al punto 2 de la trayectoria principal.
\item Fin de la trayectoria.
\end{enumerate}
\large{Trayectoria 2}\\
\textbf{Condición}: No hay profesores del área registrados.
\begin{enumerate}
\item *El sistema muestra el mensaje de error ME.11.
\item Regresa al punto 2 de la trayectoria principal.
\item Fin de la trayectoria.
\end{enumerate}
\large{Trayectoria 3}\\
\textbf{Condición}: No hay grupos registrados.
\begin{enumerate}
\item *El sistema muestra el mensaje de error ME.12.
\item Regresa al punto 2 de la trayectoria principal.
\item Fin de la trayectoria.
\end{enumerate}
\newpage
\noindent
\large{Trayectoria 4}\\
\textbf{Condición}: No hay horarios registrados.
\begin{enumerate}
\item *El sistema muestra el mensaje de error ME.13.
\item Regresa al punto 2 de la trayectoria principal.
\item Fin de la trayectoria.
\end{enumerate}
\large{Trayectoria 5}\\
\textbf{Condición}: Quedaron campos obligatorios en blanco.
\begin{enumerate}
\item *El sistema muestra el mensaje de error ME.1.
\item *El sistema marca de color rojo los campos vacíos.
\item Regresa al punto 3 de la trayectoria principal.
\item Fin de la trayectoria.
\end{enumerate}
\large{Trayectoria 6}\\
\textbf{Condición}: Los datos no son congruentes según la regla de negocio RN.11.
\begin{enumerate}
    \item *El sistema muestra el mensaje de error ME.15.
    \item *El sistema limpia el campo de profesor.
    \item Regresa al punto 4 de la trayectoria principal.
    \item Fin de la trayectoria.
\end{enumerate}
\large{Trayectoria 7}\\
\textbf{Condición}: Los datos no son congruentes según la regla de negocio RN.13.
\begin{enumerate}
    \item *El sistema muestra el mensaje de error ME.14.
    \item *El sistema limpia el campo de grupo.
    \item Regresa al punto 5 de la trayectoria principal.
    \item Fin de la trayectoria.
\end{enumerate}
\large{Trayectoria 8}\\
\textbf{Condición}: Los datos no son congruentes según la regla de negocio RN.14.
\begin{enumerate}
    \item *El sistema muestra el mensaje de error ME.16.
    \item *El sistema limpia el campo de horario.
    \item Regresa al punto 6 de la trayectoria principal.
    \item Fin de la trayectoria.
\end{enumerate}