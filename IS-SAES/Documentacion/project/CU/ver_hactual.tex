\newpage
\section{Caso de Uso Ver Horario Actual}
\begin{longtable}{ | p{6cm} | p{10cm} |}
\hline
\textbf{NOMBRE} & Ver Horario Actual\\
\hline
\textbf{TIPO} & Secundario\\
\hline
\textbf{DESCRIPCIÓN} & El alumno consulta el horario con el que está actualmente inscrito.\\
\hline
\textbf{ENTRADAS} & Ninguno\\
\hline
\textbf{SALIDAS} & El sistema mostrará el horario actual del alumno en cuestión en pantalla, junto con algunos de sus datos generales (número de boleta y nombre completo).\\
\hline
\textbf{PRECONDICIONES} & El alumno debe haber iniciado sesión.\\ & El alumno debe estar inscrito en el semestre en cuestión.\\
\hline
\textbf{POSTCONDICIONES} & El sistema muestra el horario actual del alumno en cuestión junto con algunos de sus datos generales (número de boleta y nombre completo).\\
\hline
\textbf{SITUACIONES DE ERROR} & Ninguno\\
\hline
\textbf{ESTADO DEL SISTEMA EN CASO DE ERROR} & No aplica.\\
\hline
\textbf{ACTORES} & Alumno\\
\hline
\textbf{MENSAJES} & \textbf{Mensaje de Alerta}: Se presenta en caso de que el alumno no esté inscrito en el semestre actual: MA.3\\
\hline
\textbf{AUTOR} & Medina Luqueño Ana Ximena\\
\hline
\end{longtable}
\vspace*{1cm}
\noindent
\Large{PROCEDIMIENTO ESTÁNDAR}
\large{}
\begin{enumerate}
    \item El usuario ingresa al sistema como alumno.
    \item El usuario selecciona la opción “Ver horario actual” en el menú de opciones.
    \item *El sistema valida que el número de materias inscritas por el alumno sea mayor a 0. \textbf{[Trayectoria 1]}
    \item *El sistema despliega el horario del alumno.
    \item Fin del caso de uso.
\end{enumerate}
\newpage
\noindent
\Large{PROCEDIMIENTOS ALTERNATIVOS}\\
\large{Trayectoria 1}\\
\textbf{Condición}: El alumno no tiene materias inscritas.
\begin{enumerate}
    \item *El sistema muestra el mensaje de alerta MA.3.
    \item Regresa al paso 5 de la trayectoria principal.
    \item Fin de la trayectoria. 
\end{enumerate}