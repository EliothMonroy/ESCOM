\newpage
\section{Caso de Uso Editar Académico}
\begin{longtable}{ | p{6cm} | p{10cm} |}
    \hline
    \textbf{NOMBRE} & Editar Académico\\
    \hline
    \textbf{TIPO} & Primario\\
    \hline
    \textbf{DESCRIPCIÓN} & El jefe de gestión escolar o un analista modifica los datos de un académico registrado en el sistema.\\
    \hline
    \textbf{ENTRADAS} &
    \begin{itemize}
    	\item Nombre del Académico: Caracteres (solo letras).
    	\item Apellido Paterno: Caracteres (solo letras).
    	\item Apellido Materno: Caracteres (solo letras).
    	\item Número de empleado: Número con longitud de 10 dígitos.
    	\item Correo Electrónico: Caracteres con el formato de correo electrónico.
    	\item Contraseña: Cadena con una longitud de 6 a 20 caracteres.
    \end{itemize}\\  
    \hline
    \textbf{SALIDAS} & Mensaje de éxito o mensaje de error según sea el caso.\\
    \hline
    \textbf{PRECONDICIONES} & El académico a editar debe estar previamente registrado.\\
    \hline
    \textbf{POSTCONDICIONES} & Se registran los nuevos datos del Académico en la base de datos.\\
    \hline
    \textbf{SITUACIONES DE ERROR} &Se presentan en los siguientes casos:
    \begin{itemize}
    	\item Se dejaron campos obligatorios en blanco.
    	\item El correo ya ha sido registrado en otra cuenta.
    	\item El formato de los datos es inválido.
    	\item El número de empleado ya fue asignado a otro Académico.
    \end{itemize}\\
    \hline
    \textbf{ESTADO DEL SISTEMA EN CASO DE ERROR} &  En espera de nuevas entradas.\\
    \hline
    \textbf{ACTORES} & Jefe de Gestión Escolar, Analista\\
    \hline
    \textbf{MENSAJES}  & \textbf{Mensaje de Confirmación}: Se muestra cuando los datos han sido almacenados correctamente: MC.3\\&\textbf{Mensajes de Error}: Se muestran en los siguientes casos:\begin{itemize}
    	\item Se dejaron campos obligatorios en blanco: ME.37
    	\item El formato de los datos es inválido: ME.31, ME.38, ME.39, ME.40
    	\item El correo está registrado en otra cuenta: ME.2
    	\item El número de empleado ya fue asignado a otro Académico: ME.4
    \end{itemize}\\
    \hline
    \textbf{AUTOR} & Monroy Martos Elioth\\
    \hline
\end{longtable}
\vspace*{1cm}
\noindent
\Large{PROCEDIMIENTO ESTÁNDAR}
\large{}
\begin{enumerate}
	\item El usuario ingresa al sistema como Jefe de Gestión Escolar o Analista.
	\item El usuario da click en la opción de <<Administrar Académico>> de la sección <<Administrar>>.
	\item El usuario ingresa el RFC del académico a modificar y presiona el botón <<Buscar>> (Figura 3.13).
	\textbf{[Trayectoria 1]}
	\item *El sistema le muestra al usuario los resultados obtenidos de su búsqueda en forma de lista.
	\item El usuario presiona el botón <<Editar>> del académico que desea editar (Figura 3.14).
	\item *El sistema le muestra el formulario de edición de académico al usuario.
	\item El usuario ingresa la nueva información del académico.
	\item El usuario da click en el botón <<Guardar>>.
	\item El sistema valida que no haya campos obligatorios en blanco. \textbf{[Trayectoria 2]}
	\item *El sistema valida los datos en el campo de nombre. \textbf{[Trayectoria 3]}
	\item *El sistema valida los datos en el campo de apellido paterno. \textbf{[Trayectoria 4]}
	\item *El sistema valida los datos en el campo de apellido materno. \textbf{[Trayectoria 5]}
	\item *El sistema valida los datos en el campo de número de empleado. \textbf{[Trayectoria 6]} \textbf{[Trayectoria 9]}
	\item *El sistema valida los datos en el campo de correo electrónico. \textbf{[Trayectoria 7]}
	\item *El sistema valida los datos en el campo de contraseña según la regla de negocio RN.2. \textbf{[Trayectoria 8]}
	\item *El sistema almacena los nuevos datos del académico y reemplaza los anteriores.
	\item *El sistema muestra el mensaje de confirmación MC.3.
	\item Fin del caso de uso.
\end{enumerate}
\vspace*{1cm}
\Large{PROCEDIMIENTOS ALTERNATIVOS}\\
\large{Trayectoria 1}\\
\textbf{Condición}: La búsqueda de académicos no obtuvo resultados.
\begin{enumerate}
	\item *El sistema muestra el mensaje de error ME.23.
	\item Regresa al punto 3 de la trayectoria principal.
	\item Fin de la trayectoria.
\end{enumerate}
\large{Trayectoria 2}\\
\textbf{Condición}: Quedaron campos obligatorios en blanco.
\begin{enumerate}
	\item *El sistema muestra el mensaje de error ME.37.
	\item *El sistema marca el campo vacío.
	\item Regresa al punto 3 de la trayectoria principal.
	\item Fin de la trayectoria.
\end{enumerate}
\large{Trayectoria 3}\\
\textbf{Condición}: El formato de texto para el campo <<Nombre>> es inválido.
\begin{enumerate}
	\item *El sistema muestra el mensaje de error ME.38.
	\item *El sistema marca el campo <<Nombre>>.
	\item Regresa al punto 7 de la trayectoria principal.
	\item Fin de la trayectoria.
\end{enumerate}
\large{Trayectoria 4}\\
\textbf{Condición}: El formato de texto para el campo <<Apellido Paterno>> es inválido.
\begin{enumerate}
	\item *El sistema muestra el mensaje de error ME.38.
	\item *El sistema marca el campo <<Apellido Paterno>>.
	\item Regresa al punto 7 de la trayectoria principal.
	\item Fin de la trayectoria.
\end{enumerate}
\large{Trayectoria 5}\\
\textbf{Condición}: El formato de texto para el campo <<Apellido Materno>> es inválido.
\begin{enumerate}
	\item *El sistema muestra el mensaje de error ME.38.
	\item *El sistema marca el campo <<Apellido Materno>>.
	\item Regresa al punto 7 de la trayectoria principal.
	\item Fin de la trayectoria.
\end{enumerate}
\large{Trayectoria 6}\\
\textbf{Condición}: El formato de los datos en el campo <<Número de Empleado>> es inválido.
\begin{enumerate}
	\item *El sistema muestra el mensaje de error ME.38.
	\item *El sistema marca el campo <<Número de Empleado>>.
	\item Regresa al punto 7 de la trayectoria principal.
	\item Fin de la trayectoria.
\end{enumerate}
\large{Trayectoria 7}\\
\textbf{Condición}: El formato del correo electrónico es inválido.
\begin{enumerate}
	\item *El sistema muestra alguno de los mensajes de error: ME.39, ME.40.
	\item *El sistema marca el campo <<Correo Electrónico>>.
	\item Regresa al punto 7 de la trayectoria principal.
	\item Fin de la trayectoria.
\end{enumerate}
\large{Trayectoria 8}\\
\textbf{Condición}: El formato de la contraseña es inválido según la regla de negocio RN.2.
\begin{enumerate}
	\item *El sistema muestra alguno de los mensajes de error: ME.31, ME.38.
	\item *El sistema marca el campo <<Contraseña>>.
	\item Regresa al punto 7 de la trayectoria principal.
	\item Fin de la trayectoria.
\end{enumerate}
\large{Trayectoria 9}\\
\textbf{Condición}: El número de empleado asignado ya ha sido registrado.
\begin{enumerate}
	\item *El sistema muestra el mensaje de error ME.4.
	\item *El sistema marca el campo <<Número de Empleado>>.
	\item Regresa al punto 7 de la trayectoria principal.
	\item Fin de la trayectoria.
\end{enumerate}