\newpage
\section{Caso de Uso: Buscar Materia}
\begin{longtable}{ | p{6cm} | p{10cm} |}
\hline
\textbf{NOMBRE} & Buscar Materia\\
\hline
\textbf{TIPO} & Secundario\\
\hline
\textbf{DESCRIPCIÓN} & Se despliega las materias disponibles en la base de datos, en base al grupo seleccionado\\
\hline
\textbf{ENTRADAS} & Grupo: Cadena de caracteres de tamaño 4 o 5 \\
\hline
\textbf{SALIDAS} & Listado de materias del grupo seleccionado\\
\hline
\textbf{PRECONDICIONES} & Estar registrado en el sistema
Registro de materias y grupos con anterioridad\\
\hline
\textbf{POSTCONDICIONES} & No aplica\\
\hline
\textbf{SITUACIONES DE ERROR} & \begin{itemize}
    \item Se dejaron campos obligatorios en blanco.
    \item No hay grupos registrados con anterioridad.
    \item No hay materias registradas con anterioridad.
\end{itemize}
\\
\hline
\textbf{ESTADO DEL SISTEMA EN CASO DE ERROR} & No se despliega una tabla con las materias\\
\hline
\textbf{ACTORES} & Alumno\\
\hline
\textbf{MENSAJES} & \textbf{Mensasje de Error:}Se muestra en lso siguientes casos: \begin{itemize}
    \item Se dejaron campos en blanco : ME1
    \item No hay grupos registrados: ME12
    \item No hay materias registradas: ME10
\end{itemize} 
\\
\hline
\textbf{AUTOR} & Matus López Carlos Eduardo\\
\hline
\end{longtable}
\vspace*{1cm}
\noindent
\Large{PROCEDIMIENTO ESTÁNDAR}
\large{}
\begin{enumerate}
    \item El alumno ingresa al sistema e inicia sesión.
    \item El alumno da click en la opción <<Buscar Materia>> de la sección Reincripcion (Figura 3.7)
    \item El sistema despliega el formulario para la busqueda de materias.
    \item El alumno selecciona el grupo de cual desea visualizar las materias \textbf{[trayectoria 1]} \textbf{[trayectoria 2]}
    \item Se despliega las materias del grupo que eligio el alumno \textbf{[trayectoria 3]} 
    \item Fin del caso de uso.
\end{enumerate}
\Large{PROCEDIMIENTO ALTERNATIVOS}\\
\large{}
Trayectoria 1\\
\textbf{Condición:}Se dejaron campos obligatorios en blanco
\large{}
\begin{enumerate}
    \item *El sistema muestra el mensaje de error ME1
    \item *El sistema marca de color rojo los campos vacíos.
    \item Regresa al punto 3 de la trayectoria principal
    \item Fin de la trayectoria.
\end{enumerate}

Trayectoria 2\\
\textbf{Condición:}No hay grupos registrados previamente
\large{}
\begin{enumerate}
    \item *El sistema muestra el mensaje de error ME12
    \item Regresa al punto 5 de la trayectoria principal
    \item Fin de la trayectoria.
\end{enumerate}

Trayectoria 3\\
\textbf{Condición:}No hay materias registradas previamente
\large{}
\begin{enumerate}
    \item *El sistema muestra el mensaje de error ME10
    \item *El sistema muestra una pantalla con una tabla vacia
    \item Regresa al punto 5 de la trayectoria principal
    \item Fin de la trayectoria.
\end{enumerate}