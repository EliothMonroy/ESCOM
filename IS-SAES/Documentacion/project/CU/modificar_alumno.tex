\newpage
\section{Caso de Uso Modificar Alumno}
\begin{longtable}{ | p{6cm} | p{10cm} |}
\hline
\textbf{NOMBRE} & Modificar Alumno\\
\hline
\textbf{TIPO} & Secundario\\
\hline
\textbf{DESCRIPCIÓN} & El usuario modifica los datos del alumno en un formulario.\\
\hline
\textbf{ENTRADAS} & Nombre del alumno: Caracteres (solo letras)\\& Apellido Paterno: Caracteres (solo letras)\\& Apellido Materno: Caracteres (solo letras)\\&Número de Boleta: Números con longitud de 10 dígitos\\& Correo Electrónico: Caracteres con el formato de correo electrónico\\& Contraseña: longitud de 6 a 20 caracteres.\\
\hline
\textbf{SALIDAS} & Se mostrará un mensaje de confirmación si la modificacion del registro fue exitosa.\\& Se mostrará un mensaje de error cuando falten campos por llenar o el formato no sea correcto.\\
\hline
\textbf{PRECONDICIONES} & Registro de alumnos previamente \\
\hline
\textbf{POSTCONDICIONES} & Se modifican los datos del alumno en la base de datos  \\
\hline
\textbf{SITUACIONES DE ERROR} & \begin{itemize}
   \item  No hay alumnos registrados
    \item Se dejaron campos obligatorios en blanco.
    \item El correo ya ha sido registrado en otra cuenta.
    \item El formato de los datos es inválido.
    \item El número de boleta ya fue asignado a otro alumno.
\end{itemize}\\
\hline
\textbf{ESTADO DEL SISTEMA EN CASO DE ERROR} & No se modifican los datos del alumno\\
\hline
\textbf{ACTORES} & Analista, Jefe de Gestión Escolar\\
\hline
\textbf{MENSAJES} & \textbf{Mensaje de Confirmación}: Se muestra cuando los datos han sido modificados correctamente: MC.2\\&\textbf{Mensajes de Error}: Se muestran en los siguientes casos:\begin{itemize}
\item No hay alumnos registrados ME.17
    \item Se dejaron campos obligatorios en blanco: ME.1
    \item El formato de los datos es inválido: ME.2
    \item El correo está registrado en otra cuenta: ME.3
    \item El número de boleta ya fue asignado a otro alumno: ME.4
\end{itemize}\\
\hline
\textbf{AUTOR} & Matus López Carlos Eduardo\\
\hline
\end{longtable}
\vspace*{1cm}
\noindent
\Large{PROCEDIMIENTO ESTÁNDAR}
\large{}
\begin{enumerate}
    \item El usuario ingresa al sistema como Analista o Jefe de Gestión Escolar.
    \item El usuario da click en la opción de <<Modificar Alumno>> de la sección Modificar (Figura 3.)
    \item El usuario ingresa el numero de boleta del alumno a realizar modificacion\textbf{[Trayectoria 1]}
    \item El usuario ingresa el nombre del alumno.
    \item El usuario ingresa el apellido paterno.
    \item El usuario ingresa el apellido materno.
    \item El usuario ingresa el correo electrónico.
    \item El usuario ingresa la contraseña del alumno.
    \item El usuario da click en el botón enviar.
    \item El sistema valida que no haya campos obligatorios en blanco. \textbf{[Trayectoria 2]}
    \item *El sistema valida los datos en el campo de nombre. \textbf{[Trayectoria 3]}
    \item *El sistema valida los datos en el campo de apellido paterno. \textbf{[Trayectoria 4]}
    \item *El sistema valida los datos en el campo de apellido materno. \textbf{[Trayectoria 5]}
    \item *El sistema valida los datos en el campo de correo electrónico. \textbf{[Trayectoria 6]} \textbf{[Trayectoria 7]}
    \item *El sistema valida los datos en el campo de contraseña según la regla de negocio RN.2. \textbf{[Trayectoria 8]}
    \item Se modifican los datos del alumno.
    \item Se muestra un mensaje de confirmación MC.2.
    \item Fin del caso de uso
\end{enumerate}
\vspace*{1cm}
\Large{PROCEDIMIENTOS ALTERNATIVOS}\\
\large{Trayectoria 1}\\
\textbf{Condición}: La boleta ingresada no se encuentra registrada.
\begin{enumerate}
    \item *El sistema muestra el mensaje de error ME.17.
    \item Regresa al punto 2 de la trayectoria principal.
    \item Fin de la trayectoria.
\end{enumerate}
\large{Trayectoria 2}\\
\textbf{Condición}: Quedaron campos obligatorios en blanco.
\begin{enumerate}
    \item *El sistema muestra el mensaje de error ME.1.
    \item *El sistema marca de color rojo los campos vacios.
\item Regresa al punto 3 de la trayectoria principal.
    \item Fin de la trayectoria.
\end{enumerate}
\large{Trayectoria 3}\\
\textbf{Condición}: El formato de texto para el campo nombre del alumno es inválido.
\begin{enumerate}
    \item *El sistema muestra el mensaje de error ME.2.
    \item *El sistema marca de color rojo el campo de nombre.
    \item Regresa al punto 3 de la trayectoria principal.
    \item Fin de la trayectoria.
\end{enumerate}
\large{Trayectoria 4}\\
\textbf{Condición}: El formato de texto para el campo apellido paterno es inválido.
\begin{enumerate}
    \item *El sistema muestra el mensaje de error ME.2.
    \item *El sistema marca de color rojo el campo de apellido paterno.
    \item Regresa al punto 4 de la trayectoria principal.
    \item Fin de la trayectoria.
\end{enumerate}
\large{Trayectoria 5}\\
\textbf{Condición}: El formato de texto para el campo apellido materno es inválido.
\begin{enumerate}
    \item *El sistema muestra el mensaje de error ME.2.
    \item *El sistema marca de color rojo el campo de apellido materno.
    \item Regresa al punto 5 de la trayectoria principal.
    \item Fin de la trayectoria.
\end{enumerate}

\large{Trayectoria 6}\\
\textbf{Condición}: El formato del correo electrónico es inválido.
\begin{enumerate}
    \item *El sistema muestra el mensaje de error ME.2.
    \item *El sistema marca de color rojo el campo de correo electrónico.
    \item Regresa al punto 6 de la trayectoria principal.
    \item Fin de la trayectoria.
\end{enumerate}
\large{Trayectoria 7}\\
\textbf{Condición}: El correo ingresado ya esta registrado en el sistema.
\begin{enumerate}
    \item *El sistema muestra el mensaje de error ME.3.
    \item *El sistema marca de color rojo el campo de correo electrónico.
    \item Regresa al punto 6 de la trayectoria principal.
    \item Fin de la trayectoria.
\end{enumerate}
\large{Trayectoria 8}\\
\textbf{Condición}: El formato de la contraseña es inválido según la regla de negocio RN.2.
\begin{enumerate}
    \item *El sistema muestra el mensaje de error ME.2.
    \item *El sistema marca de color rojo el campo de contraseña.
    \item Regresa al punto 7 de la trayectoria principal.
    \item Fin de la trayectoria.
\end{enumerate}