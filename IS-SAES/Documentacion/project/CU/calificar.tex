\newpage
\section{Caso de Uso Calificar}
\begin{longtable}{ | p{6cm} | p{10cm} |}
\hline
\textbf{NOMBRE} & Calificar\\
\hline
\textbf{TIPO} & Secundario\\
\hline
\textbf{DESCRIPCIÓN} & El profesor le otorga una calificación al alumno al final del semestre.\\
\hline
\textbf{ENTRADAS} & Calificación: Selección de una lista numérica (0 a 10)\\ & Forma de Evaluación: Selección de una lista (Ordinario/Extraordinario)\\
\hline
\textbf{SALIDAS} & Se muestra un mensaje de confirmación en caso de que los datos se hayan guardado correctamente.\\ & Se muestra un mensaje de error correspondiente dependiendo de la situación que se presente.\\
\hline
\textbf{PRECONDICIONES} & El registro de evaluaciones esté activado, tener grupos, tener alumnos\\
\hline
\textbf{POSTCONDICIONES} & Se guardan las calificaciones establecidas para cada uno de los alumnos.\\
\hline
\textbf{SITUACIONES DE ERROR} & Se registro el tipo de evaluación pero no la calificación\\
\hline
\textbf{ESTADO DEL SISTEMA EN CASO DE ERROR} & No se registran los datos.\\
\hline
\textbf{ACTORES} & Profesor\\
\hline
\textbf{MENSAJES} & \textbf{Mensaje de Alerta}: Se presenta cuando no a todos los alumnos se les otorgó una calificación: MA.5\\ & \textbf{Mensaje de Confirmación}: Se presenta cuando todos los datos se almacenaron correctamente: MC.1\\ & \textbf{Mensajes de Error}: Se presenta cuando se registró el tipo de evaluación pero no la calificación: ME.1\\
\hline
\textbf{AUTOR} & Hernández Pineda Miguel Angel\\
\hline
\end{longtable}
\vspace*{1cm}
\noindent
\Large{PROCEDIMIENTO ESTÁNDAR}
\large{}
\begin{enumerate}
\item El usuario inicia el caso de uso ver lista de alumnos.
\item El usuario se dirige a la columna de calificación en la tabla de los alumnos inscritos.
\item El usuario selecciona una calificación para el alumno.
\item El usuario se dirige a la columna de Tipo de Evaluación.
\item El usuario selecciona un tipo de evaluación para el alumno.
\item El usuario selecciona da clic en el botón terminar. \textbf{[Trayectoria 1]}
\item *El sistema valida que se haya seleccionado calificación tanto como tipo de evaluación. \textbf{[Trayectoria 2]} \textbf{[Trayectoria 3]}
\item *El sistema guarda la información en la base de datos.
\item Se muestra el mensaje de confirmación MC.1. 
\item Fin del caso de uso.
\end{enumerate}
\vspace*{1cm}
\Large{PROCEDIMIENTOS ALTERNATIVOS}\\
\large{Trayectoria 1}\\
\textbf{Condición}: No a todos los alumnos se les asigna una calificación.
\begin{enumerate}
\item *El sistema muestra el mensaje de alerta MA.5.
\item El usuario da clic en el botón aceptar \textbf{[Trayectoria 4]}
\item Regresa al punto 7 de la trayectoria principal.
\item Fin de la trayectoria.
\end{enumerate}
\large{Trayectoria 2}\\
\textbf{Condición}: Se seleccionó una calificación pero no el tipo de evaluación.
\begin{enumerate}
\item *El sistema muestra el mensaje de error ME.1.
\item *El sistema marca de color rojo los campos que se dejaron en blanco de tipo de evaluación.
\item Regresa al punto 5 de la trayectoria principal.
\item Fin de la trayectoria.
\end{enumerate}
\large{Trayectoria 3}\\
\textbf{Condición}: Se seleccionó una tipo de evaluación pero no la calificación.
\begin{enumerate}
\item *El sistema muestra el mensaje de error ME.1.
\item *El sistema marca de color rojo los campos que se dejaron en blanco de calificación.
\item Regresa al punto 3 de la trayectoria principal.
\item Fin de la trayectoria.
\end{enumerate}
\large{Trayectoria 4}\\
\textbf{Condición}: El usuario dío clic al botón cancelar del mensaje de alerta.
\begin{enumerate}
\item *El sistema cierra el mensaje de alerta.
\item Regresa al punto 6 de la trayectoria principal.
\item Fin de la trayectoria.
\end{enumerate}