\newpage
\section{Caso de Uso: Modificar Materia}
\begin{longtable}{ | p{6cm} | p{10cm} |}
\hline
\textbf{NOMBRE} & Modificar Materia\\
\hline
\textbf{TIPO} & Primario\\
\hline
\textbf{DESCRIPCIÓN} & Se muestran las materias disponibles en la base de datos, para su modificación en los campos de:nombre,nivel y creditos\\
\hline
\textbf{ENTRADAS} & Materia a modificar y campos a modificar\\
\hline
\textbf{SALIDAS} & Se mostrará un mensaje de confirmación si la modificación de la materia seleccionada
Se mostrará un mensaje de error de error cuando falten campos por llenar o el formato no sea correcto\\
\hline
\textbf{PRECONDICIONES} & Estar registrado en el sistema
Registro de materias con anterioridad\\
\hline
\textbf{POSTCONDICIONES} & Modificación en la materia seleccionada\\
\hline
\textbf{SITUACIONES DE ERROR} & \begin{itemize}
    \item Se dejaron campos obligatorios en blanco
    \item El formato de los datos es invalido
    \item No hay materias registradas con anterioridad.
\end{itemize}
\\
\hline
\textbf{ESTADO DEL SISTEMA EN CASO DE ERROR} & No se modifica la materia seleccionada\\
\hline
\textbf{ACTORES} & Gestión Escolar y Analista\\
\hline
\textbf{MENSAJES} & \textbf{Mensasje de Error:}Se muestra en lso siguientes casos: \begin{itemize}
    \item Se dejaron campos en blanco : ME1
    \item El formato de los datos es invalido: ME2
    \item No hay materias registradas: ME10
\end{itemize} 
\textbf{Mensasje de Confirmación:}Se muestra cuando los datos han sidos modificados correctamente MC2
\\
\hline
\textbf{AUTOR} & Matus López Carlos Eduardo\\
\hline
\end{longtable}
\vspace*{1cm}
\noindent
\Large{PROCEDIMIENTO ESTÁNDAR}
\large{}
\begin{enumerate}
    \item El usuario ingresa al sistema como Jefe de Gestión Escolar o Analista
    \item El usuario da click en la opción <<Modificar Materia>> de la sección Administar
    \item El sistema despliega las materias que estan registradas \textbf{[trayectoria 1]}
    \item El usuario selecciona la materia que desea modificar 
    \item *El sistema despliega el formalario de registro de materia
    \item El usuario ingresa el campo de nombre de la materia
    \item El usuario ingresa el campo de nivel de la materia
     \item El usuario ingresa el campo de creditos de la materia
    \item El usuario da clic en el boton de aceptar
    \item *El sistema valida que no haya campos obligatorios vacios.\textbf{[Trayectoria 2]} 
    \item *El sistema valida el campo de nombre \textbf{[Trayectoria 3]} 
    \item *El sistema valida el campo de creditos en base a la RN15 \textbf{[Trayectoria 4]}
    \item Se despliega el mensaje de confirmacíon MC2
    \item Fin del caso de uso.
\end{enumerate}
\Large{PROCEDIMIENTO ALTERNATIVOS}\\

\large{Trayectoria 1}\\
\textbf{Condición:}No hay materias registradas previamente
\large{}
\begin{enumerate}
    \item *El sistema muestra el mensaje de error ME10
    \item *El sistema muestra una pantalla con una tabla vacia
    \item Regresa al punto 14 de la trayectoria principal
    \item Fin de la trayectoria.
\end{enumerate}

\large{Trayectoria 2}\\
\textbf{Condición:}Se dejaron campos obligatorios en blanco
\large{}
\begin{enumerate}
    \item *El sistema muestra el mensaje de error ME1
    \item *El sistema marca de color rojo los campos vacíos.
    \item Regresa al punto 5 de la trayectoria principal
    \item Fin de la trayectoria.
\end{enumerate}

\large{Trayectoria 3}\\
\textbf{Condición:}Se introdujo con un formato incorrecto el nombre de la materia
\begin{enumerate}
    \item *El sistema muestra el mensaje de error ME2
    \item *El sistema marca de color rojo el campo de nombre.
    \item Regresa al punto 6 de la trayectoria principal
    \item Fin de la trayectoria.
\end{enumerate}

\large{Trayectoria 4}\\
\textbf{Condición:}Se introdujo con un formato incorrecto el numero de creditos de la materia
\begin{enumerate}
    \item *El sistema muestra el mensaje de error ME2
    \item *El sistema marca de color rojo el campo de materia.
    \item Regresa al punto 7 de la trayectoria principal
    \item Fin de la trayectoria.
\end{enumerate}