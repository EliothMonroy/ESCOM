\newpage
\section{Caso de Uso Ver Horario y disponibilidadl}
\begin{longtable}{ | p{6cm} | p{10cm} |}
\hline
\textbf{NOMBRE} & Ver Horario y disponibilidad\\
\hline
\textbf{TIPO} & Primario\\
\hline
\textbf{DESCRIPCIÓN} & Se muestran las materias en base al nivel que selecciono el alumno, teniendo en cuenta si se selecciono un o varios de los siguientes filtros:Grupo,Profesor,Materia\\
\hline
\textbf{ENTRADAS} & Nivel y filtros.\\
\hline
\textbf{SALIDAS} & Listado de las materias basadas en la seleccion del alumno\\
\hline
\textbf{PRECONDICIONES} & Estar registrado en el sistema\\
\hline
\textbf{POSTCONDICIONES} & No aplica\\
\hline
\textbf{SITUACIONES DE ERROR} & Se dejo en blanco el campo de nivel de la materia.\\
\hline
\textbf{ESTADO DEL SISTEMA EN CASO DE ERROR} & Se despliega el mensaje de error ME1\\
\hline
\textbf{ACTORES} & Alumno\\
\hline
\textbf{MENSAJES} & \textbf{Mensasje de Error:}Se muestra en el siguiente caso. Se dejaron campos en blanco : ME1\\
\hline
\textbf{AUTOR} & Matus López Carlos Eduardo\\
\hline
\end{longtable}
\vspace*{1cm}
\noindent
\Large{PROCEDIMIENTO ESTÁNDAR}
\large{}
\begin{enumerate}
    \item El alumno ingresa al sistema e inicia sesión.
    \item El alumno da click en la opción <<Horarios y Disponibilidad>> de la sección Reincripción (Figura 3.7)
    \item El sistema despliega el formulario para la búsqueda de materias.
    \item El alumno selecciona el nivel de las materias que quiere visualizar \textbf{[Trayectoria 1]}
    \item El alumno selecciona los filtros que desea para visualizar las materias. \textbf{[Trayectoria 2]}
    \item Se despliega las materias del nivel que eligió el alumno
    \item Fin del caso de uso.
\end{enumerate}
\vspace*{1cm}
\Large{PROCEDIMIENTO ALTERNATIVOS}\\
\large{Trayectoria 1}\\
\textbf{Condición:}Se dejaron campos obligatorios en blanco
\begin{enumerate}
    \item *El sistema muestra el mensaje de error ME1
    \item *El sistema marca de color rojo los campos vacíos.
    \item Regresa al punto 3 de la trayectoria principal
    \item Fin de la trayectoria.
\end{enumerate}
\large{Trayectoria 2}\\
\textbf{Condición:}Se seleccionaron filtros de selección.
\begin{enumerate}
    \item El sistema aplica los filtros de selección a la búsqueda de las materias.
    \item Regresa al punto 6 de la trayectoria principal
    \item Fin de la trayectoria.
\end{enumerate}