\newpage
\section{Caso de Uso Reinscribir}
\begin{longtable}{ | p{6cm} | p{10cm} |}
\hline
\textbf{NOMBRE} & Reinscribir\\
\hline
\textbf{TIPO} & Primario\\
\hline
\textbf{DESCRIPCIÓN} & El usuario inscribe materias para el periodo que está próximo a cursarse.\\
\hline
\textbf{ENTRADAS} & Nombre de la materia: Caracteres\\ & Grupo: Caracteres \\ & Nivel:Selección de una lista numérica.\\ & Horario: Selección de horario de una lista.\\ & Horario armado: Archivo .txt con un horario definido\\
\hline
\textbf{SALIDAS} & Se mostrará un mensaje de confirmación cuando se concluya con el trámite de reinscripción y se generará un archivo PDF con el horario seleccionado.\\ & Se mostrará un mensaje de error cuando los créditos sean distintos a los permitidos, no haya cupo, no se seleccionaron materias, la cita de reinscripcion expiró, el documento seleccionado no es compatible o alguna materia seleccionada esta seriada con una que no ha sido cursada.\\
\hline
\textbf{PRECONDICIONES} & La temporada de reinscripciones esté habilitada y se haya generado una cita de reinscripción al alumno.\\
\hline
\textbf{POSTCONDICIONES} & El alumno tendrá inscritas las materias seleccionadas.\\
\hline
\textbf{SITUACIONES DE ERROR} & \begin{itemize}
    \item La materia seleccionada no tiene cupo.
    \item El documento seleccionado no es compatible.
    \item La cita de reinscripción expiró.
    \item No se seleccionaron materias.
    \item Se inscribieron más o menos créditos de los permitidos.
\end{itemize}\\
\hline
\textbf{ESTADO DEL SISTEMA EN CASO DE ERROR} & No se lleva a cabo la reinscripción.\\
\hline
\textbf{ACTORES} & Alumno, Analista\\
\hline
\textbf{MENSAJES} & \textbf{Mensaje de Alerta}: Se presenta cuando se presiona el botón terminar: MA.2\\ & \textbf{Mensaje de Confirmación}: Se presenta cuando se ha concluido con la reinscripción: MC.20\\ & \textbf{Mensajes de error}: Se presentan en los siguientes casos:\begin{itemize}
    \item La materia seleccionada no tiene cupo: ME.19
    \item La cita de reinscripción expiró: ME.28
    \item No se seleccionaron materias: ME.29
    \item Se inscribieron más créditos de los permitidos: ME.30
    \item Se inscribieron menos créditos de los permitidos: ME.31
\end{itemize}\\
\hline
\textbf{AUTOR} & Hernández Pineda Miguel Angel\\
\hline
\end{longtable}
\vspace*{1cm}
\noindent
\Large{PROCEDIMIENTO ESTÁNDAR PARA ALUMNO}
\large{}
\begin{enumerate}
    \item El alumno inicia sesión el el sistema.
    \item El alumno selecciona la opción <<Reinscripción>> dentro de la sección <<Reinscripción>> (Figura 3.10) \textbf{[Trayectoria 1]}
    \item El alumno lleva a cabo la selección de materias. \textbf{[Trayectoria 2]} \textbf{[Trayectoria 3]}
    \item El alumno presiona el botón terminar. \textbf{[Trayectoria 4]}
    \item *El sistema valida los datos seleccionados. \textbf{[Trayectoria 5]} \textbf{[Trayectoria 6]} \textbf{[Trayectoria 7]}
    \item *El sistema muestra el mensaje de alerta MA.2.
    \item El alumno da click en la opción aceptar. \textbf{[Trayectoria 8]}
    \item *El sistema muestra el mensaje de confirmación MC.20.
    \item Fin del caso de uso.
\end{enumerate}
\vspace*{1cm}
\Large{PROCEDIMIENTO ESTÁNDAR PARA ANALISTA}
\large{}
\begin{enumerate}
    \item El analista inicia sesión en el sistema.
    \item Inicia el caso de uso Administrar Alumno. (Figura 3.16)
    \item El analista da click en la opción inscribir. (Figura 3.10)
    \item Fin del caso de uso Administrar Alumno.
    \item Inicio de Caso de Uso Buscar Materia. (Figura 3.10)
    \item Procedimientos del caso de Uso Buscar Materia.
    \item Fin del Caso de Uso Buscar Materia.
    \item El analista da click en el botón terminar. \textbf{[Trayectoria 1]}
    \item *El sistema valida los datos seleccionados. \textbf{[Trayectoria 2]} \textbf{[Trayectoria 3]} \textbf{[Trayectoria 4]}
    \item *El sistema muestra el mensaje de alerta MA.2.
    \item El analista da click en la opción aceptar. \textbf{[Trayectoria 5]}
    \item *El sistema muestra el mensaje de confirmación MC.20.
    \item Fin del caso de uso.
\end{enumerate}
\vspace*{1cm}
\Large{PROCEDIMIENTOS ALTERNATIVOS PARA ALUMNO}\\
\large{Trayectoria 1}\\
\textbf{Condición}: La cita de reinscripción ha expirado
\begin{enumerate}
    \item *El sistema muestra el mensaje de error ME.28.
    \item Fin del caso de uso.
    \item Fin de la trayectoria.
\end{enumerate}
\large{Trayectoria 2}\\
\textbf{Condición}: Comienza la búsqueda de materias
\begin{enumerate}
    \item Se inicia el Caso de Uso Buscar Materias
    \item Fin del caso de uso Buscar Materia.
    \item Regresa al punto 4 de la trayectoria principal.
    \item Fin de la trayectoria.
\end{enumerate}
\large{Trayectoria 3}\\
\textbf{Condición}: Se carga un horario ya establecido.
\begin{enumerate}
    \item El alumno da click en el botón cargar.
    \item El alumno selecciona un horario creado anteriormente.
    \item *El sistema carga el archivo.
    \item Regresa al punto 4 de la trayectoria principal.
    \item Fin de la trayectoria.
\end{enumerate}
\large{Trayectoria 4}\\
\textbf{Condición}: No se han seleccionado materias.
\begin{enumerate}
    \item *El sistema muestra el mensaje de error ME.29.
    \item Regresa al punto 3 de la trayectoria principal.
    \item Fin de la trayectoria.
\end{enumerate}
\large{Trayectoria 5}\\
\textbf{Condición}: Una o más materias seleccionadas no tienen cupo.
\begin{enumerate}
    \item *El sistema muestra el mensaje de error ME.19.
    \item *El sistema elimina de la selección la materia que no tiene cupo.
    \item Regresa al punto 3 de la trayectoria principal.
    \item Fin de la trayectoria.
\end{enumerate}
\large{Trayectoria 6}\\
\textbf{Condición}: El horario seleccionado tiene mas créditos de los permitidos según las reglas de negocio de RN.3 a RN.6, RN.7 y RN.8..
\begin{enumerate}
    \item *El sistema muestra el mensaje de error ME.30.
    \item Regresa al punto 3 de la trayectoria principal.
    \item Fin de la trayectoria.
\end{enumerate}
\large{Trayectoria 7}\\
\textbf{Condición}: El horario seleccionado tiene menos créditos de los permitido según las reglas de negocio de RN.3 a RN.6, RN.7 y RN.8.s.
\begin{enumerate}
    \item *El sistema muestra el mensaje de error ME.31.
    \item Regresa al punto 3 de la trayectoria principal.
    \item Fin de la trayectoria.
\end{enumerate}
\large{Trayectoria 8}\\
\textbf{Condición}: El alumno da click en la opción cancelar.
\begin{enumerate}
    \item *El sistema cierra el mensaje de alerta.
    \item Regresa al punto 3 de la trayectoria principal.
    \item Fin de la trayectoria.
\end{enumerate}
\newpage
\Large{PROCEDIMIENTOS ALTERNATIVOS PARA ANALISTA}\\
\large{Trayectoria 1}\\
\textbf{Condición}: No se han seleccionado materias.
\begin{enumerate}
    \item *El sistema muestra el mensaje de error ME.29.
    \item Regresa al punto 3 de la trayectoria principal.
    \item Fin de la trayectoria.
\end{enumerate}
\large{Trayectoria 2}\\
\textbf{Condición}: Una o más materias seleccionadas no tienen cupo.
\begin{enumerate}
    \item *El sistema muestra el mensaje de error ME.19.
    \item *El sistema elimina de la selección la materia que no tiene cupo.
    \item Regresa al punto 3 de la trayectoria principal.
    \item Fin de la trayectoria.
\end{enumerate}
\large{Trayectoria 3}\\
\textbf{Condición}: El horario seleccionado tiene mas créditos de los permitidos según las reglas de negocio de RN.3 a RN.6, RN.7 y RN.8.
\begin{enumerate}
    \item *El sistema muestra el mensaje de error ME.30.
    \item Regresa al punto 3 de la trayectoria principal.
    \item Fin de la trayectoria.
\end{enumerate}
\large{Trayectoria 4}\\
\textbf{Condición}: El horario seleccionado tiene menos créditos de los permitidos según las reglas de negocio de RN.3 a RN.6, RN.7 y RN.8..
\begin{enumerate}
    \item *El sistema muestra el mensaje de error ME.31.
    \item Regresa al punto 3 de la trayectoria principal.
    \item Fin de la trayectoria.
\end{enumerate}
\large{Trayectoria 5}\\
\textbf{Condición}: El alumno da click en la opción cancelar.
\begin{enumerate}
    \item *El sistema cierra el mensaje de alerta.
    \item Regresa al punto 3 de la trayectoria principal.
    \item Fin de la trayectoria.
\end{enumerate}