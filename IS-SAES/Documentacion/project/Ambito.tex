\chapter{Ámbito del Software}
\noindent
El módulo de reinscripción del Sistema de Administración Escolar (en adelante SAES) del Instituto Politécnico Nacional (en adelante IPN) se volverá a hacer añadiendo mejoras y corrigiendo los errores que se presentan en el mismo actualmente.\\
El proyecto se desarrollará desde la fase de análisis hasta la fase de pruebas, por lo que será necesario tener algunos otros módulos como el registro de los distintos tipos de usuario, así como algunas funciones de estos; tendremos los módulos de registro de alumno y administrador, así como el de iniciar sesión, además es necesario contar con algunas de las funciones del administrador para poder llevar a cabo el registro de materias, profesores y otros datos.
Centrándonos en el módulo de reinscripción, se trabajará en el basándonos estrictamente según las normas establecidas en el Reglamento General de Estudios del IPN donde se definen algunos criterios a considerar para poder llevar a cabo este proceso.\\
Las funciones esenciales del alumno serán todas las relacionadas con la reinscripción tomando algunas de las que ya han sido establecidas en el módulo actual, así como nuevas funciones que permitan una mejor navegación y manejo de las distintas secciones dentro del sistema.\\
El sistema le permitirá al alumno buscar información de las materias del semestre al que se va a inscribir un par de semanas antes de que llegue la fecha de inscripciones, en ese mismo momento será posible llevar a cabo el armado de uno o varios horarios tentativos de acuerdo a sus preferencias para que llegada su cita de inscripción verifique si puede inscribir alguno de estos, además, para hacer más ágil este proceso se contarán con diferentes filtros para que la búsqueda de materias, profesores o grupos sea más fácil durante este proceso; se llevarán a cabo las validaciones pertinentes para informarle al alumno si es posible reinscribirse a través del SAES, en caso de que no se le permita, el sistema le informará que debe acudir a la oficina de gestión escolar de su unidad académica para llevar a cabo este proceso. 
\newpage
\noindent
De acuerdo a lo establecido actualmente, las citas de reinscripción se generarán según el promedio del alumno, teniendo prioridad los alumnos con un mejor promedio. Durante el proceso de reinscripción, el alumno será capaz de administrar las materias que desea inscribir, es decir, podrá agregar o eliminar materias a su lista de reinscripción, de acuerdo a los créditos que tenga disponibles, antes de que verifique y termine con su reinscripción.\\
Finalmente, el sistema le mostrará, durante el periodo de reinscripción, diferentes opciones de horario al alumno de acuerdo al semestre que cursará y basándose en la información de los cupos de cada materia en el momento, así como en el horario asignado a cada una de estas, reduciendo así, las horas muertas se verán reducidas en estas opciones, es importante resaltar que solo son recomendaciones o sugerencias para el alumno por lo que es su decisión si desea inscribir o no el horario. 
