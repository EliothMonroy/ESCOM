\section{Conclusiones}
El programa desarrollado en esta práctica, funcionó como un primer acercamiento hacia el formato de archivos wav, el cuál es un formato de audio digital. El familiarizarnos con el mismo nos permite poder realizar trabajos posteriores con este, debido a que es una forma conveniente de representar señales.\\ Durante este primer acercamiento, pude determinar como leer la información contenida en el archivo wav (las cabeceras y datos), además de comprender como tratar esta información para poderla modificar y así generar nuevos archivos en formato wav.\\ Una parte importante de este análisis, fue la lectura de cabeceras del archivo wav, ya que dependiendo del tamaño de la cabecera se tiene que usar un tipo de dato capaz de almacenarla, para las cabeceras de 4 bytes use variables de tipo int y para las cabeceras de 2 bytes use variables de tipo short (cada una puede almacenar hasta 4 y 2 bytes respectivamente).