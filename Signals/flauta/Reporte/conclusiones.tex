\section{Conclusiones}
Como se pudo observar en la prueba realizada, el programa detectó correctamente todas las notas tocadas en la grabación, por lo cual podemos decir que el objetivo del proyecto se cumplió, al menos hasta el alcance establecido (detectar todas las notas tocadas por una flauta dulce en una grabación). Pero el trabajo realizado puede ser extendido en proyectos más complejos, los cuales permitan determinar que notas y por cuanto tiempo fueron tocadas, siendo útil para la generación de partituras o archivos midi a partir de una grabación contenida en un archivo wav. Además de que el proyecto puede ser extendido no solo para la detección de notas musicales tocadas por una flauta dulce, si no para más instrumentos de viento.\\ La parte más compleja de realizar del proyecto fue determinar la resolución necesaria para evitar el traslape de las notas, debido a que algunas notas tienen una frecuencia muy cercana a la de otras, y si no es determinada correctamente la resolución es posible que se generen traslapes entre las mismas y el programa puede confundir distintas notas o marcar que una nota que no fue tocada fue detectada.