\section{Código}
El proyecto fue implementado en dos módulos.\\ El primero desarrollado en python el cual consiste en la grabación de audio y el segundo desarrollado en c el cual es el encargado de realizar el analisis en frecuencia del archivo de grabación generado por el primer módulo. A continuación se presenta el código de ambos.\\
record.py:
\begin{lstlisting}[style=CStyle]
import pyaudio
import wave
from subprocess import call
CHUNK = 1024
FORMAT = pyaudio.paInt16
CHANNELS = 1
RATE = 4000
RECORD_SECONDS = 10
WAVE_OUTPUT_FILENAME = "output.wav"
p = pyaudio.PyAudio()
stream = p.open(format=FORMAT,
channels=CHANNELS,
rate=RATE,
input=True,
frames_per_buffer=CHUNK)
print("* recording")
frames = []
for i in range(0, int(RATE / CHUNK * RECORD_SECONDS)):
	data = stream.read(CHUNK)
	frames.append(data)

print("* done recording")

stream.stop_stream()
stream.close()
p.terminate()

wf = wave.open(WAVE_OUTPUT_FILENAME, 'wb')
wf.setnchannels(CHANNELS)
wf.setsampwidth(p.get_sample_size(FORMAT))
wf.setframerate(RATE)
wf.writeframes(b''.join(frames))
wf.close()

call(["flauta.exe", "output.wav"])

\end{lstlisting}
funciones.h:
\begin{lstlisting}[style=CStyle]
#ifndef __FUNCIONES_H__
#define __FUNCIONES_H__
	//Librerías de C
	#include <stdio.h>
	#include <stdlib.h>
	#include <math.h>
	#include <string.h>
	//Librería que contiene los máximos y mínimos de los diferentes tipos de datos en c
	#include <limits.h>
	//Libreria para conocer tiempo de ejecución
	#include <time.h>
	//Metodos
	void leerCabeceras(char**);
	void escribirArchivo(short*,short*);
	void leerMuestras(short*);
	void leerMuestras2Canales(short*,short*);
	void convertirFloat(short*,float*,float*,int);
	void convertirShort(short*,short*,float*,float*,int);
	//Cabeceras
	int chunkid;
	int chunksize;
	int format;
	int subchunk1id;
	int subchunk1size;
	short audioformat;
	short numchannels;
	int samplerate;
	int byterate;
	short blockalign;
	short bitspersample;
	int subchunk2id;
	int subchunk2size;
	//Archivo
	FILE* entrada;
	FILE* salida;
	//Variables para muestras
	short muestra;
	int aux_conteo=0;
	int total_muestras_originales;
	int total_muestras;
	int hayheaders=0;
	short headers[37];
	//Métodos TDF
	#define PI acos(-1.0)//Defino la constante PI
	void calcularFFT(short*,int);
	void calcularFFTI(short*,short*);
	int calcularNuevoNumeroMuestras(int);
	void obtenerNota(short*,int);
	//Inversión de bits
	#define SWAP(x,y) do {typeof(x) _x = x;typeof(y) _y = y;x = _y;y = _x;} while(0)
	//Variables para obtener tiempo de ejecución
	clock_t inicio, final;
	double total;
	//Notas
	float duracion;
	int aux1;
	int amp=3000;
	int bandera=0;
	//Arreglos de frecuencias
	int f_do[3] = {262,523,1046};
	int f_doG[3] = {277,554,1108};
	int f_re[3] = {294,587,1174};
	int f_reG[3] = {311,622,1244};
	int f_mi[3] = {330,659,1318};
	int f_fa[3] = {349,698,1396};
	int f_faG[3] = {370,740,1480};
	int f_sol[3] = {392,784,1568};
	int f_solG[3] = {416,831,1662};
	int f_la[3] = {440,880,1760};
	int f_laG[3] = {466,932,1864};
	int f_si[3] = {494,988,1976};
	//Métodos detectar nota
	void esDo3(short*);
	void esDoG3(short*);
	void esRe3(short*);
	void esReG3(short*);
	void esMi3(short*);
	void esFa3(short*);
	void esFaG3(short*);
	void esSol3(short*);
	void esSolG3(short*);
	void esLa3(short*);
	void esLaG3(short*);
	void esSi3(short*);
	void esDo4(short*);
	void esDoG4(short*);
	void esRe4(short*);
	void esReG4(short*);
	void esMi4(short*);
	void esFa4(short*);
	void esFaG4(short*);
	void esSol4(short*);
	void esSolG4(short*);
	void esLa4(short*);
	void esLaG4(short*);
	void esSi4(short*);
	void esDo5(short*);
	void esDoG5(short*);
	void esRe5(short*);
	void esReG5(short*);
	void esMi5(short*);
	void esFa5(short*);
	void esFaG5(short*);
	void esSol5(short*);
	void esSolG5(short*);
	void esLa5(short*);
	void esLaG5(short*);
	void esSi5(short*);
#endif
\end{lstlisting}
flauta.c:
\begin{lstlisting}[style=CStyle]
#include"funciones.h"
int main(int argc, char *argv[]){
	//Leo las cabeceras
	leerCabeceras(argv);
	//Defino variables
	total_muestras_originales=subchunk2size/blockalign;
	printf("Total muestras originales:%d\n",total_muestras_originales);
	//Necesitamos que el total de muestras sea una potencia de 2
	total_muestras=calcularNuevoNumeroMuestras(total_muestras_originales);
	printf("Nuevo total de muestras:%d\n", total_muestras);
	short *muestras=(short *)malloc(total_muestras * sizeof(short));
	//Leo las muestras
	int t=256;
	leerMuestras(muestras);
	short *aux_muestras=(short *)malloc(t * sizeof(short));
	int i;
	int j;
	for (i = 0; i < total_muestras/t; i++){
		for(j=0;j<t;j++){
			aux_muestras[j]=muestras[aux_conteo];
			aux_conteo+=1;
		}
		calcularFFT(aux_muestras,t);
	}
}
void leerCabeceras(char ** argv){
	entrada = fopen(argv[1], "rb");
	if(!entrada) {
		perror("\nFile opening failed");
		exit(0);
	}
	fread(&chunkid,sizeof(int),1,entrada);
	fread(&chunksize,sizeof(int),1,entrada);
	fread(&format,sizeof(int),1,entrada);
	fread(&subchunk1id,sizeof(int),1,entrada);
	fread(&subchunk1size,sizeof(int),1,entrada);
	fread(&audioformat,sizeof(short),1,entrada);
	fread(&numchannels,sizeof(short),1,entrada);
	fread(&samplerate,sizeof(int),1,entrada);
	fread(&byterate,sizeof(int),1,entrada);
	fread(&blockalign,sizeof(short),1,entrada);
	fread(&bitspersample,sizeof(short),1,entrada);
	fread(&subchunk2id,sizeof(int),1,entrada);
	fread(&subchunk2size,sizeof(int),1,entrada);
}
void leerMuestras(short *muestras){
	int i=0;
	while (feof(entrada) == 0){
		if(i<total_muestras_originales){
			fread(&muestra,sizeof(short),1,entrada);
			muestras[i]=muestra;
			i++;
		}else{
			hayheaders=1;
			fread(&headers,sizeof(short),37,entrada);
			break;
		}
	}
	//Ajuste por si las muestras originales no fueron potencia de dos
	if(total_muestras_originales<total_muestras){
		for (i = total_muestras_originales; i < total_muestras; i++){
			muestras[i]=0;
		}
	}
	fclose(entrada);
}
void escribirArchivo(short* muestrasRe,short* muestrasIm){
	//Escribo el archivo
	fwrite(&chunkid,sizeof(int),1,salida);
	fwrite(&chunksize,sizeof(int),1,salida);
	fwrite(&format,sizeof(int),1,salida);
	fwrite(&subchunk1id,sizeof(int),1,salida);
	fwrite(&subchunk1size,sizeof(int),1,salida);
	fwrite(&audioformat,sizeof(short),1,salida);
	fwrite(&numchannels,sizeof(short),1,salida);
	fwrite(&samplerate,sizeof(int),1,salida);
	fwrite(&byterate,sizeof(int),1,salida);
	fwrite(&blockalign,sizeof(short),1,salida);
	fwrite(&bitspersample,sizeof(short),1,salida);
	fwrite(&subchunk2id,sizeof(int),1,salida);
	fwrite(&subchunk2size,sizeof(int),1,salida);
	//Ahora escribo las muestras
	int i=0;
	for(i=0;i<total_muestras;i++){
		fwrite(&muestrasRe[i],sizeof(short),1,salida);
		fwrite(&muestrasIm[i],sizeof(short),1,salida);
	}
	//Y por último los headers de goldwave
	if(hayheaders){
		for(i=0;i<37;i++){
			fwrite(&headers[i],sizeof(short),1,salida);
		}
	}
	fclose(salida);
}
void calcularFFT(short *muestras_recibidas, int total_muestras_recibidas){
	//Aquí va el algoritmo para la FFT
	float *Xre=(float *)malloc(total_muestras_recibidas * sizeof(float));
	float *Xim=(float *)malloc(total_muestras_recibidas * sizeof(float));
	int i;
	//Convierto las muestras de short a float
	convertirFloat(muestras_recibidas, Xre, Xim,total_muestras_recibidas);
	//FFT
	int j, k, fk, m, n, ce, c, w;
	float arg, seno, coseno, tempr, tempi;
	//Bit reversal
	m=log((float)total_muestras_recibidas)/log(2.0);
	j=w=0;
	for (i = 0; i < total_muestras_recibidas; i++){
		if (j>i){
			SWAP(Xre[i],Xre[j]);
			SWAP(Xim[i],Xim[j]);
		}
		w=total_muestras_recibidas/2;
		while(w>=2 && j>=w){
			j-=w;
			w>>=1;
		}
		j+=w;
	}
	ce=m;
	c=0;
	//Mariposas
	for (i = 0; i < m; i++) {
		for(j = 0; j < (int)pow(2,ce-1); j++){
			n = (int)pow(2,i);
			for(k = 0; k < n; k++){
				fk=k*(int)pow(2,ce-1);
				coseno=cos((-1)*2*PI*fk/total_muestras_recibidas);
				seno=sin((-1)*2*PI*fk/total_muestras_recibidas);
				tempr=Xre[c+n];
				Xre[c+n]=(Xre[c+n]*coseno) - (Xim[c+n]*seno);
				Xim[c+n]=(Xim[c+n]*coseno) + (tempr*seno);
				tempr=(Xre[c]+Xre[c+n])/2;
				tempi=(Xim[c]+Xim[c+n])/2;
				Xre[c+n]=(Xre[c]-Xre[c+n])/2;
				Xim[c+n]=(Xim[c]-Xim[c+n])/2;
				Xre[c]=tempr;
				Xim[c]=tempi;
				c++;
			}
			c += n;
		}
		c = 0;
		ce -= 1;
	}
	short *Reales=(short *)malloc(total_muestras_recibidas * sizeof(short));
	short *Imaginarias=(short *)malloc(total_muestras_recibidas * sizeof(short));
	//Regreso las muestras calculadas a short
	convertirShort(Reales,Imaginarias,Xre,Xim,total_muestras_recibidas);
	obtenerNota(Reales,total_muestras_recibidas);
}
int calcularNuevoNumeroMuestras(int total){
	if ((total & (total-1))==0){
		puts("Ya es potencia de 2");
	}else{
		puts("No es potencia de 2");
		int i;
		i=(int)ceil((float)log(total_muestras_originales)/(float)log(2));
		printf("i:%d\n", i);
		total=pow(2,i);
	}
	return total;
}
void convertirFloat(short *muestras, float *Xre, float *Xim, int total_muestras_recibidas){
	int i;
	for (i = 0; i < total_muestras_recibidas; i++){
		Xre[i]=(float)muestras[i]/(float)(SHRT_MAX);
		Xim[i]=0.0;
	}
}
void convertirShort(short *Reales, short *Imaginarias, float *Xre, float *Xim, int total_muestras_recibidas){
	int i;
	for (i = 0; i < total_muestras_recibidas; i++){
		Reales[i]=Xre[i]*(SHRT_MAX);
		Imaginarias[i]=Xim[i]*(SHRT_MAX);
	}
}
void obtenerNota(short *Xre,int muestras_recibidas){
	//Obtengo la duración del archivo
	duracion=(float)muestras_recibidas/(float)samplerate;
	//printf("Duracion del archivo: %f\n", duracion);
	int i;
	for(i=0;i<muestras_recibidas/2;i++){
		if(Xre[i]>amp)
		printf("Xre[%d] = %d\n", i, Xre[i]);
	}
	//Aquí reviso que notas fueron identificadas
	esDo3(Xre);
	esDoG3(Xre);
	esRe3(Xre);
	esReG3(Xre);
	esMi3(Xre);
	esFa3(Xre);
	esFaG3(Xre);
	esSol3(Xre);
	esSolG3(Xre);
	esLa3(Xre);
	esLaG3(Xre);
	esSi3(Xre);
	esDo4(Xre);
	esDoG4(Xre);
	esRe4(Xre);
	esReG4(Xre);
	esMi4(Xre);
	esFa4(Xre);
	esFaG4(Xre);
	esSol4(Xre);
	esSolG4(Xre);
	esLa4(Xre);
	esLaG4(Xre);
	esSi4(Xre);
	esDo5(Xre);
	esDoG5(Xre);
	esRe5(Xre);
	esReG5(Xre);
	esMi5(Xre);
	esFa5(Xre);
	esFaG5(Xre);
	esSol5(Xre);
	esSolG5(Xre);
	esLa5(Xre);
	esLaG5(Xre);
	esSi5(Xre);
}
void esDo3(short *Xre){
	aux1=round((float)f_do[0]*duracion);
	if(Xre[aux1]>amp || Xre[aux1+1]>amp){
		puts("La nota corresponde a un do3");
		bandera=1;
	}
}
void esDoG3(short *Xre){
	aux1=round((float)f_doG[0]*duracion);
	if(Xre[aux1]>amp || Xre[aux1+1]>amp){
		puts("La nota corresponde a un do#3");
		bandera=1;
	}
}
void esRe3(short *Xre){
	aux1=round((float)f_re[0]*duracion);
	if(Xre[aux1]>amp || Xre[aux1+1]>amp){
		puts("La nota corresponde a un re3");
		bandera=1;
	}
}
void esReG3(short *Xre){
	aux1=round((float)f_reG[0]*duracion);
	if(Xre[aux1]>amp || Xre[aux1+1]>amp){
		puts("La nota corresponde a un re#3");
		bandera=1;
	}
}
void esMi3(short *Xre){
	aux1=round((float)f_mi[0]*duracion);
	if(Xre[aux1]>amp || Xre[aux1+1]>amp){
		puts("La nota corresponde a un mi3");
		bandera=1;
	}
}
void esFa3(short *Xre){
	aux1=round((float)f_fa[0]*duracion);
	if(Xre[aux1]>amp || Xre[aux1+1]>amp){
		puts("La nota corresponde a un fa3");
		bandera=1;
	}
}
void esFaG3(short *Xre){
	aux1=round((float)f_faG[0]*duracion);
	if(Xre[aux1]>amp || Xre[aux1+1]>amp){
		puts("La nota corresponde a un fa#3");
		bandera=1;
	}
}
void esSol3(short *Xre){
	aux1=round((float)f_sol[0]*duracion);
	if(Xre[aux1]>amp || Xre[aux1+1]>amp){
		puts("La nota corresponde a un sol3");
		bandera=1;
	}
}
void esSolG3(short *Xre){
	aux1=round((float)f_solG[0]*duracion);
	if(Xre[aux1]>amp || Xre[aux1+1]>amp){
		puts("La nota corresponde a un sol#3");
		bandera=1;
	}
}
void esLa3(short *Xre){
	aux1=round((float)f_la[0]*duracion);
	if(Xre[aux1]>amp || Xre[aux1+1]>amp){
		puts("La nota corresponde a un la3");
		bandera=1;
	}
}
void esLaG3(short *Xre){
	aux1=round((float)f_laG[0]*duracion);
	if(Xre[aux1]>amp || Xre[aux1+1]>amp){
		puts("La nota corresponde a un la#3");
		bandera=1;
	}
}
void esSi3(short *Xre){
	aux1=round((float)f_si[0]*duracion);
	if(Xre[aux1]>amp || Xre[aux1+1]>amp){
		puts("La nota corresponde a un si3");
		bandera=1;
	}
}
void esDo4(short *Xre){
	aux1=round((float)f_do[1]*duracion);
	if(Xre[aux1]>amp || Xre[aux1+1]>amp){
		puts("La nota corresponde a un do4");
		bandera=1;
	}
}
void esDoG4(short *Xre){
	aux1=round((float)f_doG[0]*duracion);
	if(Xre[aux1]>amp || Xre[aux1+1]>amp){
		puts("La nota corresponde a un do#4");
		bandera=1;
	}
}
void esRe4(short *Xre){
	aux1=round((float)f_re[1]*duracion);
	if(Xre[aux1]>amp || Xre[aux1+1]>amp){
		puts("La nota corresponde a un re4");
		bandera=1;
	}
}
void esReG4(short *Xre){
	aux1=round((float)f_re[1]*duracion);
	if(Xre[aux1]>amp || Xre[aux1+1]>amp){
		puts("La nota corresponde a un re#4");
		bandera=1;
	}
}
void esMi4(short *Xre){
	aux1=round((float)f_mi[1]*duracion);
	if(Xre[aux1]>amp || Xre[aux1+1]>amp){
		puts("La nota corresponde a un mi4");
		bandera=1;
	}
}
void esFa4(short *Xre){
	aux1=round((float)f_fa[1]*duracion);
	if(Xre[aux1]>amp || Xre[aux1+1]>amp){
		puts("La nota corresponde a un fa4");
		bandera=1;
	}
}
void esFaG4(short *Xre){
	aux1=round((float)f_faG[1]*duracion);
	if(Xre[aux1]>amp || Xre[aux1+1]>amp){
		puts("La nota corresponde a un fa#4");
		bandera=1;
	}
}
void esSol4(short *Xre){
	aux1=round((float)f_sol[1]*duracion);
	if(Xre[aux1]>amp || Xre[aux1+1]>amp){
		puts("La nota corresponde a un sol4");
		bandera=1;
	}
}
void esSolG4(short *Xre){
	aux1=round((float)f_solG[1]*duracion);
	if(Xre[aux1]>amp || Xre[aux1+1]>amp){
		puts("La nota corresponde a un sol#4");
		bandera=1;
	}
}
void esLa4(short *Xre){
	aux1=round((float)f_la[1]*duracion);
	if(Xre[aux1]>amp || Xre[aux1+1]>amp){
		puts("La nota corresponde a un la4");
		bandera=1;
	}
}
void esLaG4(short *Xre){
	aux1=round((float)f_laG[1]*duracion);
	if(Xre[aux1]>amp || Xre[aux1+1]>amp){
		puts("La nota corresponde a un la#4");
		bandera=1;
	}
}
void esSi4(short *Xre){
	aux1=round((float)f_si[1]*duracion);
	if(Xre[aux1]>amp || Xre[aux1+1]>amp){
		puts("La nota corresponde a un si4");
		bandera=1;
	}
}
void esDo5(short *Xre){
	aux1=round((float)f_do[2]*duracion);
	if(Xre[aux1]>amp || Xre[aux1+1]>amp){
		puts("La nota corresponde a un do5");
		bandera=1;
	}
}
void esDoG5(short *Xre){
	aux1=round((float)f_doG[0]*duracion);
	if(Xre[aux1]>amp || Xre[aux1+1]>amp){
		puts("La nota corresponde a un do#5");
		bandera=1;
	}
}
void esRe5(short *Xre){
	aux1=round((float)f_re[2]*duracion);
	if(Xre[aux1]>amp || Xre[aux1+1]>amp){
		puts("La nota corresponde a un re5");
		bandera=1;
	}
}
void esReG5(short *Xre){
	aux1=round((float)f_reG[2]*duracion);
	if(Xre[aux1]>amp || Xre[aux1+1]>amp){
		puts("La nota corresponde a un re#5");
		bandera=1;
	}
}
void esMi5(short *Xre){
	aux1=round((float)f_mi[2]*duracion);
	if(Xre[aux1]>amp || Xre[aux1+1]>amp){
		puts("La nota corresponde a un mi5");
		bandera=1;
	}
}
void esFa5(short *Xre){
	aux1=round((float)f_fa[2]*duracion);
	if(Xre[aux1]>amp || Xre[aux1+1]>amp){
		puts("La nota corresponde a un fa5");
		bandera=1;
	}
}
void esFaG5(short *Xre){
	aux1=round((float)f_faG[2]*duracion);
	if(Xre[aux1]>amp || Xre[aux1+1]>amp){
		puts("La nota corresponde a un fa#5");
		bandera=1;
	}
}
void esSol5(short *Xre){
	aux1=round((float)f_sol[2]*duracion);
	if(Xre[aux1]>amp || Xre[aux1+1]>amp){
		puts("La nota corresponde a un sol5");
		bandera=1;
	}
}
void esSolG5(short *Xre){
	aux1=round((float)f_solG[2]*duracion);
	if(Xre[aux1]>amp || Xre[aux1+1]>amp){
		puts("La nota corresponde a un sol#5");
		bandera=1;
	}
}
void esLa5(short *Xre){
	aux1=round((float)f_la[2]*duracion);
	if(Xre[aux1]>amp || Xre[aux1+1]>amp){
		puts("La nota corresponde a un la5");
		bandera=1;
	}
}
void esLaG5(short *Xre){
	aux1=round((float)f_laG[2]*duracion);
	if(Xre[aux1]>amp || Xre[aux1+1]>amp){
		puts("La nota corresponde a un la#5");
		bandera=1;
	}
}
void esSi5(short *Xre){
	aux1=round((float)f_si[2]*duracion);
	if(Xre[aux1]>amp || Xre[aux1+1]>amp){
		puts("La nota corresponde a un si5");
		bandera=1;
	}
}
\end{lstlisting}