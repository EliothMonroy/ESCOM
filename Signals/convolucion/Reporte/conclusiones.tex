\section{Conclusiones}
Los filtros pasabajas tienen una gran cantidad de aplicaciones posibles, filtrado de ruido, en comunicaciones, electrónica, etc. Debido a que permiten el paso de una banda de frecuencias menores a la frecuencia de corte del filtro. Las frecuencias mayores a la misma son filtradas (atenuadas). Por lo cual, son una herramienta importante en el trabajo con comunicaciones y señales, donde muchas veces resulta de interés, el solo obtener cierto rango de frecuencias de una grabación o una transmisión.\\ Por ejemplo, cuando se realiza una grabación de audio mediante un micrófono en una habitación con una lampara eléctrica, la cual podría generar ``ruido'' en la grabación al estar encendida. Por lo cual, resultaría útil filtrar el espectro de frecuencias producido por la lampara de la grabación.\\ La implementación más sencilla de un filtro pasabajas es mediante un circuito RC, en el cual, la frecuencia de corte del filtro estará dada por los valores de la resistencia y el capacitor, en aplicaciones de electrónica primero se propondría un valor para el capacitor y posteriormente el de la resistencia para así obtener la frecuencia de corte deseada.\\ Sin embargo, con el avance de la tecnología ahora resulta mucho más sencillo el simular un circuito RC, ya que puede ser hecho mediante software en pocas lineas de código mediante el uso de la convolución.