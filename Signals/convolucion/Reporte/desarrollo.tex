\section{Código}
El programa elaborado en esta práctica, consistió en simular un filtro pasabajas mediante la convolución de dos señales. Una de ellas recibida como un parámetro de entrada y la otra sería la respuesta al impulso de un circuito RC (filtro pasabajas), la cual fue definida dentro del programa (dentro de la función ``calcularFiltro()''). Esto con el fin de simular un filtro pasabajas con una frecuencia de corte de 1000Hz.\\ A continuación, se presenta el código elaborado.\\
funciones.h:
\begin{lstlisting}[style=CStyle]
#ifndef __FUNCIONES_H__
#define __FUNCIONES_H__	
	//Librerías de C
	#include <stdio.h>
	#include <stdlib.h>
	#include <math.h>
	//Métodos
	void leerCabeceras(char**);
	void escribirArchivo(short*);
	//Cabeceras
	int chunkid;
	int chunksize;
	int format;
	int subchunk1id;
	int subchunk1size;
	short audioformat;
	short numchannels;
	int samplerate;
	int byterate;
	short blockalign;
	short bitspersample;
	int subchunk2id;
	int subchunk2size;
	//Archivo
	FILE* entrada;
	FILE* salida;
	//Variables para muestras
	short muestra;
	int total_muestras;
	short headers[37];
	//Filtro
	#define PI acos(-1.0)//Defino la constante PI
	#define TOTAL_COEFICIENTES 20
	double filtro[TOTAL_COEFICIENTES];//Arreglo para el filtro
	double suma_filtro;
	double e;
	void calcularFiltro();
	void calcularNuevasMuestras(short*);
#endif
\end{lstlisting}
filtro-psb.c:
\begin{lstlisting}[style=CStyle]
#include"funciones.h"
int main(int argc, char *argv[]){
	//Leo las cabeceras
	leerCabeceras(argv);
	//Defino variables
	int i=0;
	total_muestras=subchunk2size/blockalign;
	short muestras[total_muestras];
	//Leo las muestras
	while (feof(entrada) == 0)
	{
		if(i<total_muestras){
			fread(&muestra,sizeof(short),1,entrada);
			muestras[i]=muestra;
			i++;
		}else{
			fread(&headers,sizeof(short),37,entrada);
			break;
		}
	}
	calcularFiltro();
	calcularNuevasMuestras(muestras);
}
void leerCabeceras(char ** argv){
	entrada = fopen(argv[1], "rb");
	salida=fopen(argv[2],"wb");
	if(!entrada) {
		perror("\nFile opening failed");
		exit(0);
	}
	fread(&chunkid,sizeof(int),1,entrada);
	fread(&chunksize,sizeof(int),1,entrada);
	fread(&format,sizeof(int),1,entrada);
	fread(&subchunk1id,sizeof(int),1,entrada);
	fread(&subchunk1size,sizeof(int),1,entrada);
	fread(&audioformat,sizeof(short),1,entrada);
	fread(&numchannels,sizeof(short),1,entrada);
	fread(&samplerate,sizeof(int),1,entrada);
	fread(&byterate,sizeof(int),1,entrada);
	fread(&blockalign,sizeof(short),1,entrada);
	fread(&bitspersample,sizeof(short),1,entrada);
	fread(&subchunk2id,sizeof(int),1,entrada);
	fread(&subchunk2size,sizeof(int),1,entrada);
}
void escribirArchivo(short* muestras){
	//Escribo cabeceras del archivo
	fwrite(&chunkid,sizeof(int),1,salida);
	fwrite(&chunksize,sizeof(int),1,salida);
	fwrite(&format,sizeof(int),1,salida);
	fwrite(&subchunk1id,sizeof(int),1,salida);
	fwrite(&subchunk1size,sizeof(int),1,salida);
	fwrite(&audioformat,sizeof(short),1,salida);
	fwrite(&numchannels,sizeof(short),1,salida);
	fwrite(&samplerate,sizeof(int),1,salida);
	fwrite(&byterate,sizeof(int),1,salida);
	fwrite(&blockalign,sizeof(short),1,salida);
	fwrite(&bitspersample,sizeof(short),1,salida);
	fwrite(&subchunk2id,sizeof(int),1,salida);
	fwrite(&subchunk2size,sizeof(int),1,salida);
	//Ahora escribo las muestras
	int i=0;
	for(i=0;i<total_muestras;i++){
		fwrite(&muestras[i],sizeof(short),1,salida);
	}
	//Y por último los headers de goldwave
	for(i=0;i<37;i++){
		fwrite(&headers[i],sizeof(short),1,salida);
	}
}
void calcularFiltro(){
	e=exp(1);//Aquí obtengo el valor de e 
	int i;
	for (i= 0; i < 20; i++){
		filtro[i]=(2000*PI)*pow(e,((-1)*2000*PI*i)/44100);
		suma_filtro+=filtro[i];
		printf("%f\n", filtro[i]);
	}
}
void calcularNuevasMuestras(short* muestras){
	//Algoritmo para la convolución usando Input Side Algorithm
	short nuevas_muestras[total_muestras];
	int i,j;
	//Primero inicializar a cero el arreglo
	for (i = 0; i < total_muestras; i++){
		nuevas_muestras[i]=0;
	}
	//Ahora si el algoritmo
	for (i = 0; i < total_muestras; i++){
		for (j = 0; j < TOTAL_COEFICIENTES; j++){
			nuevas_muestras[i+j]+=(muestras[i]*filtro[j])/suma_filtro;
		}
	}
	//Escribo el archivo
	escribirArchivo(nuevas_muestras);
}
\end{lstlisting}
La convolución entre las señales fue realizada mediante el Input Side Algorithm$ ^{1}$.