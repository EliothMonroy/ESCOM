\section{Código}
La implementación del filtro mediante multiplicación de complejos en frecuencia se realizó con los siguientes programas:\\
funciones.h:
\begin{lstlisting}[style=CStyle]
#ifndef __FUNCIONES_H__
#define __FUNCIONES_H__
	//Librerías de C
	#include <stdio.h>
	#include <stdlib.h>
	#include <math.h>
	//Librería que contiene los máximos y mínimos de los diferentes tipos de datos en c
	#include <limits.h>
	//Libreria para conocer tiempo de ejecución
	#include <time.h>
	//Metodos
	void leerCabeceras(char**);
	void escribirArchivo(short*,short*);
	void leerMuestras(short*);
	void leerMuestras2Canales(short*,short*);
	//Cabeceras
	int chunkid;
	int chunksize;
	int format;
	int subchunk1id;
	int subchunk1size;
	short audioformat;
	short numchannels;
	int samplerate;
	int byterate;
	short blockalign;
	short bitspersample;
	int subchunk2id;
	int subchunk2size;
	//Archivo
	FILE* entrada;
	FILE* salida;
	int modo;
	//Variables para muestras
	short muestra;
	int total_muestras;
	short headers[37];
	//Métodos TDF
	#define PI acos(-1.0)//Defino la constante PI
	void calcularTDF(short*);
	void calcularTDF1(short*);
	void calcularTDF2(short*);
	void calcularTDF3(short*);
	void calcularTDFI(short*,short*);
#endif
\end{lstlisting}
tdf.c:
\begin{lstlisting}[style=CStyle]
#include"funciones.h"
int main(int argc, char *argv[]){
	//Leo las cabeceras
	leerCabeceras(argv);
	//Defino variables    
	total_muestras=subchunk2size/blockalign;
	short *muestras=(short *)malloc(total_muestras * sizeof(short));
	printf("Total de muestras:%d\n", total_muestras);
	//Leo las muestras
	leerMuestras(muestras);
	//Calculo la TDF
	if (modo==0){
		//Transformada normal
		calcularTDF(muestras);
	}else if (modo==1){
		//Canal izquierdo señal original, derecho magnitud
		calcularTDF1(muestras);
	}else if(modo==2){
		//Canal izquierdo parte real TDF, derecho magnitud
		calcularTDF2(muestras);
	}else if(modo==3){
		//Canal izquierdo magnitud TDF, derecho fase TDF
		calcularTDF3(muestras);
	}
}
void leerCabeceras(char ** argv){
	//Primero voy a recordar a leer archivos
	entrada = fopen(argv[1], "rb");
	salida=fopen(argv[2],"wb");
	modo=atoi(argv[3]);
	if(!entrada){
		perror("\nFile opening failed");
		exit(0);
	}
	fread(&chunkid,sizeof(int),1,entrada);
	fread(&chunksize,sizeof(int),1,entrada);
	fread(&format,sizeof(int),1,entrada);
	fread(&subchunk1id,sizeof(int),1,entrada);
	fread(&subchunk1size,sizeof(int),1,entrada);
	fread(&audioformat,sizeof(short),1,entrada);
	fread(&numchannels,sizeof(short),1,entrada);
	fread(&samplerate,sizeof(int),1,entrada);
	fread(&byterate,sizeof(int),1,entrada);
	fread(&blockalign,sizeof(short),1,entrada);
	fread(&bitspersample,sizeof(short),1,entrada);
	fread(&subchunk2id,sizeof(int),1,entrada);
	fread(&subchunk2size,sizeof(int),1,entrada);
}
void leerMuestras(short *muestras){
	int i=0;
	while (feof(entrada) == 0){
		if(i<total_muestras){
			fread(&muestra,sizeof(short),1,entrada);
			muestras[i]=muestra;
			i++;
		}else{
			fread(&headers,sizeof(short),37,entrada);
			break;
		}
	}
}
void escribirArchivo(short* muestrasRe,short* muestrasIm){
	//Escribo el archivo
	fwrite(&chunkid,sizeof(int),1,salida);
	fwrite(&chunksize,sizeof(int),1,salida);
	fwrite(&format,sizeof(int),1,salida);
	fwrite(&subchunk1id,sizeof(int),1,salida);
	fwrite(&subchunk1size,sizeof(int),1,salida);
	fwrite(&audioformat,sizeof(short),1,salida);
	fwrite(&numchannels,sizeof(short),1,salida);
	fwrite(&samplerate,sizeof(int),1,salida);
	fwrite(&byterate,sizeof(int),1,salida);
	fwrite(&blockalign,sizeof(short),1,salida);
	fwrite(&bitspersample,sizeof(short),1,salida);
	fwrite(&subchunk2id,sizeof(int),1,salida);
	fwrite(&subchunk2size,sizeof(int),1,salida);
	//Ahora escribo las muestras
	int i=0;
	for(i=0;i<total_muestras;i++){
		fwrite(&muestrasRe[i],sizeof(short),1,salida);
		fwrite(&muestrasIm[i],sizeof(short),1,salida);
	}
	//Y por último los headers de goldwave
	for(i=0;i<37;i++){
		fwrite(&headers[i],sizeof(short),1,salida);
	}
}
void calcularTDF(short* muestras){
	//Aquí va el algoritmo para la TDF
	short *Xre=(short *)malloc(total_muestras * sizeof(short));
	short *Xim=(short *)malloc(total_muestras * sizeof(short));
	//Variables para obtener tiempo de ejecución
	clock_t inicio, final;
	double total;
	inicio = clock();
	//Algoritmo TDF
	int n,k;
	for (k = 0; k < total_muestras; k++){
			Xre[k]=0;
			Xim[k]=0;
		for (n = 0; n < total_muestras; n++){
			Xre[k]+=(muestras[n]/total_muestras)*cos(2*k*n*PI/total_muestras);
			Xim[k]-=(muestras[n]/total_muestras)*sin(2*k*n*PI/total_muestras);
		}	
	}
	//Obtener tiempo e imprimir
	final = clock();
	total = (double)(final - inicio) / CLOCKS_PER_SEC;
	printf("Tiempo de ejecucion: %f\n", total);
	//La salida ahora sera un archivo tipo estereo (2 canales)
	//Por lo cual hay que cambiar el el numero de canales del archivo y todas las demas cabeceras que dependan de esta
	chunksize-=subchunk2size;
	numchannels*=2;
	byterate*=numchannels;
	blockalign*=numchannels;
	subchunk2size*=numchannels;
	chunksize+=subchunk2size;
	escribirArchivo(Xre,Xim);
}
void calcularTDF1(short *muestras){
	//Canal izquierdo señal original, derecho magnitud
	short *Xre=(short *)malloc(total_muestras * sizeof(short));
	short *Xim=(short *)malloc(total_muestras * sizeof(short));
	short *magnitud=(short *)malloc(total_muestras * sizeof(short));
	//Variables para obtener tiempo de ejecución
	clock_t inicio, final;
	double total;
	inicio = clock();
	//Algoritmo TDF
	int n,k;
	for (k = 0; k < total_muestras; k++){
		Xre[k]=0;
		Xim[k]=0;
		for (n = 0; n < total_muestras; n++){
			Xre[k]+=(muestras[n]/total_muestras)*cos(2*k*n*PI/total_muestras);
			Xim[k]-=(muestras[n]/total_muestras)*sin(2*k*n*PI/total_muestras);
		}
	}
	//Obtener tiempo e imprimir
	final = clock();
	total = (double)(final - inicio) / CLOCKS_PER_SEC;
	printf("Tiempo de ejecucion: %f\n", total);
	//Ahora calcularé la magnitud de la transformada
	for (k = 0; k < total_muestras; k++){
		magnitud[k]=sqrt(pow(Xre[k],2)+pow(Xim[k],2));
	}
	//La salida ahora sera un archivo tipo estereo (2 canales)
	//Por lo cual hay que cambiar el numero de canales del archivo 
	//y todas las demas cabeceras que dependan de esta
	chunksize-=subchunk2size;
	numchannels*=2;
	byterate*=numchannels;
	blockalign*=numchannels;
	subchunk2size*=numchannels;
	chunksize+=subchunk2size;
	escribirArchivo(muestras,magnitud);
}
void calcularTDF2(short *muestras){
	//Canal izquierdo parte real transformada, derecho magnitud
	short *Xre=(short *)malloc(total_muestras * sizeof(short));
	short *Xim=(short *)malloc(total_muestras * sizeof(short));
	short *magnitud=(short *)malloc(total_muestras * sizeof(short));
	//Variables para obtener tiempo de ejecución
	clock_t inicio, final;
	double total;
	inicio = clock();
	//Algoritmo TDF
	int n,k;
	for (k = 0; k < total_muestras; k++){
		Xre[k]=0;
		Xim[k]=0;
		for (n = 0; n < total_muestras; n++){
			Xre[k]+=(muestras[n]/total_muestras)*cos(2*k*n*PI/total_muestras);
			Xim[k]-=(muestras[n]/total_muestras)*sin(2*k*n*PI/total_muestras);
		}
	}
	//Obtener tiempo e imprimir
	final = clock();
	total = (double)(final - inicio) / CLOCKS_PER_SEC;
	printf("Tiempo de ejecucion: %f\n", total);
	//Ahora calcularé la magnitud de la transformada
	for (k = 0; k < total_muestras; k++){
		magnitud[k]=sqrt(pow(Xre[k],2)+pow(Xim[k],2));
	}
	//La salida ahora sera un archivo tipo estereo (2 canales)
	//Por lo cual hay que cambiar el numero de canales del archivo 
	//y todas las demas cabeceras que dependan de esta
	chunksize-=subchunk2size;
	numchannels*=2;
	byterate*=numchannels;
	blockalign*=numchannels;
	subchunk2size*=numchannels;
	chunksize+=subchunk2size;
	escribirArchivo(Xre,magnitud);
}
void calcularTDF3(short *muestras){
	//Canal izquierdo magnitud, derecho fase
	short *Xre=(short *)malloc(total_muestras * sizeof(short));
	short *Xim=(short *)malloc(total_muestras * sizeof(short));
	short *magnitud=(short *)malloc(total_muestras * sizeof(short));
	short *fase=(short *)malloc(total_muestras * sizeof(short));
	//Variables para obtener tiempo de ejecución
	clock_t inicio, final;
	double total;
	inicio = clock();
	//Algoritmo TDF
	int n,k;
	for (k = 0; k < total_muestras; k++){
		Xre[k]=0;
		Xim[k]=0;
		for (n = 0; n < total_muestras; n++){
			Xre[k]+=(muestras[n]/total_muestras)*cos(2*k*n*PI/total_muestras);
			Xim[k]-=(muestras[n]/total_muestras)*sin(2*k*n*PI/total_muestras);
		}
	}
	//Obtener tiempo e imprimir
	final = clock();
	total = (double)(final - inicio) / CLOCKS_PER_SEC;
	printf("Tiempo de ejecucion: %f\n", total);
	//Ahora calcularé la magnitud de la transformada
	for (k = 0; k < total_muestras; k++){
		magnitud[k]=sqrt(pow(Xre[k],2)+pow(Xim[k],2));
	}
	//La fase
	float valor=180.0/PI;
	//printf("Valor %f",valor);
	for (k = 0; k < total_muestras; k++){
		if (magnitud[k]>1000){
			//atan nos devuelve un valor en radianes, hay que pasarlo a grados
			if(Xre[k]==0){
				if (Xim[k]==0){
					fase[k]=0;
				}else if (Xim[k]<0){//Xim es negativa
					fase[k]=-90;//o 270, tengo que checarlo
				}else{
					fase[k]=90;
				}
			}else{
				//printf("arctan:%f\n", atan(Xim[k]/Xre[k]));
				fase[k]=atan(Xim[k]/Xre[k])*valor;
			}
		}else{
			fase[k]=0;
		}
		fase[k]=(fase[k]*SHRT_MAX)/180;//Para que se pueda ver en goldwave
	}
	//La salida ahora sera un archivo tipo estereo (2 canales)
	//Por lo cual hay que cambiar el numero de canales del archivo 
	//y todas las demas cabeceras que dependan de esta
	chunksize-=subchunk2size;
	numchannels*=2;
	byterate*=numchannels;
	blockalign*=numchannels;
	subchunk2size*=numchannels;
	chunksize+=subchunk2size;
	escribirArchivo(magnitud,fase);
}
\end{lstlisting}
tdfi.c:
\begin{lstlisting}[style=CStyle]
#include"funciones.h"
int main(int argc, char *argv[]){
	//Leo las cabeceras
	leerCabeceras(argv);
	//Defino variables    
	total_muestras=subchunk2size/blockalign;
	//printf("Total Muestras: %d\n", total_muestras);
	short *muestrasRe=(short *)malloc(total_muestras * sizeof(short));
	short *muestrasIm=(short *)malloc(total_muestras * sizeof(short));
	//short muestrasRe[total_muestras],muestrasIm[total_muestras];
	//Leo las muestras
	leerMuestras2Canales(muestrasRe,muestrasIm);
	//Calculo la TDF
	calcularTDFI(muestrasRe,muestrasIm);
}
void leerCabeceras(char ** argv){
	//Primero voy a recordar a leer archivos
	entrada = fopen(argv[1], "rb");
	salida=fopen(argv[2],"wb");
	if(!entrada) {
		perror("\nFile opening failed");
		exit(0);
	}
	fread(&chunkid,sizeof(int),1,entrada);
	fread(&chunksize,sizeof(int),1,entrada);
	fread(&format,sizeof(int),1,entrada);
	fread(&subchunk1id,sizeof(int),1,entrada);
	fread(&subchunk1size,sizeof(int),1,entrada);
	fread(&audioformat,sizeof(short),1,entrada);
	fread(&numchannels,sizeof(short),1,entrada);
	fread(&samplerate,sizeof(int),1,entrada);
	fread(&byterate,sizeof(int),1,entrada);
	fread(&blockalign,sizeof(short),1,entrada);
	fread(&bitspersample,sizeof(short),1,entrada);
	fread(&subchunk2id,sizeof(int),1,entrada);
	fread(&subchunk2size,sizeof(int),1,entrada);
}
void leerMuestras2Canales(short *muestrasRe,short* muestrasIm){
	int i=0;
	while (feof(entrada) == 0){
		if(i<total_muestras){
			fread(&muestrasRe[i],sizeof(short),1,entrada);
			fread(&muestrasIm[i],sizeof(short),1,entrada);
			i++;
			//printf("Muestra %s: %d\n",i,muestras[i-1]);
		}else{
			fread(&headers,sizeof(short),37,entrada);
			break;
		}
	}
}
void escribirArchivo(short* muestrasRe,short* muestrasIm){
	//Escribo el archivo
	fwrite(&chunkid,sizeof(int),1,salida);
	fwrite(&chunksize,sizeof(int),1,salida);
	fwrite(&format,sizeof(int),1,salida);
	fwrite(&subchunk1id,sizeof(int),1,salida);
	fwrite(&subchunk1size,sizeof(int),1,salida);
	fwrite(&audioformat,sizeof(short),1,salida);
	fwrite(&numchannels,sizeof(short),1,salida);
	fwrite(&samplerate,sizeof(int),1,salida);
	fwrite(&byterate,sizeof(int),1,salida);
	fwrite(&blockalign,sizeof(short),1,salida);
	fwrite(&bitspersample,sizeof(short),1,salida);
	fwrite(&subchunk2id,sizeof(int),1,salida);
	fwrite(&subchunk2size,sizeof(int),1,salida);
	//Ahora escribo las muestras
	int i=0;
	for(i=0;i<total_muestras;i++){
		fwrite(&muestrasRe[i],sizeof(short),1,salida);
		fwrite(&muestrasIm[i],sizeof(short),1,salida);
	}
	//Y por último los headers de goldwave
	for(i=0;i<37;i++){
		fwrite(&headers[i],sizeof(short),1,salida);
	}
}
void calcularTDFI(short* muestrasRe,short* muestrasIm){
	//Aquí va el algoritmo para la TDF
	short *Xre=(short *)malloc(total_muestras * sizeof(short));
	short *Xim=(short *)malloc(total_muestras * sizeof(short));
	//Variables para obtener tiempo de ejecución
	clock_t inicio, final;
	double total;
	inicio = clock();
	//Algoritmo TDFI
	int n,k;
	double algo;
	for (k = 0; k < total_muestras; k++){
		Xre[k]=0;
		Xim[k]=0;
		for (n = 0; n < total_muestras; n++){
			//Lo que va dentro del coseno y seno.
			algo=(2*k*n*PI)/(total_muestras);
			//Calculo las partes real e imaginarias
			Xre[k]+=muestrasRe[n]*cos(algo)-muestrasIm[n]*sin(algo);
			Xim[k]+=muestrasRe[n]*sin(algo)+muestrasIm[n]*cos(algo);
		}
	}
	//Obtener tiempo e imprimir
	final = clock();
	total = (double)(final - inicio) / CLOCKS_PER_SEC;
	printf("Tiempo de ejecucion: %f\n", total);
	//La salida seguirá siendo un archivo tipo estereo (2 canales) por lo cual no hay que alinear nada
	escribirArchivo(Xre,Xim);
}
\end{lstlisting}
funciones2.h:
\begin{lstlisting}[style=CStyle]
#ifndef __FUNCIONES2_H__
#define __FUNCIONES2_H__
	//Librerías de C
	#include <stdio.h>
	#include <stdlib.h>
	//Librería que contiene los máximos y mínimos de los diferentes tipos de datos en c
	#include <limits.h>
	//Metodos
	void leerCabeceras(char**);
	void leerMuestras(short*,short*);
	void leerMuestras2Canales(short*,short*,short*,short*);
	void escribirArchivo(short*,int);
	void escribirArchivo2Canales(short*,short*,int);
	void calcularProducto(short*,short*);
	void calcularProductoComplejos(short*,short*,short*,short*);
	//Cabeceras primer archivo
	int chunkid1;
	int chunksize1;
	int format1;
	int subchunk1id1;
	int subchunk1size1;
	short audioformat1;
	short numchannels1;
	int samplerate1;
	int byterate1;
	short blockalign1;
	short bitspersample1;
	int subchunk2id1;
	int subchunk2size1;
	//Cabeceras segundo archivo
	int chunkid2;
	int chunksize2;
	int format2;
	int subchunk1id2;
	int subchunk1size2;
	short audioformat2;
	short numchannels2;
	int samplerate2;
	int byterate2;
	short blockalign2;
	short bitspersample2;
	int subchunk2id2;
	int subchunk2size2;
	//Archivo
	FILE* entrada1;
	FILE* entrada2;
	FILE* salida;
	//Variables para muestras primer archivo
	int total_muestras1;
	//Variables segundo archivo
	int total_muestras2;
	//Variables que se pueden reutilizar
	short muestra;
	short headers[37];
#endif
\end{lstlisting}
producto.c:
\begin{lstlisting}[style=CStyle]
#include"funciones2.h"
int main(int argc, char *argv[]){
	//Leo las cabeceras
	leerCabeceras(argv);
	if(numchannels1==1){
		//Defino variables para el primer archivo
		total_muestras1=subchunk2size1/blockalign1;
		short *muestras1=(short *)malloc(total_muestras1 * sizeof(short));
		//Defino variables para el segundo archivo
		total_muestras2=subchunk2size2/blockalign2;
		short *muestras2=(short *)malloc(total_muestras2 * sizeof(short));
		//Leo las muestras del primer archivos
		leerMuestras(muestras1,muestras2);
		//Calculo el producto
		calcularProducto(muestras1,muestras2);
	}else if(numchannels1==2){
		//Defino variables para el primer archivo
		total_muestras1=subchunk2size1/blockalign1;
		short *muestrasRe1=(short *)malloc(total_muestras1 * sizeof(short));
		short *muestrasIm1=(short *)malloc(total_muestras1 * sizeof(short));
		//Defino variables para el segundo archivo
		total_muestras2=subchunk2size2/blockalign2;
		short *muestrasRe2=(short *)malloc(total_muestras2 * sizeof(short));
		short *muestrasIm2=(short *)malloc(total_muestras2 * sizeof(short));
		//Leo las muestras del primer archivos
		leerMuestras2Canales(muestrasRe1,muestrasIm1,muestrasRe2,muestrasIm2);
		//Calculo el producto
		calcularProductoComplejos(muestrasRe1,muestrasIm1,muestrasRe2,muestrasIm2);
	}
}
void leerCabeceras(char ** argv){
	//Primero voy a recordar a leer archivos
	entrada1 = fopen(argv[1], "rb");
	entrada2 = fopen(argv[2], "rb");
	salida=fopen(argv[3],"wb");
	if(!entrada1 & !entrada2){
		perror("\nFile opening failed");
		exit(0);
	}
	//Lectura de cabeceras primer archivo
	fread(&chunkid1,sizeof(int),1,entrada1);
	fread(&chunksize1,sizeof(int),1,entrada1);
	fread(&format1,sizeof(int),1,entrada1);
	fread(&subchunk1id1,sizeof(int),1,entrada1);
	fread(&subchunk1size1,sizeof(int),1,entrada1);
	fread(&audioformat1,sizeof(short),1,entrada1);
	fread(&numchannels1,sizeof(short),1,entrada1);
	fread(&samplerate1,sizeof(int),1,entrada1);
	fread(&byterate1,sizeof(int),1,entrada1);
	fread(&blockalign1,sizeof(short),1,entrada1);
	fread(&bitspersample1,sizeof(short),1,entrada1);
	fread(&subchunk2id1,sizeof(int),1,entrada1);
	fread(&subchunk2size1,sizeof(int),1,entrada1);
	//Lectura de cabeceras segundo archivo
	fread(&chunkid2,sizeof(int),1,entrada2);
	fread(&chunksize2,sizeof(int),1,entrada2);
	fread(&format2,sizeof(int),1,entrada2);
	fread(&subchunk1id2,sizeof(int),1,entrada2);
	fread(&subchunk1size2,sizeof(int),1,entrada2);
	fread(&audioformat2,sizeof(short),1,entrada2);
	fread(&numchannels2,sizeof(short),1,entrada2);
	fread(&samplerate2,sizeof(int),1,entrada2);
	fread(&byterate2,sizeof(int),1,entrada2);
	fread(&blockalign2,sizeof(short),1,entrada2);
	fread(&bitspersample2,sizeof(short),1,entrada2);
	fread(&subchunk2id2,sizeof(int),1,entrada2);
	fread(&subchunk2size2,sizeof(int),1,entrada2);
}
void leerMuestras(short *muestras1, short *muestras2){
	int i=0;
	//Lectura muestras primer archivo
	while (feof(entrada1) == 0){
		if(i<total_muestras1){
			fread(&muestra,sizeof(short),1,entrada1);
			printf("Muestra %d del primer archivo:%hi\n",i+1,muestra);
			muestras1[i]=muestra;
			i++;
		}else{
			fread(&headers,sizeof(short),37,entrada1);
			break;
		}
	}
	//Lectura muestras segundo archivo
	i=0;
	while (feof(entrada2) == 0){
		if(i<total_muestras2){
			fread(&muestra,sizeof(short),1,entrada2);
			printf("Muestra %d segundo archivo:%hi\n", muestra);
			muestras2[i]=muestra;
			i++;
		}else{
			fread(&headers,sizeof(short),37,entrada2);
			break;
		}
	}
}
void leerMuestras2Canales(short *muestrasRe1,short* muestrasIm1,short* muestrasRe2, short* muestrasIm2){
	int i=0;
	//Lectura de muestras del primer archivo
	while (feof(entrada1) == 0){
		if(i<total_muestras1){
			fread(&muestrasRe1[i],sizeof(short),1,entrada1);
			fread(&muestrasIm1[i],sizeof(short),1,entrada1);
			//printf("Muestra Real %d: %hi\n",i+1,muestrasRe1[i]);
			//printf("Muestra Imaginaria %d: %hi\n",i+1,muestrasIm1[i]);
			i++;		
		}else{
			fread(&headers,sizeof(short),37,entrada1);
			break;
		}
	}
	i=0;
	//Lectura de muestras del segundo archivo
	while (feof(entrada2) == 0){
		if(i<total_muestras2){
			fread(&muestrasRe2[i],sizeof(short),1,entrada2);
			fread(&muestrasIm2[i],sizeof(short),1,entrada2);
			i++;
		}else{
			fread(&headers,sizeof(short),37,entrada2);
			break;
		}
	}
}
void escribirArchivo(short* muestras,int decision){
	//Escribo las cabeceras del archivo de salida
	if (decision==0){
		//Se escriben las cabeceras del primer archivo
		fwrite(&chunkid1,sizeof(int),1,salida);
		fwrite(&chunksize1,sizeof(int),1,salida);
		fwrite(&format1,sizeof(int),1,salida);
		fwrite(&subchunk1id1,sizeof(int),1,salida);
		fwrite(&subchunk1size1,sizeof(int),1,salida);
		fwrite(&audioformat1,sizeof(short),1,salida);
		fwrite(&numchannels1,sizeof(short),1,salida);
		fwrite(&samplerate1,sizeof(int),1,salida);
		fwrite(&byterate1,sizeof(int),1,salida);
		fwrite(&blockalign1,sizeof(short),1,salida);
		fwrite(&bitspersample1,sizeof(short),1,salida);
		fwrite(&subchunk2id1,sizeof(int),1,salida);
		fwrite(&subchunk2size1,sizeof(int),1,salida);
		//Ahora escribo las muestras
		int i=0;
		for(i=0;i<total_muestras1;i++){
			fwrite(&muestras[i],sizeof(short),1,salida);
		}
		//Y por último los headers de goldwave
		for(i=0;i<37;i++){
			fwrite(&headers[i],sizeof(short),1,salida);
		}
	}else{
		//Se escriben las cabeceras del segundo archivo
		fwrite(&chunkid2,sizeof(int),1,salida);
		fwrite(&chunksize2,sizeof(int),1,salida);
		fwrite(&format2,sizeof(int),1,salida);
		fwrite(&subchunk1id2,sizeof(int),1,salida);
		fwrite(&subchunk1size2,sizeof(int),1,salida);
		fwrite(&audioformat2,sizeof(short),1,salida);
		fwrite(&numchannels2,sizeof(short),1,salida);
		fwrite(&samplerate2,sizeof(int),1,salida);
		fwrite(&byterate2,sizeof(int),1,salida);
		fwrite(&blockalign2,sizeof(short),1,salida);
		fwrite(&bitspersample2,sizeof(short),1,salida);
		fwrite(&subchunk2id2,sizeof(int),1,salida);
		fwrite(&subchunk2size2,sizeof(int),1,salida);
		//Ahora escribo las muestras
		int i=0;
		for(i=0;i<total_muestras2;i++){
			fwrite(&muestras[i],sizeof(short),1,salida);
		}
		//Y por último los headers de goldwave
		for(i=0;i<37;i++){
			fwrite(&headers[i],sizeof(short),1,salida);
		}
	}    
}
void escribirArchivo2Canales(short* muestrasRe, short* muestrasIm,int decision){
	//Escribo las cabeceras del archivo de salida
	if (decision==0){
		//Se escriben las cabeceras del primer archivo
		fwrite(&chunkid1,sizeof(int),1,salida);
		fwrite(&chunksize1,sizeof(int),1,salida);
		fwrite(&format1,sizeof(int),1,salida);
		fwrite(&subchunk1id1,sizeof(int),1,salida);
		fwrite(&subchunk1size1,sizeof(int),1,salida);
		fwrite(&audioformat1,sizeof(short),1,salida);
		fwrite(&numchannels1,sizeof(short),1,salida);
		fwrite(&samplerate1,sizeof(int),1,salida);
		fwrite(&byterate1,sizeof(int),1,salida);
		fwrite(&blockalign1,sizeof(short),1,salida);
		fwrite(&bitspersample1,sizeof(short),1,salida);
		fwrite(&subchunk2id1,sizeof(int),1,salida);
		fwrite(&subchunk2size1,sizeof(int),1,salida);
		//Ahora escribo las muestras
		int i=0;
		for(i=0;i<total_muestras1;i++){
			fwrite(&muestrasRe[i],sizeof(short),1,salida);
			fwrite(&muestrasIm[i],sizeof(short),1,salida);
		}
		//Y por último los headers de goldwave
		for(i=0;i<37;i++){
			fwrite(&headers[i],sizeof(short),1,salida);
		}
	}else{
		//Se escriben las cabeceras del segundo archivo
		fwrite(&chunkid2,sizeof(int),1,salida);
		fwrite(&chunksize2,sizeof(int),1,salida);
		fwrite(&format2,sizeof(int),1,salida);
		fwrite(&subchunk1id2,sizeof(int),1,salida);
		fwrite(&subchunk1size2,sizeof(int),1,salida);
		fwrite(&audioformat2,sizeof(short),1,salida);
		fwrite(&numchannels2,sizeof(short),1,salida);
		fwrite(&samplerate2,sizeof(int),1,salida);
		fwrite(&byterate2,sizeof(int),1,salida);
		fwrite(&blockalign2,sizeof(short),1,salida);
		fwrite(&bitspersample2,sizeof(short),1,salida);
		fwrite(&subchunk2id2,sizeof(int),1,salida);
		fwrite(&subchunk2size2,sizeof(int),1,salida);
		//Ahora escribo las muestras
		int i=0;
		for(i=0;i<total_muestras2;i++){
			fwrite(&muestrasRe[i],sizeof(short),1,salida);
			fwrite(&muestrasIm[i],sizeof(short),1,salida);
		}
		//Y por último los headers de goldwave
		for(i=0;i<37;i++){
			fwrite(&headers[i],sizeof(short),1,salida);
		}
	}    
}
void calcularProducto(short* muestras1, short* muestras2){
	int i;
	//Reviso que archivo tiene más muestras
	if(total_muestras1<=total_muestras2){
		//El segundo archivo tiene más
		puts("El segundo archivo tiene mas muestras");
		short *resultado=(short *)malloc(total_muestras2 * sizeof(short));
		for (i = 0; i < total_muestras2; i++){
			if(i<total_muestras1){
				resultado[i]=(muestras1[i]*muestras2[i])/(SHRT_MAX+2);
			}else{
				resultado[i]=0;
			}
		}
		//Escribo el resultado y cabeceras segundo archivo
		escribirArchivo(resultado,1);
	}else{
	//El primer archivo tiene más
	puts("El primer archivo tiene mas muestras");
	short *resultado=(short *)malloc(total_muestras1 * sizeof(short));
	for (i = 0; i < total_muestras1; i++){
		if(i<total_muestras2){
			resultado[i]=(muestras1[i]*muestras2[i])/(SHRT_MAX+2);
		}else{
			resultado[i]=0;
		}
	}
	//Escribo el resultado y cabeceras primer archivo
	escribirArchivo(resultado,0);
	}
}
void calcularProductoComplejos(short* muestrasRe1, short* muestrasIm1, short* muestrasRe2, short* muestrasIm2){
	int i;
	//Reviso que archivo tiene más muestras
	if(total_muestras1<=total_muestras2){
	//El segundo archivo tiene más
	puts("El segundo archivo tiene mas muestras o igual numero de muestras");
	short *resultadoRe=(short *)malloc(total_muestras2 * sizeof(short));
	short *resultadoIm=(short *)malloc(total_muestras2 * sizeof(short));
	for (i = 0; i < total_muestras2; i++){
		if(i<total_muestras1){
			//Tenemos que hacer producto de complejos
			resultadoRe[i]=((muestrasRe1[i]*muestrasRe2[i])-(muestrasIm1[i]*muestrasIm2[i]))/(SHRT_MAX+2);
			resultadoIm[i]=((muestrasRe1[i]*muestrasIm2[i])+(muestrasRe2[i]*muestrasIm1[i]))/(SHRT_MAX+2);
		}else{
			resultadoRe[i]=0;
			resultadoIm[i]=0;
		}
	}
	//Escribo el resultado y cabeceras segundo archivo
	escribirArchivo2Canales(resultadoRe,resultadoIm,1);
	}else{
		//El primer archivo tiene más
		puts("El primer archivo tiene mas muestras");
		short *resultadoRe=(short *)malloc(total_muestras1 * sizeof(short));
		short *resultadoIm=(short *)malloc(total_muestras1 * sizeof(short));
	for (i = 0; i < total_muestras1; i++){
		if(i<total_muestras2){
			//Tenemos que hacer producto de complejos
			resultadoRe[i]=((muestrasRe1[i]*muestrasRe2[i])-(muestrasIm1[i]*muestrasIm2[i]))/(SHRT_MAX+2);
			resultadoIm[i]=((muestrasRe1[i]*muestrasIm2[i])+(muestrasRe2[i]*muestrasIm1[i]))/(SHRT_MAX+2);//FFT: 5650
		}else{
			resultadoRe[i]=0;
			resultadoIm[i]=0;
		}
	}
	//Escribo el resultado y cabeceras primer archivo
	escribirArchivo2Canales(resultadoRe,resultadoIm,0);
	}
}
\end{lstlisting}