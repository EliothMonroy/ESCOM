\section{Conclusiones}
El análisis de Fourier es una herramienta útil no solo en las matemáticas, si no en muchos otros campos de la ciencia e ingeniería, como lo son el análisis y detección de voz, de instrumentos musicales, y de otros sonidos, teniendo como principal condición, que estos tengan una frecuencia mediante la cual se puedan identificar. Para poderlo aplicar en la computación, es necesario el uso de la Transformada Discreta de Fourier (TDF) la cual nos permite trabajar con señales discretas (se caracterizan por tener información cada cierto tiempo, según sea su frecuencia de muestreo), esto quiere decir que en lugar de tener una señal continua que tiene un valor en cada instante de tiempo (es decir, que contengan información infinita), tenemos una señal que solo contiene información cada cierto tiempo, gracias a lo cual, la señal tiene una cantidad de información finita, lo que permite que esta pueda ser procesada por una computadora (recordando que las computadoras cuentan con una memoria finita, lo cual les impide trabajar con una cantidad ilimitada de información).\\ El efecto que tiene la TDF sobre la señal de entrada a la que se le aplique, es que primeramente, el espectro en frecuencia de la señal de entrada se volverá periódico. Y segundo, la TDF al igual que la TF transforman una señal en el domino del tiempo al dominio de la frecuencia. Siendo esto muy útil debido a que es posible realizar un analisis en frecuencia a la señal de entrada.\\ Además, se puedo visualizar la equivalencia entre la convolución en el tiempo y el producto en frecuencia, Ya que cuando se realiza la convolución en el tiempo entre dos señales, uno de los efectos que esto causa, es que ambos espectros de frecuencia se multiplican como producto de complejos. y viceversa, uno de los efectos de realizar el producto en frecuencia, es que es lo mismo que convolucionar en el tiempo.\\ Como principal defecto, la TDF es un algoritmo con una gran complejidad (O(N$ ^{2} $)), por lo cual es un algoritmo muy lento en su ejecución, este tiempo aumenta conforme aumenta el número de muestras para las que se tienen que calcular la TDF.\\