\section{Código}
Para la realización de esta práctica, se hizo uso del programa de multiplicación desarrollado en la práctica anterior (04A), y se modifico además el programa de la TDF para que esta imprimiera el tiempo de ejecución del algoritmo, los resultados de esto se pueden apreciar en la sección de pruebas.\\ Para esta práctica se desarrollaron dos programas, uno que calcula la Trasformada Rápida de Fourier (FFT) y otro que calcula la Transformada Rápida de Fourier Inversa (FFTI). De lo cuales se anexa el código elaborado a continuación:\\
funciones.h:
\begin{lstlisting}[style=CStyle]
#ifndef __FUNCIONES_H__
#define __FUNCIONES_H__
	//Librerías de C
	#include <stdio.h>
	#include <stdlib.h>
	#include <math.h>
	#include <string.h>
	//Librería que contiene los máximos y mínimos de los diferentes tipos de datos en c
	#include <limits.h>
	//Libreria para conocer tiempo de ejecución
	#include <time.h>
	//Metodos
	void leerCabeceras(char**);
	void escribirArchivo(short*,short*);
	void leerMuestras(short*);
	void leerMuestras2Canales(short*,short*);
	void convertirFloat(short*,float*,float*);
	void convertirShort(short*,short*,float*,float*);
	//Cabeceras
	int chunkid;
	int chunksize;
	int format;
	int subchunk1id;
	int subchunk1size;
	short audioformat;
	short numchannels;
	int samplerate;
	int byterate;
	short blockalign;
	short bitspersample;
	int subchunk2id;
	int subchunk2size;
	//Archivo
	FILE* entrada;
	FILE* salida;
	//Variables para muestras
	short muestra;
	int total_muestras_originales;
	int total_muestras;
	short headers[37];
	//Métodos TDF
	#define PI acos(-1.0)//Defino la constante PI
	void calcularFFT(short*);
	void calcularFFTI(short*,short*);
	int calcularNuevoNumeroMuestras(int);
	void intercambiar(float**,int,int);
	//Inversión de bits
	#define SWAP(x,y) do {typeof(x) _x = x;typeof(y) _y = y;x = _y;y = _x;} while(0)
	//Variables para obtener tiempo de ejecución
	clock_t inicio, final;
	double total;
#endif
\end{lstlisting}
fft.c:
\begin{lstlisting}[style=CStyle]
#include"funciones.h"
int main(int argc, char *argv[]){
	//Leo las cabeceras
	leerCabeceras(argv);
	//Defino variables
	total_muestras_originales=subchunk2size/blockalign;
	printf("Total muestras originales:%d\n",total_muestras_originales);
	//Necesitamos que el total de muestras sea una potencia de 2
	total_muestras=calcularNuevoNumeroMuestras(total_muestras_originales);
	printf("Nuevo total de muestras:%d\n", total_muestras);
	short *muestras=(short *)malloc(total_muestras * sizeof(short));
	//Leo las muestras
	leerMuestras(muestras);
	//Calculo la FFT
	calcularFFT(muestras);
}
void leerCabeceras(char ** argv){
	entrada = fopen(argv[1], "rb");
	salida=fopen(argv[2],"wb");
	if(!entrada){
		perror("\nFile opening failed");
		exit(0);
	}
	fread(&chunkid,sizeof(int),1,entrada);
	fread(&chunksize,sizeof(int),1,entrada);
	fread(&format,sizeof(int),1,entrada);
	fread(&subchunk1id,sizeof(int),1,entrada);
	fread(&subchunk1size,sizeof(int),1,entrada);
	fread(&audioformat,sizeof(short),1,entrada);
	fread(&numchannels,sizeof(short),1,entrada);
	fread(&samplerate,sizeof(int),1,entrada);
	fread(&byterate,sizeof(int),1,entrada);
	fread(&blockalign,sizeof(short),1,entrada);
	fread(&bitspersample,sizeof(short),1,entrada);
	fread(&subchunk2id,sizeof(int),1,entrada);
	fread(&subchunk2size,sizeof(int),1,entrada);
}
void leerMuestras(short *muestras){
	int i=0;
	while (feof(entrada) == 0){
		if(i<total_muestras_originales){
			fread(&muestra,sizeof(short),1,entrada);
			muestras[i]=muestra;
			i++;
		}else{
			fread(&headers,sizeof(short),37,entrada);
			break;
		}
	}
	//Ajuste por si las muestras originales no fueron potencia de dos
	if(total_muestras_originales<total_muestras){
		for (i = total_muestras_originales; i < total_muestras; i++){
			muestras[i]=0;
		}
	}
	fclose(entrada);
}
void escribirArchivo(short* muestrasRe,short* muestrasIm){
	//Escribo el archivo
	fwrite(&chunkid,sizeof(int),1,salida);
	fwrite(&chunksize,sizeof(int),1,salida);
	fwrite(&format,sizeof(int),1,salida);
	fwrite(&subchunk1id,sizeof(int),1,salida);
	fwrite(&subchunk1size,sizeof(int),1,salida);
	fwrite(&audioformat,sizeof(short),1,salida);
	fwrite(&numchannels,sizeof(short),1,salida);
	fwrite(&samplerate,sizeof(int),1,salida);
	fwrite(&byterate,sizeof(int),1,salida);
	fwrite(&blockalign,sizeof(short),1,salida);
	fwrite(&bitspersample,sizeof(short),1,salida);
	fwrite(&subchunk2id,sizeof(int),1,salida);
	fwrite(&subchunk2size,sizeof(int),1,salida);
	//Ahora escribo las muestras
	int i=0;
	for(i=0;i<total_muestras;i++){
		fwrite(&muestrasRe[i],sizeof(short),1,salida);
		fwrite(&muestrasIm[i],sizeof(short),1,salida);
	}
	//Y por último los headers de goldwave
	for(i=0;i<37;i++){
		fwrite(&headers[i],sizeof(short),1,salida);
	}
	fclose(salida);
}
void calcularFFT(short *muestras){
	//Aquí va el algoritmo para la FFT
	float *Xre=(float *)malloc(total_muestras * sizeof(float));
	float *Xim=(float *)malloc(total_muestras * sizeof(float));
	int i;
	//Convierto las muestras de short a float
	convertirFloat(muestras, Xre, Xim);
	//Iniciar relog
	inicio = clock();
	//FFT
	int j, k, fk, m, n, ce, c, w;
	float arg, seno, coseno, tempr, tempi;
	//Bit reversal
	m=log((float)total_muestras)/log(2.0);
	j=w=0;
	for (i = 0; i < total_muestras; i++){
		if (j>i){
			SWAP(Xre[i],Xre[j]);
			SWAP(Xim[i],Xim[j]);
		}
		w=total_muestras/2;
		while(w>=2 && j>=w){
			j-=w;
			w>>=1;
		}	
		j+=w;
	}
	ce=m;
	c=0;
	//Mariposas
	for (i = 0; i < m; i++) {
		for(j = 0; j < (int)pow(2,ce-1); j++){
			n = (int)pow(2,i);
			for(k = 0; k < n; k++){
				fk=k*(int)pow(2,ce-1);
				coseno=cos((-1)*2*PI*fk/total_muestras);
				seno=sin((-1)*2*PI*fk/total_muestras);
				tempr=Xre[c+n];
				Xre[c+n]=(Xre[c+n]*coseno) - (Xim[c+n]*seno);
				Xim[c+n]=(Xim[c+n]*coseno) + (tempr*seno);
				tempr=(Xre[c]+Xre[c+n])/2;
				tempi=(Xim[c]+Xim[c+n])/2;
				Xre[c+n]=(Xre[c]-Xre[c+n])/2;
				Xim[c+n]=(Xim[c]-Xim[c+n])/2;
				Xre[c]=tempr;
				Xim[c]=tempi;
				c++;
			}
			c += n;
		}
		c = 0;
		ce -= 1;
	}
	short *Reales=(short *)malloc(total_muestras * sizeof(short));
	short *Imaginarias=(short *)malloc(total_muestras * sizeof(short));
	//Obtener tiempo e imprimir
	final = clock();
	total = (double)(final - inicio) / CLOCKS_PER_SEC;
	printf("Tiempo de ejecucion: %f\n", total);
	//Regreso las muestras calculadas a short
	convertirShort(Reales,Imaginarias,Xre,Xim);
	//La salida ahora sera un archivo tipo estereo (2 canales)
	//Por lo cual hay que cambiar el numero de canales del archivo 
	//y todas las demas cabeceras que dependan de esta
	chunksize-=subchunk2size;
	numchannels*=2;
	byterate*=numchannels;
	blockalign*=numchannels;
	subchunk2size=total_muestras*blockalign;
	chunksize+=subchunk2size;
	escribirArchivo(Reales,Imaginarias);
}
int calcularNuevoNumeroMuestras(int total){
	if ((total & (total-1))==0){
		puts("Ya es potencia de 2");
	}else{
		puts("No es potencia de 2");
		int i;
		i=(int)ceil((float)log(total_muestras_originales)/(float)log(2));
		printf("i:%d\n", i);
		total=pow(2,i);
	}
	return total;
}
void convertirFloat(short *muestras, float *Xre, float *Xim){
	int i;
	for (i = 0; i < total_muestras; i++){
		Xre[i]=(float)muestras[i]/(float)(SHRT_MAX);
		Xim[i]=0.0;
	}
}
void convertirShort(short *Reales, short *Imaginarias, float *Xre, float *Xim){
	int i;
	for (i = 0; i < total_muestras; i++){
		Reales[i]=Xre[i]*(SHRT_MAX);
		Imaginarias[i]=Xim[i]*(SHRT_MAX);
	}
}
\end{lstlisting}
\newpage
ffti.c:
\begin{lstlisting}[style=CStyle]
#include"funciones.h"
int main(int argc, char *argv[]){
	//Leo las cabeceras
	leerCabeceras(argv);
	//Defino variables    
	total_muestras=subchunk2size/blockalign;
	printf("Total muestras %d\n",total_muestras);
	short *muestrasRe=(short *)malloc(total_muestras * sizeof(short));
	short *muestrasIm=(short *)malloc(total_muestras * sizeof(short));
	//Leo las muestras
	leerMuestras2Canales(muestrasRe,muestrasIm);
	//Calculo la TDF
	calcularFFTI(muestrasRe,muestrasIm);
}
void leerCabeceras(char ** argv){
	entrada = fopen(argv[1], "rb");
	salida=fopen(argv[2],"wb");
	if(!entrada){
		perror("\nFile opening failed");
		exit(0);
	}
	fread(&chunkid,sizeof(int),1,entrada);
	fread(&chunksize,sizeof(int),1,entrada);
	fread(&format,sizeof(int),1,entrada);
	fread(&subchunk1id,sizeof(int),1,entrada);
	fread(&subchunk1size,sizeof(int),1,entrada);
	fread(&audioformat,sizeof(short),1,entrada);
	fread(&numchannels,sizeof(short),1,entrada);
	fread(&samplerate,sizeof(int),1,entrada);
	fread(&byterate,sizeof(int),1,entrada);
	fread(&blockalign,sizeof(short),1,entrada);
	fread(&bitspersample,sizeof(short),1,entrada);
	fread(&subchunk2id,sizeof(int),1,entrada);
	fread(&subchunk2size,sizeof(int),1,entrada);
}
void leerMuestras2Canales(short *muestrasRe,short* muestrasIm){
	int i=0;
	while (feof(entrada) == 0){
		if(i<total_muestras){
			fread(&muestrasRe[i],sizeof(short),1,entrada);
			fread(&muestrasIm[i],sizeof(short),1,entrada);
			i++;
		}else{
			fread(&headers,sizeof(short),37,entrada);
			break;
		}
	}
}
void escribirArchivo(short* muestrasRe,short* muestrasIm){
	//Escribo el archivo
	fwrite(&chunkid,sizeof(int),1,salida);
	fwrite(&chunksize,sizeof(int),1,salida);
	fwrite(&format,sizeof(int),1,salida);
	fwrite(&subchunk1id,sizeof(int),1,salida);
	fwrite(&subchunk1size,sizeof(int),1,salida);
	fwrite(&audioformat,sizeof(short),1,salida);
	fwrite(&numchannels,sizeof(short),1,salida);
	fwrite(&samplerate,sizeof(int),1,salida);
	fwrite(&byterate,sizeof(int),1,salida);
	fwrite(&blockalign,sizeof(short),1,salida);
	fwrite(&bitspersample,sizeof(short),1,salida);
	fwrite(&subchunk2id,sizeof(int),1,salida);
	fwrite(&subchunk2size,sizeof(int),1,salida);
	//Ahora escribo las muestras
	int i=0;
	for(i=0;i<total_muestras;i++){
		fwrite(&muestrasRe[i],sizeof(short),1,salida);
		fwrite(&muestrasIm[i],sizeof(short),1,salida);
	}
	//Y por último los headers de goldwave
	for(i=0;i<37;i++){
		fwrite(&headers[i],sizeof(short),1,salida);
	}
}
void calcularFFTI(short *Re, short *Im){
	//Aquí va el algoritmo para la FFT
	int i;
	float *Xre=(float *)malloc(total_muestras * sizeof(float));
	float *Xim=(float *)malloc(total_muestras * sizeof(float));
	for (i = 0; i < total_muestras; i++){
		Xre[i]=(float)Re[i]/(float)SHRT_MAX;
		Xim[i]=(float)Im[i]/(float)SHRT_MAX;
	}
	//Inicio relog
	inicio = clock();
	//FFTI
	int j, k, fk, m, n, ce, c, w;
	float arg, seno, coseno, tempr, tempi;
	//Bit reversal
	m=log((float)total_muestras)/log(2.0);
	j=w=0;
	for (i = 0; i < total_muestras; i++){
		if (j>i){
			SWAP(Xre[i],Xre[j]);
			SWAP(Xim[i],Xim[j]);
		}
		w=total_muestras/2;
		while(w>=2 && j>=w){
			j-=w;
			w>>=1;
		}
		j+=w;
	}
	ce=m;
	c=0;
	//Mariposas
	for (i = 0; i < m; i++) {
		for(j = 0; j < (int)pow(2,ce-1); j++){
			n = (int)pow(2,i);
			for(k = 0; k < n; k++){
				fk=k*(int)pow(2,ce-1);
				coseno=cos(2*PI*fk/total_muestras);
				seno=sin(2*PI*fk/total_muestras);
				tempr=Xre[c+n];
				Xre[c+n]=(Xre[c+n]*coseno) - (Xim[c+n]*seno);
				Xim[c+n]=(Xim[c+n]*coseno) + (tempr*seno);
				tempr=(Xre[c]+Xre[c+n]);
				tempi=(Xim[c]+Xim[c+n]);
				Xre[c+n]=(Xre[c]-Xre[c+n]);
				Xim[c+n]=(Xim[c]-Xim[c+n]);
				Xre[c]=tempr;
				Xim[c]=tempi;
				c++;
			}
			c += n;
		}
		c = 0;
		ce -= 1;
	}
	//Obtener tiempo e imprimir
	final = clock();
	total = (double)(final - inicio) / CLOCKS_PER_SEC;
	printf("Tiempo de ejecucion: %f\n", total);
	short *Reales=(short *)malloc(total_muestras * sizeof(short));
	short *Imaginarias=(short *)malloc(total_muestras * sizeof(short));
	convertirShort(Reales,Imaginarias,Xre,Xim);
	//Escribo el resultado en el archivo
	escribirArchivo(Reales,Imaginarias);
}
void convertirFloat(short *muestras, float *Xre, float *Xim){
	int i;
	for (i = 0; i < total_muestras; i++){
		Xre[i]=(float)muestras[i]/(float)SHRT_MAX;
		Xim[i]=0.0;
	}
}
void convertirShort(short *Reales, short *Imaginarias, float *Xre, float *Xim){
	int i;
	for (i = 0; i < total_muestras; i++){
		Reales[i]=Xre[i]*SHRT_MAX;
		Imaginarias[i]=Xim[i]*SHRT_MAX;
	}
}
\end{lstlisting}
