\section{Introducción}
	Los autómatas son programas comúnmente utilizados para reconocer lenguajes regulares (indica si una palabra pertenece o no a un lenguaje). Estos están compuestos de un conjunto de estados terminales, no terminales, un alfabeto, un estado inicial y una función de transferencia la cual podría considerarse como el control con la cual el autómata sabe como desplazarse dentro de sus estados.\\Para la labor de reconocer lenguajes regulares, se pueden usar los automatas finitos deterministas (AFD) y los automatas finitos no deterministas (AFND), siendo la principal diferencia entre ellos la cantidad de estados en los que se puede encontrar un autómata en un determinado momento. Para los AFD solo es posible encontrarse en un estado en todo momento mientras que para los AFND es posible estar en más de un estado en un determinado tiempo.\\Esto le permite a los AFND ser más eficientes en el reconocimiento de lenguajes regulares, sin embargo, los AFD son más eficientes en tiempo de procesamiento, ya que los AFND para ser implementados requieren el uso de funciones recursivas o de hilos, mientras que los AFD pueden ser implementados mucho más fácilmente.\\
	Al final, el autómata entrega como resultado sólo dos posibles estados; Aceptado o no aceptado, esto refiriéndose a la cadena que recibió de entrada.\\
	El siguiente trabajo, consiste en la implementación de una clase AFD y otra AFND (en el lenguaje Python) las cuales sean capaces de realizar la validación de cadenas que sean validas para cierto autómata, el cual será cargado mediante un archivo, el cual contendrá los 5 componentes que todo autómata debe tener mencionados arriba.
	