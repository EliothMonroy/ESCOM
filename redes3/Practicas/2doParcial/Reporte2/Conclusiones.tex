\chapter{Conclusiones}
\noindent
\textbf{Hernández Pineda Miguel Angel:}\\
En el desarrollo de esta practica se trabajaron con dos algoritmos que nos permiten obtener la información cuando se presenta una falla, esto se puede traducir como el momento en que el comportamiento del objeto que se está monitorizando presenta un comportamiento anómalo, sin embargo la detección de dicho comportamiento varía de acuerdo con la lectura de los datos, pues si los datos recibidos tienen un comportamiento lineal se utilizan algunos algoritmos basados en predicción de datos mientras que los datos que tienen un comportamiento no lineal se utilizan algoritmos de muestreo y detección de errores.En esta ocasión los algoritmos utilizados fueron el de Mínimos cuadrados para el comportamiento lineal y Holt-Winters para el comportamiento no lineal. En cuestiones de implementación el algoritmo de Holt-Winters es más complicado debido a todos los parametros necesarios que deben definirse sobre la marcha para que el muestreo de la información sea satisfactorio y no se identifiquen errores donde no los hay por lo que, basados en su complejidad en la implementación, podemos decir que es bastante robusto y efectivo con los parámetros correctos.
\noindent
\\
\textbf{Monroy Martos Elioth:}\\
Considero que el desarrollo de esta práctica se fortalecieron distintas habilidades y hubo que hacer uso de varios conocimientos, sin embargo lo más importante desde mi perspectiva, es la importancia que hay detrás de un sistema de detección de fallas. Temas como mínimos cuadrados, métodos de linea base, método de Holt-Winters, e incluso una estrategia para la propuesta de umbrales para la detección de fallas, componen en conjunto una de las partes con mayor importancia funcional en nuestra herramienta. Además de que se desarrollaron varios scripts que resultan ser útiles para todo tipo de situaciones como el detector de umbrales programado, el logger, el notificador o el graficador. Los cuales pueden ser scripts que nos faciliten la vida en un futuro.\\
\noindent
\\
\textbf{Zuñiga Hernández Carlos:}\\
La práctica tuvo como principal objetivo la predicción del rendimiento y de fallas de los recursos disponibles en una computadora, así como la detección de fallas y el monitoreo del rendimiento en tiempo real. La predicción se implementó por medio de los algoritmos Mínimos cuadrados y Holt-Winters, cada uno con diferentes características; aunque Holt-winters demostró ser más preciso y eficiente, debido a que hacía uso de datos históricos para hacer la predicción. Es claro que es necesario monitorizar la actividad en los dispositivos que conforman una red, para que cuando se presente una falla, se pueda actuar a tiempo para resolverla o reducir el impacto y que todo el proceso que conlleva esta detección es complejo y debe hacerse de manera detallada.