\chapter{Códigos}
\section{Examen 1}
A continuación, se muestran los códigos elaborados para la realización del examen.

El archivo con el que se inicia la ejecución del programa, es con main.py, el cual solo crea una instancia del objeto Gestor, el cual se encuentra dentro del archivo gestor.py. Esto para ajustarnos al desarrollo orientado a objetos.\newline
main.py:
\lstinputlisting[language=Python]{src/primero/main.py}

El archivo gestor.py, es el que maneja toda la interfaz principal.\newline
gestor.py:
\lstinputlisting[language=Python]{src/primero/gestor.py}
Este archivo, maneja todo lo referente a los agentes.\newline
agente.py:
\lstinputlisting[language=Python]{src/primero/agente.py}
Este archivo, permite el registro de un nuevo agente.\newline
agregar\_agente.py:
\lstinputlisting[language=Python]{src/primero/agregaragente.py}
Ejemplo de como se almacena la información sobre los agentes, para la persistencia.\newline
agentes.json:
\lstinputlisting[language=Java]{src/primero/agentes.json}
Permite monitorear a un agente y además realiza la consultas correspondientes para mostrar las gráficas.\newline
monitor.py:
\lstinputlisting[language=Python]{src/primero/monitor.py}
Archivo en donde se encuentran todas las consultas necesarias para el correspondiente monitoreo.\newline
SNMP.py:
\lstinputlisting[language=Python]{src/primero/SNMP.py}
Posteriormente, para la detección de umbrales, se uso el siguiente código.\newline
detector\_umbrales.py:
\lstinputlisting[language=Python]{src/primero/detectorumbrales.py}
Para enviar los correos electrónicos de notificación cuando se sobrepasa un umbral se uso el código siguiente.\newline
notificador.py:
\lstinputlisting[language=Python]{src/primero/notificador.py}
Al igual que se envía una notificación, también se escribe en un archivo de log, el cual es controlado por el siguiente script.\newline
logger.py:
\lstinputlisting[language=Python]{src/primero/logger.py}

\section{Examen 2}
En este examen, se hizo uso de un script hecho en python el cual permite graficar un archivo rrd dado, el script usado es el siguiente.\newline
Graficador.py:
\lstinputlisting[language=Python]{src/segundo/Graficador.py}
Además se hizo uso del siguiente script que permite actualizar un archivo rrd con información nueva.\newline
P2Ejercicio5.py:
\lstinputlisting[language=Python]{src/segundo/P2Ejercicio5.py}
Y por último, se realizaron algunas modificaciones a los programas de notificación y logger que se mostraron en la sección anterior, el código se puede ver a continuación.\newline
notificador.py:
\lstinputlisting[language=Python]{src/segundo/notificador.py}
logger.py:
\lstinputlisting[language=Python]{src/segundo/logger.py}