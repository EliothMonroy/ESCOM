\chapter{Conclusiones}
\noindent
\textbf{Hernández Pineda Miguel Angel:}\\
Como pudimos observar, el rendimiento de los servicios varía mucho de acuerdo con las peticiones realizadas a cada uno de estos, no importando el tipo de servidor que se ocupe ya sea de información como HHTP o de archivos como FTP. Parte importante de ésta práctica fue el aprender a configurar diferentes tipos de servidores para entender su funcionamiento, además de que se monitoreó el rendimiento del mismo de acuerdo con las peticiones realizadas, ya que durante el ejercicio se montaron todos los servidores en una misma computadora, lo que nos permitió supervizar el rendimiento del CPU, la RAM y el HDD de la misma y las variaciones que se presentaban de acuerdo con el número de peticiones que se realizaban a la misma. Finalmente, uno de los ejercicios era llevar a cabo pruebas de estrés que nos permitieran conocer el limite de rendimiento de nuestros servicios, sin embargo este ejercicio no pudimos realizarlo debido a la falta de conocimientos sobre el control de las peticiones a los servidores mencionados.
\noindent
\\
\textbf{Monroy Martos Elioth:}\\
Para el desarrollo de está práctica, fue necesario investigar bastante para poder montar cada uno de los servicios solicitados, siendo el servicio de correo el que más trabajo costó, mientras que http y ssh fueron los dos más sencillos, en la práctica, montar un servidor de http puede ser posible de muchas formas, sin embargo, nos resultó más sencillo implementarlo en python, ya que es un entorno que conocemos. Además, el monitorear cada uno de estos servidores, de igual forma resultó complicado, ya que no resulta tan trivial con en un momento llegamos a pensar, se requiere tener conocimiento de cada uno de los servidores para saber que variables medir y como hacerlo, finalmente. el hacer las pruebas bajo estrés resulto aún más complicado ya que no teníamos conocimiento de que herramientas podíamos usar para poder llevar a cabo esta tarea.
\clearpage
\noindent
\textbf{Orta Cisneros Sabrina:}\\
\noindent
El desarrollo de esta práctica ayudó a conocer el rendimiento en situaciones ideales de diferentes servicios que un servidor puede tener, así como para saber cómo funcionan los protocolos que definen a cada uno de estos servicios sumamente importantes para nuestras actividades diarias al navegar por la web. Es muy conveniente monitorizar estos servicios pues así podemos saber, por ejemplo, cuántas solicitudes de cuántos clientes puede soportar y de esta manera evitar futuras fallas y caídas del servicio pues como se mencionó anteriormente, éstos son indispensables para las diversas acciones que se realizan a través de Internet.
\\
\textbf{Ramírez Centeno Hugo Enrique:}\\
\noindent
Los servicios en red son aplicaciones que se encuentran ejecutándose en diferentes equipos y prestan su servicio a través de la red, por lo que es importante poder instalar estos servicios y además monitorizarlos, lo cual fue el objetivo planteado al principio de esta práctica, mismo, el que cumplimos. Cabe mencionar que cada servicio tiene su complejidad para ser instalado, ya que dependiendo de la plataforma sea windows, linux o macOS aumentan la cantidad de pasos para instalar el servicio, otro factor influyente son los usuarios con permisos en los diferentes sistemas operativos, llegamos a la conclusión que es mas sencillo trabajar con un sistema operativo linux.
En la parte de monitoreo, fue esencial un trabajo colaborativo para entender cada servicio y poder realizar una herramienta en python que pudiera monitorizar todo al mismo tiempo.
\\
\textbf{Saldaña Aguilar Andrés Arnulfo:}\\
El monitoreo del rendimiento de los diferentes servicios en el servidor que dimos de alta fue sencillo gracias al uso de las bibliotecas que Python tiene, ya que tienen muchas funcionalidades ya implementadas para darles el propósito que se busca, el único problema que tuvimos fue hacer las pruebas de estrés, que consideramos serian sencillas de realizar y no nos dimos el tiempo necesario para implementarlas, por lo que no pudimos someter los servicios a pruebas para poder tener pruebas mas realistas del rendimiento de nuestros servicios.
\noindent
\\
\textbf{Zuñiga Hernández Carlos:}\\
El desarrollo de la práctica implicó familiarizarnos con distintos servicios, con la finalidad de poder medir su rendimiento en la siguiente práctica. Cada servicio contó con mediciones diferentes del desempeño de algunos de sus recursos y eso necesitó de una investigación sobre librerías de Python que nos facilitaran las conexiones a dichos servicios y la obtención de los índices devueltos por sus recursos. También, se notó el impacto de estos servicios en el desempeño del equipo en el que se encontraban alojados, lo cual es bastante útil para notar cuándo es necesario optimizar un servicio o cuándo está sucediendo un ataque o mal funcionamiento. No es trivial definir cuando estos servicios han alcanzado su límite o algún nivel de riesgo, pero la práctica sirve de mucho para adentrarnos en la monitorización de distintos servidores.
%